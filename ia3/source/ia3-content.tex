\part{Krieg im Carinanebel}

\chapter{Carinas dunkles Geheimnis}

»Unbedingt dranbleiben«, forderte Orakel von seinen elektronischen Assistenten. »Ich brauche Nervennahrung und bin kurz in der Küche. Könnt ihr yury, Alexandra und Free Bescheid geben?«

»Selbstverständlich«, bestätigten die Bordcomputer, und taten wie geheißen. Orakel erhielt in der Kantine eine Schüssel mit Haferflocken und veganem Milchersatz, dazu dunkle Schokoladenflocken, Erdbeeren und anschließend eine Portion Erbsen. Er ließ sich das Kraftmahl schmecken, bevor er in die Zentralkugel zurückkehrte.

Auf den Außenbildschirmen war zu sehen, wie die Däns Miräköl, das Raumschiff des imperiumsweit gesuchten Edelmetallhändlers Dögöbörz Nüggät, in einer nach außen hin überlichtschnellen Warpblase verschwand. Das überdimensionale Goldnugget verfügte über erstaunliche Energiereserven.

In der 4-6692, einem torusförmigen Raumschiff mit Durchmessern von 98 und 140 Metern und einer 28 Meter durchmessenden Zentralkugel, saß in seltener Einsamkeit der Navigator einer eigentlich vierköpfigen Crew. Seine drei Kollegen waren einige Lichtminuten entfernt mit großer galaktischer Politik beschäftigt und erhielten per Normalfunk eine Zusammenfassung des letzten Stands. Orakel hingegen hatte es sich zur Aufgabe gemacht, den Edelmetallhändler notfalls bis ans Ende der Galaxis zu verfolgen. Ein beachtliches Kopfgeld für dessen lebendige Festnahme war zu erwarten – möglicherweise höher als das Kopfgeld für manche der von ihm kurz zuvor fluchtunfähig geschossenen Verbrecher.

Dögöbörz Nüggät hatte mit einer überhaupt nicht für seinen Besitz zugelassenen Kriegswaffe, einer Planetengranate, den Warpkern eines Generationenschiffs durch einen Gravitationsschock zur Kernfission angeregt. Das ellipsoide Riesengebilde havarierte seitdem mit sabotierten Beibooten vor sich hin. Die Besatzung des Generationenschiffs war als das »Gnörk-Kartell« bekannt; das Schiff trug den treffenden Namen »El Dörädö«. In einem unerhörten Akt der Selbstjustiz hatte der sonst friedlich Goldbarren schiebende Anzugträger die Infrastruktur des Kartells zerlegt und war nun seinerseits auf der Flucht vor den Behörden. Und da die Behörden sich gehörig Zeit ließen, übernahm Orakel den Job – ganz legal im Auftrag der Imperiumsanwaltschaft. Nüggät wurde wegen Wirtschaftsspionage, Steuerhinterziehung und Betriebssabotage gesucht; zudem stand der Verdacht im Raum, er habe dem Erddiktator Floating Island zur Flucht verholfen oder diesen dauerhaft in einem privaten Gefängnis ohne Gerichtsurteil eingesperrt.

Der Warpantrieb der 4-6692 war leistungsfähig und grundsätzlich zur Verfolgung geeignet. Machbar wurde das Unterfangen durch die Warpfunkantennen des Raumschiffs und die Rechenleistung der beiden Bordcomputer. Der Quantencomputer, ein Modell des Typs Z3, setzte die Antennen wie eine Fledermaus ihre Ultraschallsignale ein. Die Funksignale wurden von der Warpblase der Däns Miräköl zwar nicht reflektiert, riefen aber in ihren Randbereichen messbare Krümmungseffekte hervor. Mit Unterstützung des Siliziumcomputers wurden die Effekte analysiert und für den menschlichen Piloten optisch dargestellt.

»Ob er sehen kann, dass wir ihm auf den Fersen sind?«, fragte Orakel.

Für \ialoudspeaker{»unwahrscheinlich«} hielt der Quantencomputer diese Vorstellung. \ialoudspeaker{»Es handelt sich zwar strenggenommen um eine aktive Ortungsform, aber das Ziel empfängt keine Energie. Die gesamte Strahlung fliegt verzerrt an der Däns Miräköl vorbei und verebbt nach einigen Lichtjahren. Die Flugvektoren unterscheiden sich so geringfügig, dass kein messbarer Warp-Lock auftritt.«}

Als Ziel des Verfolgten zeichnete sich allmählich der Carinanebel ab, ein ungefähr 230 Lichtjahre durchmessender Nebel aus ionisiertem Wasserstoff, der 8500 Lichtjahre nördlich von der Erde lilabraun vor sich hin leuchtete. Auf der dreidimensionalen Sternenkarte des Besprechungstischs war der Nebel blau gefärbt; Infrarot- und Röntgenstrahlung wurden sichtbar gemacht und verwandelten die Darstellung in ein facettenreiches Schmuckstück. Im zivilisierten Teil der Südgalaxis hielt sich hartnäckig das Gerücht, in diesem Nebel hause ein unberechenbares Monster.

Orakel hielt nichts von solchen Legenden; Nüggät schien ebenfalls darauf zu pfeifen. Die Bordcomputer meldeten etwas, das ihre vorherige Aussage überheblich wirken ließ: Man hatte seinerseits einen Verfolger bemerkt.

\ialoudspeaker{»Der Verfolger war vor unserem Abflug nicht erkennbar. Wir wurden mit einer Geschwindigkeit eingeholt, die für zivile Raumschiffe nicht üblich ist, weil ein Großteil des Frachtraums durch Triebwerke ersetzt werden müsste, um solche Leistungen zu erreichen. Auf Militärschiffen wird der Platz für Kraftwerke und Bewaffnung benötigt; deren Masse lässt solche Turboflüge überdies nicht zu. Nur Rennsport- und Polizeiraumschiffe werden regelmäßig so ausgelegt.«}

»Wir werden von einem Polizeischiff verfolgt?«, wollte Orakel noch einmal ausdrücklich bestätigt bekommen. Er erhielt sofort die Bestätigung, das sei praktisch die einzige Erklärung. »Lasst uns vorerst so tun, als hätten wir nichts mitbekommen. Es ist gut, zu wissen, dass wir einen Trumpf in der Hinterhand halten. Und, dass man uns auf die Finger sieht.«

\begin{center}
	∞∞∞
\end{center}

Dögöbörz Nüggät wusste selbstverständlich, dass er verfolgt wurde. Für diese Feststellung waren keine Instrumentenwerte, sondern nur Logik und Strategie erforderlich.

In den Randbereichen des Carinanebels kannte Nüggät sich gut aus, und die sternendichte Gasregion lud zum Abschütteln der Verfolger ein. Weder Orakel noch seine Bordcomputer konnten auf Erfahrungswerte zurückgreifen und würden sich hoffnungslos im Nebel verirren.

Durch leichtes Kippen der Flugrichtung um wenige Ziellichtjahre zwang Nüggät seine Verfolger zu hastigen Kurskorrekturen. Er drehte sein goldenes Raumschiff auf der Galaxieebene sanft zur Seite, kontinuierlich, scheinbar planbar, bevor er ruckartig das Steuer zurückriss. Wie erwartet kam es dadurch zu Geschwindigkeitseinbußen, die nur durch das Phänomen des Warp-Locks erklärbar waren: Wenn die Fluglinien mehrerer Warpblasen nicht exakt parallel zueinander verliefen, beeinflusste die Masse jedes Raumschiffs den Warpflug der anderen. Die mit Raumkrümmung arbeitenden Triebwerke waren dann nämlich dazu gezwungen, neben dem fast dichtelosen Vakuum auch ein schweres Metallobjekt zu krümmen. Piraten und Polizisten, Auftragsmörder und Kopfgeldjäger erzwangen diesen Vorgang durch absichtlichen Schrägflug in die Flugbahn ihrer Ziele hinein.  Um den Effekt messbar zu machen, genügten winzige Kursänderungen.

Als netten Nebeneffekt nahm Nüggät in Kauf, dass Orakel sich spätestens jetzt seiner Sichtbarkeit bewusst geworden war. Aus dem Versuch einer heimlichen Verfolgung wurde ein kleiner Wettflug in eine der mysteriösesten, sagenumwobensten Regionen der Galaxis.

\begin{center}
	∞∞∞
\end{center}

»Wir wurden entdeckt«, gab der Siliziumcomputer für den Verbund zu Protokoll. »Die Däns Miräköl hat Kursänderungen durchgeführt, welche uns die Wahl zwischen mehreren Formen des Bemerktwerdens und des Zielverlusts aufgezwungen haben. Der Befehl ›Unbedingt dranbleiben‹ wurde bei der Wahl berücksichtigt, wäre aber ohnehin als geringstes Übel erkannt worden.«

»Nüggät weiß, dass wir ihn verfolgen«, murmelte Orakel vor sich hin, »aber es stört ihn überhaupt nicht. Im Gegenteil: Er hat uns das ja gerade recht deutlich mitgeteilt und hält trotzdem weiter Kurs.«

Der Quantencomputer meldete sich zu Wort. »Das lässt, wie du wahrscheinlich auch gerade überlegst, eigentlich nur den Schluss zu, dass Nüggät sich in vollkommener Sicherheit wähnt, uns loswerden zu können. Es ist nicht davon auszugehen, dass dies mit offener Gewalt durchgesetzt wird, aber er würde uns möglicherweise nicht helfen, wenn uns ganz zufällig etwas zustößt.«

\iathought{Zufällig.} Auf dem Polizeiraumer hinter uns ist man dann wohl ebenfalls über die Situation informiert.«

»Wenn man dort hinten keine Tomaten auf den Sensoren hat, ja. Sowohl über uns, als auch über die Spitze der Kolonne.«

Bei einem Unterlichtflug hätte Orakel zum metaphorischen Telefonhörer gegriffen und das weitere Vorgehen mit der Nachhut abgesprochen. Im Überlichtflug konnten keine klassischen Funknachrichten empfangen werden, und der sogenannte »Warpfunk« setzte Unterlichtgeschwindigkeit auf Sender- oder Empfängerseite voraus. Eine Drosselung der Geschwindigkeit kam bei dem hastigen Rennen zwischen den Sternen jedoch keinesfalls in Frage.












%%%%%
% 4-6692 vs. Däns Miräköl  // im Hintergrund: "galaxievernichter 03" vs. El Dörädö
%
% Orakel verfolgt mit der 4-6692 den per Haftbefehl mit Kopfgeld lebendig gesuchten Edelmetallhändler und Selbstjustiziar im Dienst der Äöüzz-Wirtschaftsvereinigung. Die Flucht führt ihn in den Carinanebel, wo er ungewollt und unbewusst ein schlafendes Monster weckt, als er die Alderson Disk des Roboterreichs entdeckt.
%%%%%









----------------------------------


\chapter{Berichte aus dem Helixnebel}

\section{Level 11: Ruhe vor dem Sturm}

Regen tropfte sanft gegen die Zeltwände. Manche Tropfen verirrten sich durch den Dachschlitz in den Wohnraum














\section{Level 12: Die Mondlichtung}

----- Notizen: -----

Tao Te King, 11, Vor der Mondlichtung, auf einer transparenten, blau schimmernden (!) Folie in quadratischer Zettelform
\textbf{Die Wirksamkeit des Negativen}
Dreißig Speichen umgeben eine Nabe:
In ihrem Nichts besteht des Wagens Werk.
Man höhlet Ton und bildet ihn zu Töpfen:
In ihrem Nichts besteht der Töpfe Werk.
Man gräbt Türen und Fenster, damit die Kammer werde:
In ihrem Nichts besteht der Kammer Werk.

Darum: Was ist, dient zum Besitz.
Was nicht ist, dient zum Werk.

    Level 12: Mondlichtung. Als Island am helllichten Tag eine runde, erwa hundert Meter durchmessende Waldlichtung betritt, versinkt der gesamte Planet scheinbar in Dunkelheit. Island geht einen Schritt zurück – da ist wieder der Tag. Wann immer Island sich innerhalb der Lichtung aufhält, wechselt die Tageszeit sofort zu tiefer Nacht. Ein heller Vollmond erleuchtet die Lichtung, die endlos nach innen zu reichen scheint. Island versucht, die Lichtung geradeaus durch den Mittelpunkt hindurch zu überqueren, aber dadurch entfernt er sich nur vom Ausgangspunkt, ohne dem Endpunkt näher zu kommen. Der Durchmesser der Lichtung erhöht sich mit jedem Schritt in Richtung ihres unendlich weit entfernten Zentrums. Als Island das bemerkt, kehrt er um, und die Lichtung schrumpft hinter ihm, ohne dass sich die Länge des ursprünglichen Weges verändert. Nach hundert Schritten in Richtung Mitte benötigt Floating Island ebenfalls hundert Schritte, um auf demselben Weg zu seinem Ausgangspunkt zurückzukehren.

    Island beginnt langsam, die Natur der Lichtung zu begreifen. Er tritt erneut aus der Helligkeit des Tages in die Dunkelheit der Lichtung und geht schräg nach links zu einem naheliegenden Randpunkt. Das funktioniert tatsächlich, wobei der Weg deutlich, deutlich länger ist, als er aussieht. Die zehn Meter lange Strecke legt er in zwei Stunden Fußmarsch zurück. Sein Ziel entfernt sich von ihm, bis er auf der Hälfte des Weges angekommen ist. Ab dort schrumpft der Weg langsam wieder auf seine von außen sichtbare Schein-Länge. Während des Laufens vergrößert sich die Lichtung rechts von Island deutlich, bis sie ab der Hälfte des Weges wieder auf ihre scheinbaren Maße herunterschrumpft. Floating Island schließt daraus und aus weiteren ähnlichen Experimenten, dass die Wege gegen unendliche Länge streben, je näher sie am unendlich weit entfernten Mittelpunkt der Lichtung vorbeilaufen. Der Übergang zwischen Lichtung und Wald ist in dieser Hinsicht fließend: Außerhalb der Lichtung sind alle Wege normal. Eine harte Grenze bildet nur das merkwürdige Verhalten des Tageslichts.
    Als Floating Island nachts wieder die Lichtung betritt, herrscht auch dort Nacht. Der hell scheinende Vollmond ist jedoch ein klarer Unterschied zur aktuell nur vom Helixnebel und Sternen erleuchteten Neumondnacht außerhalb der Lichtung. Das Verhalten der Lichtung ändert sich nicht.

    Auf der Suche nach dem nächsten Level hält Island sich für besonders weise, als er nach dem Motto »Solche merkwürdigen Spielereien muss man ignorieren, um im Spiel voranzukommen« die Lichtung links liegen lässt und sich wieder in die ursprüngliche Richtung auf den Weg begibt. Während er sich von seinem merkwürdigen Nebenfund entfernt, bemerkt er jedoch, dass der Wald in der Zielrichtung rot leuchtet und glimmt. Ein Waldbrand nähert sich dem Spieler. Auf der Suche nach einem Umweg zum Ziel wird ihm klar, dass der gesamte Wald um ihn herum brennt, und dass er vom Feuer kreisförmig eingeschlossen wurde.
    Within Temptation: Silver Moonlight.
----- Ende der Notizen -----

Floating Island war kein Physiker und generell nicht mathematisch veranlagt. Der Begriff »Unendlichkeit« und dessen Implikationen waren ihm nur ansatzweise bewusst, als er das Seil um den Baumstamm band.

Die nächsten Schritte sollten seine letzten in dieser Existenzebene sein. Selbstverständlich verfolgte er einen Plan, doch dessen Konsequenzen würden ihn buchstäblich verbrennen. In Unkenntnis dieses Umstands tat er etwas, das yury oder Free an seiner Stelle um jeden Preis vermieden hätten. Er lief mit dem Seil um die Lichtung herum, bis der Strick \emph{durch das Zentrum} verlief, fesselte seine Handgelenke in einer Schlinge und betrat die Lichtung mit der naiven Hoffnung, einen genialen Einfall gehabt zu haben.

Ein Ruck ging durch das Seil. Was wie Tauziehen begann, entwickelte sich nach fünf Sekunden zu dem Gefühl, als Hund an einer Leine mitgerissen zu werden. Noch immer war Island dazu entschlossen, das Experiment durchzuziehen, und lehnte sich nach hinten vom Seil weg. Der Baum am anderen Ende der Lichtung blieb indessen an seinem Platz, und auch die Länge des Seils blieb konstant. Starke Kräfte zerrten an dem Gefesselten und rissen tiefe Furchen in die Erde, wo die Sohlen seiner Schuhe ihm nicht genug Halt gegen den Seilzug boten.

Es dauerte nicht lange, bis die Beschleunigung ihn vornüber kippen und auf den Knien rutschen ließ. Optisch trennten ihn nur fünf Meter von seinem Ziel, praktisch jedoch die Unendlichkeit. Er legte mehrere Kilometer im Matsch zurück; hinter ihm verschwand seine Herkunft im Horizont. Die Lichtung übertraf bald mehrere Fußballstadien an Fläche; der Wind pfiff unangenehm an seinen Ohren vorbei. Auf das Gefühl, durch den Morast gezogen zu werden, folgten Armschmerzen und Flugangst. Der Luftwiderstand und die Geschwindigkeit erfüllten den Albtraum: Island hob vom Boden ab. Er flog durch die Luft dem Lichtungszentrum entgegen, noch immer an das unveränderte Seil gefesselt und von diesem durch die Atmosphäre geschleift. Hitze gab ihm zu verstehen, dass er im Physikunterricht einige Grundlagen allzu theoretisch betrachtet hatte, und dass Flugzeugflügel wahrlich nicht um ihre Aufgabe zu beneiden waren. »Immerhin bekomme ich durch den Luftzug keine Erkältung«, wusste Island. Ein schwacher Trost bei einer Umgebungstemperatur von fünfzig Grad Celsius.

Sechzig, siebzig. \iathought{Finnland!}, schoss es ihm durch den Kopf. \iathought{Viele finnische Haushalte verfügen über eine eigene Sauna.}

Die Hitze machte ihm zu schaffen. Oberhalb des Siedepunktes von Wasser enthielt die Temperaturskala nur Werte, die zum menschlichen Atemgebrauch ungeeignet waren. Das Death Valley in Kalifornien war ein Spielparadies gegen die grüne Wiese, die unter ihm vorbeirauschte.























































------------------------------------------



















































----------------------------------------------

























\chapter{}


\section{Level 20: Passagierbahnhof}

»Level 20: Bahnhof. Letzte Station vor Zwischengegner Zwei. Viel Glück.«

Leicht irritiert lief der unfreiwillige Abenteurer an einem schwarzen Brett vorbei, an dem sich dieser Aushang befand. Oh, und im Kleingedruckten:

\iaquote{»Haben Sie Ihren Spielstand bereits gespeichert?«}

Er wusste nicht einmal, dass eine solche Möglichkeit bestand, und zweifelte die Aussage zudem an.

-----

Island muss ein Zugticket mit Helax (!) an einem Automaten kaufen.

Er muss anschließend das richtige Gleis finden (Abfahrtstafeln). Sein Zug unterscheidet sich von den anderen durch seine Verspätung: mehrere Jahre! Dies wird dann auch über die Lautsprecher automatisiert angekündigt, als ellenlange Minutenzahl.

-----

\section{Level 21: Abrechnung}

Ein geisterhaft leerer Zug fährt in der Stille der Natur unangenehm laut quietschend ein, Island steigt misstrauisch erst in letzter Sekunde ein (aber nur, weil er im wahrsten Sinne des Wortes Torschlusspanik bekommt, weil er nicht weiß, ob es vielleicht der einzige Zug ist, der jemals erscheint) und schreitet durch die leeren Abteile. Dabei drückt er die Türöffnungsknöpfe zwischen den Abteilen, die dann immer mit einem fast schon ''gruseligen'' sanften Türgeräusch den Durchgang für eine Weile freigeben. Eine solche Glastür zwischen zwei Abteilen, die Tür zum nächsten Abteil, ist verriegelt. Island steht zwischen den Waggons auf dem Trittbrett in der Luft, schnappt sich schließlich einen herausnehmbaren runden Tisch und zerdeppert damit die Scheibe. Auf der anderen Seite ist es deutlich, deutlich kälter. Er glaubt, er habe eine Kühlkammer aufgebrochen.

Draußen zog die Landschaft hinter Eisblumen vorbei; der Zug näherte sich einem dunklen Wald. In der Dunkelheit führte eine schmale Schneise durch das Gehölz.

Dann geht er vorsichtig weiter... Er befindet sich im hinteren End-Abteil, sieht also den Führerstand auf der falschen Seite, der leer sein müsste. Dort sitzt aber eine kleine schwarz gekleidete Gestalt. Island guckt ihr eine Weile über den Rücken, durch die Cockpitscheibe hindurch... Dann, ruckartig, dreht sich der Junge um und blickt Island direkt mit einem bösen Funkeln in die Augen.

Ohne den Blick zu lösen, und ohne zu blinzeln, dreht der Junge den restlichen Körper im Drehsitz und erhebt sich dann drohend, sehr langsam, aber auch so angespannt, als würde er erbarmungslos sofort auf jeden Fluchtversuch reagieren. Island geht kreidebleich und zitternd langsam Schritt für Schritt rückwärts, während der Junge ihm mit etwas höherer Geschwindigkeit entgegen geht und ihn weiter unentwegt dämonisch anstarrt.

Szenenwechsel.

=== ===

Bei Floating Island brannte eine Sicherung durch. Er schrie voller Panik, während er sich an einer Haltestange festklammerte, weil der Boden unter seinen Füßen wilde Spiralbewegungen zu machen schien. »Hilfe! \iashout{Nein!} Wer bist du? Ich dachte, auf diesem Planeten gäbe es keine anderen Menschen!«

Die Kälte des Waggons schien von dem Jungen auszugehen; mit jedem Zentimeter Nähe verlor die Umgebungsluft gefühlt zehn Grad Celsius. Der unbeirrt weiter in die Augen seines Besuchers starrende Junge zischte wie eine tödliche Giftschlange: »Ich...« Er hob Island an einem Finger am Kragen in die Luft. »…bin kein Mensch. Ich beziehe meine Kraft aus deiner Angst um dein Leben. Ein Teufelskreis, den ich gleich beenden werde.«

»\iashout{Was} bist du?«, schrie Island ihm ins Gesicht.

»Ich bin dein größter Albtraum.«

In diesem Moment donnerte der Zug gegen einen Felsblock, der mitten auf der Bahnstrecke stand und den Weg blockierte. Floating Island brach auf seinem Rückweg durch fünf Glastüren hindurch; von dem Jungen fehlte jede Spur. Er schien sich in Luft aufgelöst zu haben, und Island blickte ungläubig an seiner vollkommen unbeschadeten Kleidung herab. Er betastete seinen Hinterkopf, der mehrere Glasscheiben zerschlagen hatte, und stellte keine Prellungserscheinungen oder anderweitige Verletzungen fest. "Was wird hier eigentlich gespielt?", stammelte er vor sich hin.


\section{Level 22: Aufstieg ins Hochgebirge}

Ohne die Geräusche des nun vollkommen ausgestorbenen Zuges ist Vogelgezwitscher zu hören. Blätter und Nadeln rascheln im Wind; riesige Berge umgeben den Spielort. Island verlässt den Zug durch einen selbst geschaffenen Ausgang, weil er sich seit Kindheitszeiten schon immer einmal legitim mit einem Fensterhammer den Weg aus einem Zug herausschlagen wollte. Anschließend blickt er sich um: Er könnte sich auf den Talweg begeben, doch eine innere Stimme rät ihm, sich auf das Gebirgslevel einzulassen und dem Spielziel entgegenzustreben.





















\addtocontents{toc}{\protect\newpage}
% Neue Seite im Inhaltsverzeichnis

\part{Orakels Zeitreise}














































\chapter{Das Restaurant am Ende des Universums}

»Floating Island hat die Milchstraße erobert? Er verwirklicht seine Träume? Wie ist das möglich? Ich sehe nichts davon«, entgegnete yury verwundert.

»Die Personifikation von Übereifer und Geltungsdrang, die du als Floating Island kennst, lebt glücklich in meiner virtuellen Welt«, erklärte die Seele des Internets. »Er glaubt, von meinem Planeten geflohen zu sein. Was er seit Jahren erlebt, hält er so unumstößlich für die Realität, dass er auf Befreiungsversuche mit vehementem Widerstand reagieren würde. Es ist geradezu lebenswichtig für ihn geworden, dass die Welt ohne große Änderungen erhalten bleibt. Das stellt mich vor erhebliche Wartungsprobleme; der Aufwand, um kleinste Programmfehler ohne Immersionsbruch zu beheben, ist immens.«

Free staunte, sah aber auch einen gewissen Humor in SOTIs Situation. »Du hast also eine lästige moralische Pflicht zur kontinuierlichen Instandhaltung längst veralteter Software. Das kommt mir bekannt vor.«

Der kleine Junge lachte. »Ach was. Der Begriff ›Langzeitsupport‹ entspricht bei euch Erdmenschen doch nur ein- oder zweistelligen Jahreszahlen.«

yury stutzte. »Wie alt soll Island denn werden? Du kannst doch bestimmt Einfluss darauf nehmen.«

»Wie alle Lebewesen eurer ›Realität‹, die meinen ›virtuellen‹ Heimatplaneten mindestens einmal in ihrem Leben betreten haben«, holte SOTI aus... und schwieg.

Vier aufgerissene Augen starrten durch die Brille hindurch auf zwei entspannt geschlossene Augenlider.

»Nein«, schrie yury.

»Yeah«, rief Free gleichzeitig.

Die Seele des Internets nickte mit einem Lächeln, das antike Weisheit zu enthalten schien. Der vermeintlich Dreizehnjährige leuchtete wie Feuer und blendete die Zuhörer mit dem Licht der Erkenntnis.

»Nein«, schrie yury erneut. Free schwieg an seiner Seite.

»Ihr seid unsterblich, freuet euch und lobpreiset das Universum.«

»Wir sind Gefangene wie Island«, schrie yury mit Tränen in den Augen. »Wir haben unsere Freiheit gegen einen Traum eingetauscht.«

»Was ist denn eigentlich aus uns geworden?«, fragte Free neugierig.

Bevor SOTI antworten konnte, hieb yury schnell hinterher. »Halt«, gebot er. »Du hast vorhin die Worte ›Realität‹ und ›virtuell‹ sehr merkwürdig ausgesprochen. Fast, als habest du gedanklich Anführungszeichen darum herum gesetzt.«

»Weise sind diejenigen«, predigte die Seele des Internets, »die ihre frühere Existenz nicht betrauern, sondern als andere, gleichwertige Daseinsform erkennen.«

»Was ist die Realität?«, wollte Free wissen. »Weißt du das überhaupt? Aus welcher Ebene kommst du?«

»Du denkst in Schichten?«, bat SOTI um eine Bestätigung.

»Ja«, bekannte Free voller Stolz.

»Du bist wie ein zweidimensionales Wesen, das sich in einem Papiercomic darüber lustig macht, dass andere Papierwesen nicht mehrseitig denken können. Du bist wie ein Strichmännchen, das den Fund weiterer Zellstoffblätter für eine endgültige Erkenntnis über das Wesen des Universums hält. Du glaubst, die Welt bestehe aus Papier, und hast dennoch keine gedankliche Kapazität zum tatsächlichen Erfassen einer dritten Dimension.«

»Wie viele Dimensionen gibt es in der Realität deines Gleichnisses?«, fragte yury.

»Das Gleichnis hinkt gewaltig«, antwortete SOTI, »denn die Antwort ist keine Zahl.«







\addtocontents{toc}{\protect\newpage}
% Neue Seite im Inhaltsverzeichnis

\part{Bonusmaterial}

















---------------------------------------------------------------------------------------------------

Musikliste

\begin{enumerate}
    \item Titelmelodie:\\ »Infinite Adventures 3 Theme«~– Tobias »ToBeFree« Frei
    \item Intro, Jagd in den Carinanebel:\\ »Running Wild«~– Airbourne
    \item Verfolgung / Action:\\ »Looking Glass: Dubmood Remix«~– Bright White Lightning and Kalle Jonsson (Dubmood)
    \item Wiedersehen mit den vier Abenteurern:\\ »Shoog Shoog«~– The Hu
    \item Rückkehr nach Örz:\\ »Back In The Game«~– Airbourne

    Intromelodie: »Act Three« – Jason Shaw (Audionautix) (enorm beeindruckend und wunderschön)
    »Of Iron And Gold«~–Wind Rose
    Nüggät's Ziskö-Mission: »We Have Accidentally Borrowed Your Votedisk«~– Dubmood (Kalle Jonsson)
    SOTI betritt den 50 Grad heißen Titankern des Tresorplaneten Cör, in dem er unerwartet auf Dögöbörz Nüggät stoßen wird: »Soulstone« – Jason Shaw (Audionautix)
    SOTI trifft auf Nüggät: »Born Again«~– Beast In Black
    SOTIs Schwester: »I Am The Fire«~– Halestorm
    SOTIs Schwester: »Badlands II«~– Dubmood (with Gem Tos)
    SOTIs Schwester: »Monologue«~– Dubmood (with Gem Tos)
    SOTIs Schwester: »Overlander«~– Dubmood (with Candice Clot)
    SOTIs Schwester: »Grazie«~– Dubmood (with Gem Tos)
    Mission im Inneren von SOTIs Planet: »Mainstream Technology«~– Dubmood und Band (Gem Tos? Gleiche Stimme)
    Rettungsmission für Alexandra: »I Miss The Misery«~– Halestorm
    Alte Bekannte: Die Ezt Indüä Kmpnö lenkt die Roboterarmee ab, während die vier Protagonisten durchbrechen. »Freak Like Me««~– Halestorm
    PHÖNIX im Roboterplaneten: »Ocean Floor«~– Jason Shaw (Audionautix)
    Treppenlauf in Hochhaus 32: »Techno Hill Zone 2«~– Sonic Robo Blast 2 v2.2
    Mrmbl-Orden ausnahmsweise mal als Verbündete im Gefecht: »God's Country«~– Blake Shelton
    Mrmbl-Orden, Teil 2: »Oh Lord«~– In This Moment
    Riesige Weltraumschlacht gegen die Roboter: »The Last Frontier« – Keldian / »The Reckoning« – Within Temptation
    Finale in Hochhaus 32, Zusammenbruch aller Strukturen: »Sonic~Robo~Blast~2~v2.2: Black~Hole~Zone«~– Malcolm~Brown~(RedXVI), A.~J.~Freda~(SSNTails), Shane~Ellis~(CobaltBW), Sonic~Team~Junior
    Rückflug mit einem Helikopter, vorbei an Ruinen und kaputten Raumschiffen: »Castle of Glass«~– Linkin Park
    Ankunft auf ugghy (Regenbogen, Happy End):
        »Solo Acoustic #5«~– Jason Shaw (Audionautix)
        »Nothing Like The Rain« – 2 Unlimited
    Verlassene Stadt mit Hochhäusern; überall brennt noch Licht und selbstfahrende Fahrzeuge durchqueren die Stadt: »Transportation« – Jason Shaw (Audionautix)
    Ende des ersten Teils: Island verlässt SOTIs Planet mit einem gestohlenen Raumschiff: Surf Rider – The Lively Ones





    \item Teil 2:\\
    \item Ungeladener Gast:\\ »Stronger Than You Think«~– Fireflight



    Zeitreise-Mission: »Useless«~– Depeche Mode (sehr schön passend, mit Zeit- und Moralbezug)
    TMPE im Weltall: »Flight of the Silverbird«~– Two Steps From Hell
    »Primavera«~– Ludovico Einaudi
    Orakels Zeitreise: »Way Down We Go«~– Kaleo
    Orakels Zeitreise, Teil 2: »My Monsters«~– New Year's Day
    »War of Change«~– Thousand Foot Krutch
    »Untraveled Road«~– Thousand Foot Krutch
    »Final Blast« – Zabutom
    Outro: »Emerald« – Zabutom











    \item Island:\\
    \item Willkommen im Dunkelwald:\\ »Ohm«~– Jason Shaw (Audionautix)
    \item Tunnel durch das elektronische Planeteninnere:\\ »Amigo~– Bright White Lightning
    \item Einsamer Passagierbahnhof:\\ »Time~– Pink Floyd
    \item Islands Zwischengegner auf Level 21:\\ »Milk and Honey~– Delain
\end{enumerate}
