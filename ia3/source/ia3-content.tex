\chapter{Berichte aus dem Helixnebel}

\section{Level 11: Ruhe vor dem Sturm}

Regen tropfte sanft gegen die Zeltwände. Manche Tropfen verirrten sich durch den Dachschlitz in den Wohnraum














\section{Level 12: Die Mondlichtung}

----- Notizen: -----

Tao Te King, 11, Vor der Mondlichtung, auf einer transparenten, blau schimmernden (!) Folie in quadratischer Zettelform
\textbf{Die Wirksamkeit des Negativen}
Dreißig Speichen umgeben eine Nabe:
In ihrem Nichts besteht des Wagens Werk.
Man höhlet Ton und bildet ihn zu Töpfen:
In ihrem Nichts besteht der Töpfe Werk.
Man gräbt Türen und Fenster, damit die Kammer werde:
In ihrem Nichts besteht der Kammer Werk.

Darum: Was ist, dient zum Besitz.
Was nicht ist, dient zum Werk.

    Level 12: Mondlichtung. Als Island am helllichten Tag eine runde, erwa hundert Meter durchmessende Waldlichtung betritt, versinkt der gesamte Planet scheinbar in Dunkelheit. Island geht einen Schritt zurück – da ist wieder der Tag. Wann immer Island sich innerhalb der Lichtung aufhält, wechselt die Tageszeit sofort zu tiefer Nacht. Ein heller Vollmond erleuchtet die Lichtung, die endlos nach innen zu reichen scheint. Island versucht, die Lichtung geradeaus durch den Mittelpunkt hindurch zu überqueren, aber dadurch entfernt er sich nur vom Ausgangspunkt, ohne dem Endpunkt näher zu kommen. Der Durchmesser der Lichtung erhöht sich mit jedem Schritt in Richtung ihres unendlich weit entfernten Zentrums. Als Island das bemerkt, kehrt er um, und die Lichtung schrumpft hinter ihm, ohne dass sich die Länge des ursprünglichen Weges verändert. Nach hundert Schritten in Richtung Mitte benötigt Floating Island ebenfalls hundert Schritte, um auf demselben Weg zu seinem Ausgangspunkt zurückzukehren.

    Island beginnt langsam, die Natur der Lichtung zu begreifen. Er tritt erneut aus der Helligkeit des Tages in die Dunkelheit der Lichtung und geht schräg nach links zu einem naheliegenden Randpunkt. Das funktioniert tatsächlich, wobei der Weg deutlich, deutlich länger ist, als er aussieht. Die zehn Meter lange Strecke legt er in zwei Stunden Fußmarsch zurück. Sein Ziel entfernt sich von ihm, bis er auf der Hälfte des Weges angekommen ist. Ab dort schrumpft der Weg langsam wieder auf seine von außen sichtbare Schein-Länge. Während des Laufens vergrößert sich die Lichtung rechts von Island deutlich, bis sie ab der Hälfte des Weges wieder auf ihre scheinbaren Maße herunterschrumpft. Floating Island schließt daraus und aus weiteren ähnlichen Experimenten, dass die Wege gegen unendliche Länge streben, je näher sie am unendlich weit entfernten Mittelpunkt der Lichtung vorbeilaufen. Der Übergang zwischen Lichtung und Wald ist in dieser Hinsicht fließend: Außerhalb der Lichtung sind alle Wege normal. Eine harte Grenze bildet nur das merkwürdige Verhalten des Tageslichts.
    Als Floating Island nachts wieder die Lichtung betritt, herrscht auch dort Nacht. Der hell scheinende Vollmond ist jedoch ein klarer Unterschied zur aktuell nur vom Helixnebel und Sternen erleuchteten Neumondnacht außerhalb der Lichtung. Das Verhalten der Lichtung ändert sich nicht.

    Auf der Suche nach dem nächsten Level hält Island sich für besonders weise, als er nach dem Motto »Solche merkwürdigen Spielereien muss man ignorieren, um im Spiel voranzukommen« die Lichtung links liegen lässt und sich wieder in die ursprüngliche Richtung auf den Weg begibt. Während er sich von seinem merkwürdigen Nebenfund entfernt, bemerkt er jedoch, dass der Wald in der Zielrichtung rot leuchtet und glimmt. Ein Waldbrand nähert sich dem Spieler. Auf der Suche nach einem Umweg zum Ziel wird ihm klar, dass der gesamte Wald um ihn herum brennt, und dass er vom Feuer kreisförmig eingeschlossen wurde.
    Within Temptation: Silver Moonlight.
----- Ende der Notizen -----

Floating Island war kein Physiker und generell nicht mathematisch veranlagt. Der Begriff »Unendlichkeit« und dessen Implikationen waren ihm nur ansatzweise bewusst, als er das Seil um den Baumstamm band.

Die nächsten Schritte sollten seine letzten in dieser Existenzebene sein. Selbstverständlich verfolgte er einen Plan, doch dessen Konsequenzen würden ihn buchstäblich verbrennen. In Unkenntnis dieses Umstands tat er etwas, das yury oder Free an seiner Stelle um jeden Preis vermieden hätten. Er lief mit dem Seil um die Lichtung herum, bis der Strick \emph{durch das Zentrum} verlief, fesselte seine Handgelenke in einer Schlinge und betrat die Lichtung mit der naiven Hoffnung, einen genialen Einfall gehabt zu haben.

Ein Ruck ging durch das Seil. Was wie Tauziehen begann, entwickelte sich nach fünf Sekunden zu dem Gefühl, als Hund an einer Leine mitgerissen zu werden. Noch immer war Island dazu entschlossen, das Experiment durchzuziehen, und lehnte sich nach hinten vom Seil weg. Der Baum am anderen Ende der Lichtung blieb indessen an seinem Platz, und auch die Länge des Seils blieb konstant. Starke Kräfte zerrten an dem Gefesselten und rissen tiefe Furchen in die Erde, wo die Sohlen seiner Schuhe ihm nicht genug Halt gegen den Seilzug boten.

Es dauerte nicht lange, bis die Beschleunigung ihn vornüber kippen und auf den Knien rutschen ließ. Optisch trennten ihn nur fünf Meter von seinem Ziel, praktisch jedoch die Unendlichkeit. Er legte mehrere Kilometer im Matsch zurück; hinter ihm verschwand seine Herkunft im Horizont. Die Lichtung übertraf bald mehrere Fußballstadien an Fläche; der Wind pfiff unangenehm an seinen Ohren vorbei. Auf das Gefühl, durch den Morast gezogen zu werden, folgten Armschmerzen und Flugangst. Der Luftwiderstand und die Geschwindigkeit erfüllten den Albtraum: Island hob vom Boden ab. Er flog durch die Luft dem Lichtungszentrum entgegen, noch immer an das unveränderte Seil gefesselt und von diesem durch die Atmosphäre geschleift. Hitze gab ihm zu verstehen, dass er im Physikunterricht einige Grundlagen allzu theoretisch betrachtet hatte, und dass Flugzeugflügel wahrlich nicht um ihre Aufgabe zu beneiden waren. »Immerhin bekomme ich durch den Luftzug keine Erkältung«, wusste Island. Ein schwacher Trost bei einer Umgebungstemperatur von fünfzig Grad Celsius.

Sechzig, siebzig. \iathought{Finnland!}, schoss es ihm durch den Kopf. \iathought{Viele finnische Haushalte verfügen über eine eigene Sauna.}

Die Hitze machte ihm zu schaffen. Oberhalb des Siedepunktes von Wasser enthielt die Temperaturskala nur Werte, die zum menschlichen Atemgebrauch ungeeignet waren. Das Death Valley in Kalifornien war ein Spielparadies gegen die grüne Wiese, die unter ihm vorbeirauschte.























































------------------------------------------

\chapter{Carinas dunkles Geheimnis}

»Unbedingt dranbleiben«, forderte Orakel von seinen elektronischen Assistenten. »Ich brauche Nervennahrung und bin kurz in der Küche. Könnt ihr yury, Alexandra und Free Bescheid geben?«

»Selbstverständlich«, bestätigten die Bordcomputer, und taten wie geheißen. Orakel erhielt in der Kantine eine Schüssel mit Haferflocken und veganem Milchersatz, dazu dunkle Schokoladenflocken, Erdbeeren und anschließend eine Portion Erbsen. Er ließ sich das Kraftmahl schmecken, bevor er in die Zentralkugel zurückkehrte.







%%%%%
% 4-6692 vs. Däns Miräköl  // im Hintergrund: "galaxievernichter 03" vs. El Dörädö
%
% Orakel verfolgt mit der 4-6692 den per Haftbefehl mit Kopfgeld lebendig gesuchten Edelmetallhändler und Selbstjustiziar im Dienst der Äöüzz-Wirtschaftsvereinigung. Die Flucht führt ihn in den Carinanebel, wo er ungewollt und unbewusst ein schlafendes Monster weckt, als er die Alderson Disk des Roboterreichs entdeckt.
%%%%%


















































----------------------------------------------

























\chapter{}


\section{Level 20: Passagierbahnhof}

»Level 20: Bahnhof. Letzte Station vor Zwischengegner Zwei. Viel Glück.«

Leicht irritiert lief der unfreiwillige Abenteurer an einem schwarzen Brett vorbei, an dem sich dieser Aushang befand. Oh, und im Kleingedruckten:

\iaquote{»Haben Sie Ihren Spielstand bereits gespeichert?«}

Er wusste nicht einmal, dass eine solche Möglichkeit bestand, und zweifelte die Aussage zudem an.

-----

Island muss ein Zugticket mit Helax (!) an einem Automaten kaufen.

Er muss anschließend das richtige Gleis finden (Abfahrtstafeln). Sein Zug unterscheidet sich von den anderen durch seine Verspätung: mehrere Jahre! Dies wird dann auch über die Lautsprecher automatisiert angekündigt, als ellenlange Minutenzahl.

-----

\section{Level 21: Abrechnung}

Ein geisterhaft leerer Zug fährt in der Stille der Natur unangenehm laut quietschend ein, Island steigt misstrauisch erst in letzter Sekunde ein (aber nur, weil er im wahrsten Sinne des Wortes Torschlusspanik bekommt, weil er nicht weiß, ob es vielleicht der einzige Zug ist, der jemals erscheint) und schreitet durch die leeren Abteile. Dabei drückt er die Türöffnungsknöpfe zwischen den Abteilen, die dann immer mit einem fast schon ''gruseligen'' sanften Türgeräusch den Durchgang für eine Weile freigeben. Eine solche Glastür zwischen zwei Abteilen, die Tür zum nächsten Abteil, ist verriegelt. Island steht zwischen den Waggons auf dem Trittbrett in der Luft, schnappt sich schließlich einen herausnehmbaren runden Tisch und zerdeppert damit die Scheibe. Auf der anderen Seite ist es deutlich, deutlich kälter. Er glaubt, er habe eine Kühlkammer aufgebrochen.

Draußen zog die Landschaft hinter Eisblumen vorbei; der Zug näherte sich einem dunklen Wald. In der Dunkelheit führte eine schmale Schneise durch das Gehölz.

Dann geht er vorsichtig weiter... Er befindet sich im hinteren End-Abteil, sieht also den Führerstand auf der falschen Seite, der leer sein müsste. Dort sitzt aber eine kleine schwarz gekleidete Gestalt. Island guckt ihr eine Weile über den Rücken, durch die Cockpitscheibe hindurch... Dann, ruckartig, dreht sich der Junge um und blickt Island direkt mit einem bösen Funkeln in die Augen.

Ohne den Blick zu lösen, und ohne zu blinzeln, dreht der Junge den restlichen Körper im Drehsitz und erhebt sich dann drohend, sehr langsam, aber auch so angespannt, als würde er erbarmungslos sofort auf jeden Fluchtversuch reagieren. Island geht kreidebleich und zitternd langsam Schritt für Schritt rückwärts, während der Junge ihm mit etwas höherer Geschwindigkeit entgegen geht und ihn weiter unentwegt dämonisch anstarrt.

Szenenwechsel.

=== ===

Bei Floating Island brannte eine Sicherung durch. Er schrie voller Panik, während er sich an einer Haltestange festklammerte, weil der Boden unter seinen Füßen wilde Spiralbewegungen zu machen schien. »Hilfe! \iashout{Nein!} Wer bist du? Ich dachte, auf diesem Planeten gäbe es keine anderen Menschen!«

Die Kälte des Waggons schien von dem Jungen auszugehen; mit jedem Zentimeter Nähe verlor die Umgebungsluft gefühlt zehn Grad Celsius. Der unbeirrt weiter in die Augen seines Besuchers starrende Junge zischte wie eine tödliche Giftschlange: »Ich...« Er hob Island an einem Finger am Kragen in die Luft. »…bin kein Mensch. Ich beziehe meine Kraft aus deiner Angst um dein Leben. Ein Teufelskreis, den ich gleich beenden werde.«

»\iashout{Was} bist du?«, schrie Island ihm ins Gesicht.

»Ich bin dein größter Albtraum.«

In diesem Moment donnerte der Zug gegen einen Felsblock, der mitten auf der Bahnstrecke stand und den Weg blockierte. Floating Island brach auf seinem Rückweg durch fünf Glastüren hindurch; von dem Jungen fehlte jede Spur. Er schien sich in Luft aufgelöst zu haben, und Island blickte ungläubig an seiner vollkommen unbeschadeten Kleidung herab. Er betastete seinen Hinterkopf, der mehrere Glasscheiben zerschlagen hatte, und stellte keine Prellungserscheinungen oder anderweitige Verletzungen fest. "Was wird hier eigentlich gespielt?", stammelte er vor sich hin.


\section{Level 22: Aufstieg ins Hochgebirge}

Ohne die Geräusche des nun vollkommen ausgestorbenen Zuges ist Vogelgezwitscher zu hören. Blätter und Nadeln rascheln im Wind; riesige Berge umgeben den Spielort. Island verlässt den Zug durch einen selbst geschaffenen Ausgang, weil er sich seit Kindheitszeiten schon immer einmal legitim mit einem Fensterhammer den Weg aus einem Zug herausschlagen wollte. Anschließend blickt er sich um: Er könnte sich auf den Talweg begeben, doch eine innere Stimme rät ihm, sich auf das Gebirgslevel einzulassen und dem Spielziel entgegenzustreben.























---------------------------------------------------------------------------------------------------

Musikliste

\begin{enumerate}
    \item Titelmelodie:\\ »Infinite Adventures 3 Theme«~– Tobias »ToBeFree« Frei
    \item Intro, Jagd in den Carinanebel:\\ »Looking Glass: Dubmood Remix«~– Bright White Lightning and Kalle Jonsson (Dubmood)
    \item Wiedersehen mit den vier Abenteurern:\\ »Shoog Shoog«~– The Hu
\end{enumerate}
