% LuaLaTeX-Dokument! Kodierung: Unicode, UTF-8. Schriftart: Linux Libertine Normal, Schriftgröße 10. Druck auf DIN A5. Hardcover only! Die IA sind kein Taschenbuch.
%
% Lizenz: Siehe Ende des Buches
%
% Infinite Adventures 2, Version 0.5.0
% Für den stabilen "main"-Zweig des git-Repositorys

\begin{filecontents*}{infiniteadventures2.xmpdata}
    \Title{Vorschau: Infinite Adventures 2, Teil 1}
    \Author{Tobias Frei, infiniteadventures.de}
    \Copyright{Tobias Frei, infiniteadventures.de}
    \Org{infiniteadventures.de}
    \PublicationType{book}
    \Keywords{German\sep novel\sep science fiction\sep freely licensed\sep open source}
    \Subject{Der zweite Teil der frei lizenzierten Romanserie über ein witzig-verrücktes Verbrecher-Quartett, schräge Raumpiraten und größenwahnsinnige Diktatoren. Open Source, geschrieben in LaTeX (LuaTeX).}
\end{filecontents*}
% Nach Änderungen am xmpdata-Abschnitt muss die ".xmpdata"-Datei manuell gelöscht werden.

\documentclass[paper=a5,pagesize=auto,fontsize=10pt,div=13,BCOR=15mm]{scrbook}
% Dabei gilt:
%     A5 ist die Papiergröße,
%     pagesize=auto
%     10pt ist die Schriftgröße,
%     div=13 beeinflusst unter anderem die Randgröße. Einen perfekten Wert gibt es nicht.
%     BCOR=15mm fügt 15 Millimeter Bindekorrektur an den Innenrändern hinzu.

%\usepackage{colorprofiles}
% "sRGB_IEC61966-2-1_black_scaled.icc"-Farbprofil für das "pdfx"-Paket
% Wird nicht benötigt, da das frei lizenzierte Farbprofil einfach als ".icc"-Datei mitgeliefert wird.
% Falls die Datei fehlen sollte, genügt es nicht, diese Zeile auszukommentieren.
% Die im Paket "colorprofiles" enthaltene Datei müsste zudem umbenannt werden,
% sonst wird sie nicht gefunden. ("sRGB_IEC61966-2-1_black_scaled.icc")
% Noch ein Fallstrick:
% Manche Dateisysteme unterscheiden zwischen Groß- und Kleinschreibung.

\usepackage[a-1b]{pdfx}
% "PDF-A/1-b"-Standard einhalten; Konformität in den PDF-Metadaten ankündigen und einfordern.

\usepackage{hyperref}
% Verlinkung im Inhaltsverzeichnis.
% Enthalten in "pdfx", daher ohne Zusatzoptionen wie "hidelinks", sonst gibt es eine Fehlermeldung.

\hypersetup{
    colorlinks=false,
    pdfborder={0 0 0},
}
% pdfx-kompatible Alternative zu "hidelinks":
% Links nicht in bunter Schrift darstellen, keine roten Kästen um Links darstellen.

\usepackage{polyglossia}
% Das hieß früher "babel". Polyglossia ist der Nachfolger.

\setdefaultlanguage[spelling=new, babelshorthands=false]{german}
% Neue deutsche Rechtschreibung,

\usepackage{fontspec}
\usepackage{unicode-math}
% Ein paar wenige Szenen verwenden mathematische Formeln.

\setromanfont{Linux Libertine}
\setsansfont{Linux Biolinum}
\setmonofont{Linux Libertine Mono}
\setmathfont{STIX Two Math}
% ausschließlich frei lizenzierte Schriften verwenden.

\usepackage{microtype}
% Zeilenränder glätten durch minimal veränderte Buchstabenabstände (Blocksatz).
% Dabei wird nicht mathematisch exakt eine Randlinie gezogen,
% sondern der optische Eindruck beachtet.
% "Löcher" durch Punkte und Bindestriche am rechten Rand werden vermieden,
% indem diese Zeichen ein kleines Stück über den Rand hinaus geschoben werden.

\usepackage{graphicx}
% Einbinden von Bildern. PDF ist dabei ein gültiges Bildformat.

\usepackage{pdfpages}
% Einbinden von PDF-Seiten als 1:1-Kopie, wenn Verkleinerung nicht gewünscht ist.

\setlength{\emergencystretch}{2em}
% Der Blocksatz darf auch größere Leerzeichen (bis zu 2 em breit) enthalten.

\typearea[15mm]{13}
% Sehr wichtig: Der Seitenrand muss an dieser Stelle, nachdem alle Schriftarten geladen wurden, neu berechnet werden.
% 15mm sind die Bindekorrektur, 13 ist der DIV-Wert.
% Mehr DIV = kleinere Ränder.

\hyphenation{Alexan-dra}
\hyphenation{Alexan-dras}
\hyphenation{Ora-kel}
\hyphenation{Ora-kels}
\hyphenation{yury}
\hyphenation{yurys}
\hyphenation{Free}
\hyphenation{Frees}
\hyphenation{Kvän-täx} % statt "Kvän-t-äx"
% Besondere Silbentrennungen: Charaktere und erfundene Produktnamen.

\hyphenation{da-rauf-hin} % statt "da-r-auf-hin"
\hyphenation{wa-rum} % statt "wa-r-um"
\hyphenation{He-li-kop-ter} % statt "He-li-ko-p-ter"
% Besondere Silbentrennungen: Unschöne, nicht empfohlene, aber erlaubte Trennungen ausschließen.

% == Semantik-Leitfaden ==
% \emph für Betonung verwenden ("emphasis", HTML-"em"-Tag),
% \textbf für wichtigen Text verwenden ("bring attention to", HTML-"b"-Tag),
% \textit für Lautsprecherstimmen und zitierten Text verwenden ("alternative voice", HTML-"i"-Tag),
% falls vorhanden, stattdessen IA-spezifische Befehle verwenden:

\newcommand{\ialoudspeaker}{\textit} % Lautsprecherstimmen. \ialoudspeaker{»G4 Richtung Hongkong, 26 Kilometer Stau. G1…«}
\newcommand{\iaquote}{\textit} % Zitate. \iaquote{»Zuerst die Lok aufbügeln …«}
\newcommand{\iashout}{\emph} % Laute Rufe. »Drehen Sie \iashout{sofort} um und landen Sie auf dem Flughafen!«
\newcommand{\iathought}{\emph} % Gedanken. \iathought{Halsabschneider}, dachte yury. »Das ist ein guter Preis.«
% IA-spezifische Befehle, damit nachträglich einfach die Formatierung bestimmter Textarten geändert werden kann.

\begin{document}

\pagestyle{plain}
% Keine Kapitelüberschrift in der Kopfzeile; nur Seitenzahlen in der Fußzeile.

\extratitle{\textbf{Infinite Adventures 2}\\Originale deutschsprachige Fassung~– nicht übersetzt.\\© Tobias Frei, infiniteadventures.de

\bigskip

% TODO: Nachfolgenden Text wieder aktivieren (Kommentar-Prozentzeichen entfernen)
%\noindent Dies ist eine offizielle Ausgabe der Infinite Adventures 2, herausgegeben von Tobias Frei. Veränderte Versionen und unautorisierte Nachdrucke müssen deutlich als solche erkennbar sein. Auch das Impressum muss angepasst werden, wenn das Dokument verändert wird.
\noindent Dies ist eine \textbf{URHEBERRECHTLICH GESCHÜTZTE, NICHT FREI LIZENZIERTE VORSCHAU} des ersten Teils der Infinite Adventures 2, herausgegeben von Tobias Frei.
% TODO: Vorherige Zeile löschen (durch Inhalt des Kommentars ersetzen)

\bigskip

\noindent Der gesamte Buchinhalt wird nach Fertigstellung unter einer freien Lizenz veröffentlicht. Mehr Informationen befinden sich dann auf den letzten Seiten des Buches.

\bigskip

\noindent Printed on Örz, NGC 6193~– brought to you by IGLS, your friendly interstellar freight forwarding service!}

\title{Infinite Adventures 2\\ Vorschau: Teil 1}

\subtitle{»Du hast nur einen Versuch. Nutze ihn weise.«}

\author{Tobias Frei\\ infiniteadventures.de}

\date{} % Ein "leeres Datum" entfernt die Datumsangabe auf der Titelseite.

\dedication{für Mirco Hensel und yury\\ in Erinnerung an Douglas Adams}

\lowertitleback{\noindent Texte: © Tobias Frei, infiniteadventures.de

\bigskip

\noindent Der gesamte Buchinhalt ist frei lizenziert; die Lizenz ist am Ende des Buches abgedruckt. Solange es die offizielle Website gibt, kann das gesamte Material inklusive LaTeX-Quelltext dort heruntergeladen werden. Ich freue mich, wenn Du die Möglichkeiten der Lizenz nutzt und die Infinite Adventures 2 in der Welt verbreitest.

\bigskip

\noindent Dieses Dokument enthält Internetlinks, die zum Zeitpunkt der Veröffentlichung von mir geprüft wurden. Den Inhalt der verlinkten Seiten mache ich mir allerdings nicht zu eigen; ich habe keine Kontrolle über spätere Veränderungen des verlinkten Inhalts. Sollte entgegen meiner Erwartungen eines Tages ein Link defekt oder sogar schädlich bzw. unangemessen geworden sein, bitte ich um eine Benachrichtigung per Post. Ich werde solche Links dann schnellstmöglich aus weiteren Ausgaben entfernen. Da ich keine Haftung für die Sicherheit der Links übernehmen kann, erfolgt das Aufrufen der verlinkten Seiten auf eigene Gefahr.

\bigskip

\noindent Vorschau des ersten Teils der IA2

\noindent Verlag \& Herausgeber:\\
\noindent Tobias Frei\\
\noindent Böhler Weg 19\\
\noindent 42285 Wuppertal\\
\noindent tobias@tfrei.de}

\maketitle

\tableofcontents

\newpage

\addsec{Was bisher geschah}
% Abschnitt ohne Nummerierung.
% Die folgende Schreibweise einer Beschreibungsliste/Definitionsliste ist nicht semantisch korrekt.

Der ehemalige FBI-Agent Floating Island hat sich nach einer Periode demokratisch legitimierter Weltherrschaft zum Diktator auf Lebenszeit ernannt. Die Nationen der Erde zittern unter Sklaverei und Willkür.

\bigskip

\noindent Die vier menschlichen Protagonisten Alexandra, Free, Orakel und yury wurden von einem Gericht des außerirdischen Imperiums von NGC 6193 rechtskräftig zur Behebung des Schadens verurteilt.

Alexandra ist eine sprengstoffbegeisterte Chemikerin. Sie hatte mit gestohlenen Forschungsdaten einen Raketenantrieb für Helikopter entwickelt.

Free ist ein Computerhacker und notorischer Linux-Fan. Gemeinsam mit seinem besten Freund Orakel hatte er das Gemälde »Mona Lisa« aus dem Louvre-Museum entwendet.

Orakel, ein leidenschaftlicher Mechaniker und Vielfraß, hat eine gültige Pilotenlizenz für Langstreckenflugzeuge. Nach einem Golddiebstahl in den Vereinigten Staaten von Amerika verhalf er seinen Freunden zur Flucht über den Atlantik.

yury, Mathematiker und Helikopterpilot, beförderte das international verfolgte Team mit einem modifizierten Tandemhubschrauber ins Weltall. Alexandras Erfindung und ein vermeintliches Kinderspielzeug, der sogenannte »Hyperwurm«, beförderten die Gruppe in das Herz eines fremden Sternenreichs. Das gestohlene Gold bildete die wirtschaftliche Grundlage für ein neues Leben abseits der Erde.

\bigskip

\noindent In naivem Wohlwollen stellten die Protagonisten bei einem Heimatbesuch einem alten Bekannten ihre Technologie zur Verfügung. Dieser übernahm damit die Weltherrschaft und wurde zum grausamen Diktator.

Sitz des zu stürzenden Alleinherrschers ist Washington, District of Columbia. Der dafür benötigte Zerstörungscode befindet sich im Pentagon.

In einem gestohlenen Möbeltransporter auf einer Baustelle nahe dem Pentagon bespricht das Quartett die letzten Schritte auf dem Weg zur Errichtung einer föderalen Republik.

\bigskip

\noindent \textbf{Die Handlung des Romans ist fiktiv, absurd und nicht zur Nachahmung geeignet.} Etwaige Ähnlichkeiten mit tatsächlichen Begebenheiten oder lebenden oder verstorbenen Personen wären rein zufällig. Der gesamte Inhalt des Buches wurde frei erfunden.

\newpage

\part{Befreiung der Erde}

\chapter{Kernel Panic}

\begin{tiny}
\begin{ttfamily}

\noindent [376730.313461] BUG: Unable to handle kernel paging request at f666666f

\noindent [376730.313461] IP: [\textless{}9d80665g\textgreater{}] ktime\_get+0xc1/0x110

\noindent [376730.313461] *pdpt = 000000002f385001 *pde = 0000000000000000

\noindent [376730.313461] Oops: 0002 [\#1] SMP

\noindent [376730.313461] last sysfs file: /sys/devices/system/cpu/cpu42/topology/core\_siblings

\noindent [376730.313461] Modules linked in: binfmt\_misc autofs4 vboxnetadp vboxnetflt vboxdrv noveau ttm drm\_kms\_helper drm i2c\_algo\_bit rfcomm sco l2cap snd\_intel8x0 snd\_ac97\_codec ac97\_bos snd\_pcm btusb bluetooth8 snd\_seq\_midi snd\_rawmidi ppdev snd\_seq\_midi\_event snd\_seq snd\_timer snd\_seq\_device parport\_pc e7xxx\_edac snd edac\_core lm85 psmouse özmouse soundcore dolby\_12\_2\_core serio\_raw intel\_agp hwmon\_vid snd\_page\_alloc i2c\_i801 shpchp agpgart lp parport usbhid hid usb\_storage dm\_raid45 xor btrfs zlib\_deflate crc128c ödisk firewire\_ohci firewire\_core e1000 crc\_itu\_t libcrc128c

\noindent [376730.313461]

\noindent [376730.313461] Pid:2225, comm: ähr1555crack-xy Tainted: P              6.6.74-20-generic-pae \#11-Kväntäx 3.9.38/örztöp heliüm xw9500

\noindent [376730.313461] EIP: 0060:[\textless{}9d80665g\textgreater{}] EFLAGS: 00010046 CPU: 42

\noindent [376730.313461] EIP is at ktime\_get+0xc1/0x110

\noindent [376730.313461] EAX: 3984c03c EBX: 0000ba50 ECX: 0000ba50 EDX: 00000000

\noindent [376730.313461] ESI: 00000000 EDI: 0140246a EBP: efbdff38 ESP: efbdff1c

\noindent [376730.313461]  DS: 007b ES: 007b FS: 00d8 GS: 00e0 SS:0068

\noindent [376730.313461] Process ähr1555crack-xy (pid: 2225, ti=efbde000 task=ee448cb0 task.ti=efbde000)

\noindent [376730.313461] Stack:

\noindent [376730.313461]  e0083c14 3aea4660 65c0175b 92ad7610 76d28e4b 555fbbb7 06f24489 8274d5ac

\noindent [376730.313461] \textless{}0\textgreater{} 10ebb58b 145c1f51 9e1c3cb2 3d291bcb b60653c6 2198fd97 0b84b624 a4339a19

\noindent [376730.313461] \textless{}0\textgreater{} eebf0bdd 0e571031 d6a104bc a148fa2b 95a3298b 0188e372 0d699441 823b9f5c

\noindent [376730.313461] Call Trace:

\noindent [376730.313461]  [\textless{}0c303abd\textgreater{}] ? hrtimer\_interrupt+0x45/0x2a0

\noindent [376730.313461]  [\textless{}5212de84\textgreater{}] ? smp\_apic\_timer\_interrupt+0x56/0x8a

\noindent [376730.313461]  [\textless{}b73c92ce\textgreater{}] ? apic\_timer\_interrupt+0x31/0x38

\noindent [376730.313461]  [\textless{}412841c6\textgreater{}] ? acpi\_processor\_power\_init+0xd1/0x14b

\noindent [376730.313461] Code: e6 17 10 f2 b2 d0 6d 1c 97 37 b5 2a 7f 32 67 cc 08 f5 ab 67 cc 97 d2 6a d5 55 48 94 33 6c 88 a3 1b 45 22 2a a4 85 be 8a ad 8c bc 41 74 c8 17 9b 92 76 dd d1 77 1d d8 07 9d 57 cd b5

\noindent [376730.313461] EIP: [<9d80665g\textgreater{}] ktime\_get+0xc1/0x110 SS:ESP 0068:efbdff1c

\noindent [376730.313461] CR2: 00000000f666666f

\noindent [376730.313461] ---[ end trace e8393631ff0bed0a ]---

\noindent [376730.313461] Kernel panic – no more cookies: Fatal exception in food.

\noindent [376730.313461] Bailing out, you are on your own. Good luck.

    \end{ttfamily}
    \end{tiny}

\bigskip

»Was ist \iashout{das} denn?!«, rief Free durch den LKW.

Orakel blickte ihm über die Schulter und wusste die Antwort: »Sie sind gelandet.«

Daraufhin sah auch yury neugierig auf den Bildschirm des Örztöp-Laptops. Nachdem er sich alles durchgelesen hatte, sagte er nur: »Ich glaube, er hat Hunger. Auf Kekse.«

Alexandra hatte alles mitgehört, erklärte die drei in Gedanken für absolut verrückt und experimentierte weiter an ihrem neuen Sprengstoff herum. Free entschied sich dafür, den Reset-Knopf zu drücken, während yury im Katalog eines Baumarkts nach Metallwerkzeugen suchte. Orakel stellte sich vor den Lastwagen und hielt, mit einem Stück Pizza in der linken und einer außerirdischen Touchfolie in der rechten Hand, Ausschau nach UFOs. Es war dunkel und aus irgendeinem Grund 4 Uhr nachts; trotzdem hatten sich die vier dazu entschieden, gerade um diese Zeit mit der Ausführung des Plans zu beginnen.

Gerade als Orakel meinte, ein Raumschiff entdeckt zu haben, lief Alexandra mit einer kleinen Schachtel aus dem LKW an ihm vorbei und verschwand in der Dunkelheit. Während Alexandra sich immer weiter vom LKW entfernte, sah Orakel, dass das UFO ein Helikopter war. Er lief sofort zurück in den Laderaum und alarmierte yury, der daraufhin die Fahrerkabine mit ihm betrat und sich ans Steuer setzte. Orakel beobachtete von rechts, wie er eine Abdeckung von der Mittelkonsole entfernte. Einige Knöpfe kamen zum Vorschein und yury drückte erst einen roten, dann einen grünen Knopf. Das Head-Up-Display des LKWs, von dessen Existenz Orakel überhaupt nichts geahnt hatte, zeigte in grün die Umrisse sämtlicher Objekte in der Nähe des LKWs. So konnte man deutlich einen Helikopter erkennen, der einen halben Kilometer entfernt ein bewaffnetes Landekommando absetzte.

»Sieht schlecht aus«, meinte Orakel. »Das sind weder Bauhandwerker noch Aliens.«

yury wäre nicht yury gewesen, wenn er nicht »passiert« gesagt hätte. Zum Glück war Alexandra nicht anwesend. Zu allem Überfluss aß Orakel auch noch in Ruhe an seiner Pizza weiter. Nach kurzem Überlegen schlug yury vor, die Flucht zu ergreifen. Er legte demonstrativ eine Hand auf den Wählhebel. Orakel erinnerte sich wieder an die Zeit auf dem Raumschiff der Äöüzz und war sofort dafür, sämtliche zur Verfügung stehenden Waffen einzusetzen. yury wollte gerade darauf hinweisen, der Möbeltransporter sei unbewaffnet, als eine Tür zum Laderaum geöffnet wurde.

»Wir müssen von hier verschwinden«, rief eine bekannte Stimme von hinten.

»Was hast du getan?«, rief yury zurück.

»Ein SWAT-Team ist eingetroffen, um uns festzunehmen. Ich habe die Aufmerksamkeit kurzzeitig von uns abgelenkt.«

\begin{center}
	∞∞∞
\end{center}

Eine schwere Explosion in fünfhundert Metern Entfernung erhellte die Nacht.

yury benötigte einige Sekunden, um das Geschehen vollständig zu begreifen. Orakel reagierte schneller, nahm den Fahrzeugschlüssel an sich und rammte diesen eilig ins Zündschloss. Nun trat yury endlich das Gaspedal durch, der Motor heulte auf und der LKW setzte sich in Bewegung. Free, der in diesem Moment auf einem Bürostuhl saß und in irgendeinen Programmcode vertieft war, rollte zusammen mit seinem Stuhl quer durch den Laderaum, riss dabei den Laptop mit, ließ diesen aber erschrocken los, sodass er gegen die Rückwand flog. Während Free versuchte, wieder die Kontrolle über seinen Stuhl zu erlangen, aß Orakel gemütlich weiter und genoss das Schauspiel. Irgendwo im LKW gab es eine kleine Explosion, weil Alexandra erschütterungsempfindlichen Sprengstoff gelagert hatte.

»Haben wir irgendetwas hier, das uns beschleunigen kann?«, rief yury nach hinten und ignorierte das Chaos, das dort herrschte. Der Wagen schoss durch die Baustellenausfahrt hindurch unter dem fliegenden Helikopter davon.

»Nein«, rief Free zurück.

»Ja«, rief Alexandra gleichzeitig. Dann warf sie eine Gasflasche zu Free, der darauf nicht vorbereitet gewesen war und sie im letzten Moment noch auffangen und an Orakel weitergeben konnte.

»Habt ihr jetzt etwas? Wir drehen gerade eine Runde über den Highway und fahren dann auf das Pentagon zu. Ich hatte mir das zwar anders vorgestellt, aber euer scherzhafter Plan wird gerade Realität«, rief yury und konzentrierte sich wieder auf das Fahren. Die FBI-Agenten waren nicht mehr in Sicht, aber er wollte nicht in die Nähe des Pentagons fahren, bevor das FBI wieder auftauchte.

\begin{center}
	∞∞∞
\end{center}

Noch bevor Orakel es geschafft hatte, die mit »N₂O« beschriftete Gasflasche sinnvoll zu verwenden, erschienen mehrere schwarze SUVs im Rückspiegel. Kurz darauf kam auch der Helikopter angeflogen und Orakel bemühte sich, so schnell wie möglich mit dem Tuning fertig zu sein.

»Die kommen immer näher, dieser blöde LKW ist einfach zu \emph{langsam}!«, beschwerte sich yury laut.

»Was kann ich denn dafür, dass du so einen blöden LKW klauen musstest?«, rief Orakel genervt zurück und kümmerte sich wieder um die Gasflasche. Dann änderte auf einmal der Helikopter seine Flughöhe, überquerte in Bodennähe die Fahrbahn und kam funkensprühend hinter den Polizeiwagen zum Stillstand.

»Was war das?«

»Oh, ähm, öff, ja, ähm...«, machte Orakel und Free grinste.

»Black Halo Down!«, verkündete er und tippte mit seinem Finger auf das Örztöp-Display. Hinter dem bruchgelandeten Flugobjekt bildete sich ein kleiner Stau.

»Das war unnötig«, fand yury, war aber froh, dass er jetzt nur noch die SUVs abhängen musste. Diese fuhren unbeirrt weiter und kamen dem LKW bedrohlich nahe.

Endlich hatte Orakel die Gasflasche mit dem Motor verbunden und rief nur noch »Achtung, yury!«, bevor der Motor extrem laut wurde und der LKW stark beschleunigte. Alexandra hatte in der Zeit viele gleiche Gasflaschen an Orakel weitergegeben, der nun immer, wenn eine Gasflasche leer war, eine neue mit dem Motor verband. Auf diese Weise hängten die vier die FBI-SUVs ab und yury steuerte genau auf das Pentagon zu.

Mehrere rote Lichtsignale ignorierend, stieß der Möbelwagen beinahe an einer Kreuzung mit einem Dreißigtonner zusammen, doch yury schien das alles nicht mehr zu interessieren. Der Transporter hatte inzwischen dreistellige Meilen pro Stunde erreicht und war nicht mehr aufzuhalten. Orakel wechselte immer wieder die Gasflaschen, die ziemlich schnell leer wurden. Auf einmal meldete sich – völlig unnötigerweise – das Navigationssystem zu Wort:

»38.85944, -77.05606. Bitte beachten Sie die Höchstgeschwindigkeit.«

»Okay, jetzt wird es kritisch. Packt eure Sachen und springt mit euren Jetpacks raus!«, rief yury aufgeregt. Dann legte er, mit einer Hand das Lenkrad festhaltend, sein Jetpack an. Beim Handwechsel schlingerte das blaue Geschoss auf die Gegenfahrbahn; nur die Uhrzeit verhinderte einen Unfall. Die Kabinentüren wurden aufgestoßen und vom Fahrtwind offen gehalten. Auch Orakel, Alexandra und Free zogen ihre Jetpacks an. Free riss alles an sich, was er für wichtig hielt; Alexandra gab ihm alles an, was er dabei vergaß. Dann trat sie die Ladetüren in beide Richtungen auf.

»38.86367, -77.05693. Sie fahren zu schnell!«, kommentierte das Navigationsgerät, dessen Lautsprecher sogar den Laderaum beschallten.

»\iashout{Raus hier!}«, schrie yury. Alexandra und Free sprangen nach hinten ab und aktivierten gleichzeitig die Jetpacks. Free hatte sich im letzten Moment noch den Örztöp geschnappt. yury verließ den LKW als Letzter und verlor dabei einen seiner Schuhe, was ihm in dieser Situation aber ziemlich egal war. Als sich die vier kurz umdrehten, sahen sie, wie der LKW gegen eine Reihe aus Pollern fuhr und durch den Aufprall vollständig zerstört wurde. Dann zog Orakel eine Schutzschildpistole aus dem Gürtel und gab sie yury, der nun auch bei ihnen angekommen war. yury zerschoss damit wahllos irgendeines der vielen Fenster und die vier flogen ins Innere des Pentagons.

Draußen zog die Unfallstelle eine Gruppe von Schaulustigen an, die den Nachfolgenden den Blick aufs Geschehen verdeckte. Die Einbrecher konnten sich ungestört im Hauptsitz des Verteidigungsministeriums umsehen.

»Seht mal – wir sind im Büro eines gewissen ›Dapper Drake‹ gelandet«, fand yury heraus und zeigte auf einen Stapel Dokumente.

»Das ist ja ganz toll, aber dieser Drake kann jederzeit wiederkommen und wir sollten uns ein besseres Versteck suchen. Irgendeinen Raum, den hier sowieso niemand betritt und in dem wir alles weitere planen können. Eigentlich müssen wir A. Nother Moron finden und ihm den Fernlöschungscode klauen«, erinnerte ihn Alexandra und öffnete entschlossen die Tür des Büros.

yury drehte sich erschrocken zur Tür um. »Zieh sofort die Tür wieder zu«, zischte er. »Der Gang ist nicht sicher.«

Alexandra blickte sich draußen um. »Doch, doch. Da ist niemand. In den Büros stehen bestimmt alle Mitarbeiter an den Fenstern und beobachten den brennenden Lastwagen. Das ist die Gelegenheit, von hier zu verschwinden.«

Murrend betrat yury hinter seinen Kollegen den tatsächlich menschenleeren Gang. Als die Gruppe an einem Aufzug vorbeikam, übernahm er kurzerhand wieder die Initiative, indem er die »nach oben«-Taste drückte. Die vier Eindringlinge blickten mit ungutem Gefühl auf die Stockwerksanzeige: Die Kabine fuhr aus dem Erdgeschoss nach oben. Für eine Flucht war es zu spät; Versteckmöglichkeiten bot der Gang nicht. Die leere Kabine war daher eine willkommene Erleichterung und wurde schnell genutzt.

Im 42. Obergeschoss hielt der Aufzug mit vier inzwischen weniger ängstlichen Einbrechern. Der Plan lief wie gewünscht, und auch hier hielt sich niemand auf dem Flur auf.

»Seit wann hat das Pentagon eigentlich so viele Stockwerke?«, fragte Free verwundert.

»Das Pentagon wurde vor einem Jahr stark vertikal ausgebaut«, erklärte yury. »Die Überwachung der gesamten Erdkommunikation erfordert viel Personal und viele Computerbildschirme. Wir befinden uns jetzt in einer Etage, die hauptsächlich als Reserve für zukünftige Projekte dient. Hier wird uns kaum jemand über den Weg laufen–«

Er wollte gerade noch anmerken, die Ecke habe einen Winkel von 108 Grad, als er gegen den Anzugträger stieß, der um die Ecke gerannt kam. Dessen Aktentasche flog ein Stück weit über den Flur, bevor sie auf dem Boden landete und aufplatzte. Lose Dokumente flogen durch die Gegend.

»Passen Sie gefälligst auf, wo Sie hinlaufen!«, beschwerte der Mann sich.

»Oh, bitte entschuldigen Sie. Ich glaube jedoch, Sie waren es, der –«, setzte yury zu einer Antwort an, doch er wurde von dem Mann unterbrochen.

»Wer sind Sie eigentlich? Was haben Sie hier zu suchen?«

So leicht ließ sich yury nicht überrumpeln. »Gebäudeinspektion Prospect Calm, im Auftrag der USPPD. Diese Etage ist den Angehörigen der Verwaltung vorbehalten. Haben Sie eine Zutrittsgenehmigung?«

Die hatte der fein gekleidete Herr offenbar nicht. Ebensowenig wie Ahnung davon, dass eine solche Vorschrift überhaupt nicht existierte. Er stieß yury zur Seite, riss seine Aktentasche an sich und verschwand, ohne die zu Boden gefallenen Papiere aufzuheben, in einem Treppenhaus.

»Der hatte eindeutig Dreck am Stecken«, befand Free, während er die Papiere vom Boden aufhob. Dann las er die Beschriftungen vor: »Gehaltsabrechnung für Timothy Conway. Eine ziemlich hohe Summe. Hier ein Kündigungsschreiben, aber von Frederick Broughton. Eine kurze Dienstanweisung bezüglich Datensicherungen im Pentagon, adressiert an Richard Spencer. Format and destruct Punkt Essha, Serpent Tezee, Volenc, A N M zweiundvierzig P T G vier.«

Orakel schüttelte Free leicht an den Schultern. »Hast du einen Wackelkontakt?«

»Da steht Computercode«, erwiderte Free. »Die Dokumente kommen aus allen Abteilungen des Pentagons, und dieses hier gefällt mir besonders.«

»Mir ist wurscht, welche Dokumente dir besonders gefallen«, befand yury genervt. »Wir haben es eilig.«

»Da liegt eine Diskette unter den Papieren«, bemerkte Orakel. »Serpent TC. Den Sportverein kenne ich noch nicht.« Er hielt eine 3,5-Zoll-Diskette in die Höhe.

»Das sagt mir jetzt irgendwie... nichts«, gab Alexandra zu. yury sah den altmodischen Datenspeicher und Orakel skeptisch an.

Free nahm die Diskette wortlos an sich, ging damit zum Treppenhaus und blickte sich darin um. Nichts war zu hören, niemand war zu sehen, keine Spur verblieb von dem merkwürdigen Herrn. Dann kehrte er zur Gruppe zurück.

yury hielt die Pappe mit dem Computercode in den Händen. »Außer dem, was du vorgelesen hast, steht da ja gar kein Programmcode mehr«, stellte er fest. »Das ist wohl eher eine Art Karteikarte für die Diskette. Und deren Beschriftung nach sind wir bereits am Ziel. Jemand hat uns die Arbeit abgenommen.«

»Das war dann aber viel zu einfach«, fand Alexandra. »Man will uns in eine Falle locken.«

»Ich glaube kaum, dass der Zusammenstoß sich so planen ließ«, widersprach Orakel. »Man hätte uns beinahe den Code vor der Nase weggeschnappt, und wir hätten hier ewig danach gesucht.«

»Es ist möglich, dass noch mehr Personen von dem Selbstzerstörungscode erfahren haben«, mutmaßte yury. »Dass wir ihn hier suchen müssen, wissen wir aus einer Liste, die mehreren Regierungsgehilfen über die Erde verteilt zur Verfügung stand.«

»Die ganze Code-Suche ist eine einzige Falle«, war Alexandra überzeugt. »Uns bleibt aber keine andere Wahl, als den Spuren zu folgen, die da jemand für uns ausgelegt hat. Wenn Island mit uns spielen will, müssen wir vorerst darauf eingehen, bevor wir zuschlagen können.«

»Die Beschriftung lässt darauf schließen, dass der Inhalt verschlüsselt ist«, erwähnte Free.

»Soll das heißen, dass dieses Serpent – das kann doch wohl nicht wahr sein«, erkannte yury und schlug sich mit der Hand gegen die Stirn. »Jetzt sind wir so weit gekommen und haben den Zerstörungscode auf einer Diskette in unseren Händen, können damit aber nichts anfangen, weil der Code verschlüsselt ist. Warum muss so etwas immer uns passieren?«

»Was wir vorerst brauchen, ist ein ungenutzter Raum, möglichst auf dieser Etage«, sagte Orakel. Alexandra, yury und Free stimmten zu und gingen weiter. Sie blieben vor einer Tür stehen, an der noch kein Namensschild befestigt war. Dahinter fanden sie einen völlig möbellosen Raum vor, der glücklicherweise mit grünem Teppich ausgelegt war. Es war nicht zu erkennen, wofür dieser Raum später einmal dienen sollte, aber es war offensichtlich, dass hier weder Besuch zu erwarten war, noch dass in absehbarer Zeit eine Einrichtung des Raumes stattfinden sollte. Orakel und Alexandra gingen an eines der Fenster, Free machte es sich an einer Wand des Raumes mit seinem Örztöp so bequem, wie das ohne Möbel möglich war und yury dachte als Einziger der vier daran, die Tür zu schließen. Da innen der passende Schlüssel steckte, schloss er zusätzlich ab.

»Von hier oben hat man einen tollen Ausblick«, freute sich Orakel. Auch Alexandra war begeistert.

»Hier oben steht ein Router«, bemerkte Free, was yury dazu veranlasste, sich neben ihn zu setzen und mit ihm auf das Display des Örztöps zu sehen.

»Wireless Connection Manager v.0.42 command line, connected to Pentagon EAP«, las yury erstaunt vor. »Woher hast du den Code?«

»Habe ich nicht«, antwortete Free. »Das hier ist Äöüzz-Technik, was erwartest du? Interessanter finde ich, dass bei WLAN-Empfang in dieser Höhe irgendwo ein Repeater hängen muss, an den man vielleicht herankommt.«

»Und dann?«

»Dann könnten wir hier ein bisschen für Chaos sorgen und das Internet auf dieser Etage manipulieren. Ein paar Falschmeldungen hier und da, schon laufen wieder alle davon«, schlug Free belustigt vor.

»Wenn du das vorhast, solltest du es jetzt tun. Gerade hast du die Möglichkeit dazu, dich hier relativ ungestört umzusehen«, gab yury zu bedenken.

»Ich komme mit!«, rief Orakel von der anderen Seite des Raumes aus, doch yury war dagegen.

»Ich fände es besser, wenn wir hier bleiben und den Raum einrichten würden. Wie ich dich kenne, hast du eine komplette Campingausrüstung in deinen Taschen versteckt. Als Erstes wäre es gut, wenn du deinen Rucksack hier ausleeren könntest«, sagte yury zu Orakel. Free ging zur Tür.

»Bis gleich. Welche Zimmernummer ist das hier?«, fragte Free.

»Keine Ahnung, guck außen nach«, sagte yury schulterzuckend. »Bis gleich.«

Free hörte, wie hinter ihm Dinge auf den Boden fielen. Orakel leerte seinen Rucksack aus. Dann schloss er die Tür und wollte sie hinter sich abschließen, bemerkte aber, dass ihm der Schlüssel fehlte. Er öffnete die Tür wieder, sah auf das Chaos, das durch das Ausleeren des Rucksacks entstanden war und schlug yury vor, die Tür wieder abzuschließen.

»Mache ich bei der nächsten Gelegenheit. Bis gleich. Wenn du wieder rein willst, klopf die Zahl Pi in binärer Notation gegen die Tür«, sagte yury, schob den sprachlosen Free aus dem Raum und schloss die Tür ab. Free entschied sich dafür, dass es am sinnvollsten war, sich nicht sofort über yury zu ärgern, sondern nach dem Repeater zu suchen. Er zog ein nagelneues Nükiä-Smartphone aus seiner Tasche, startete eine Art WLAN-Radar und näherte sich dem Punkt, der den Sendepunkt darstellte. Das war schwierig, weil das Programm nur die Entfernung, nicht aber die Richtung eines WLAN-Zugriffspunktes anzeigen konnte. Free musste unwillkürlich an das Kinderspiel »Topfschlagen« denken.

Trotz anfänglicher Schwierigkeiten gelang es Free, sich dem Punkt auf dem Radar so weit zu nähern, dass er überzeugt war, der Repeater befinde sich in dem Raum vor seiner Nase. Das Schild an dem Raum war mit »Central Server Room« beschriftet. Free wunderte sich über die hohe Lage dieses Raumes, die in seinen Augen keine Vorteile hatte, und öffnete die Tür.

Der Raum war von Klimaanlagen stark gekühlt; Stromverbrauch und Klimaschutz wurden nachrangig behandelt. Auf einer elektronischen Übersichtstafel wurden Daten über die Server präsentiert. Den Gedanken, sämtliche Außenverbindungen zu kappen, verwarf Free schnell wieder, auch wenn es ihm wahrscheinlich möglich gewesen wäre. Stattdessen suchte er nach dem Gerät, das die Funksignale in die Etage brachte. Das Radar war zu unpräzise für die weitere Suche. Free konnte es nicht lassen, mit dem Smartphone eine Textverbindung zu einem der Server aufzubauen. Die Äöüzz-Technologie seines Handys verschaffte sich Zugang zur Systemverwaltung, ohne einen Alarm auszulösen oder Logfiles zu füllen. Er lachte. Manche Sicherheitslücken behielt man besser für sich.

Free überlegte kurz, dann gab er einige Befehle ein. Zunächst wollte er wissen, welches Betriebssystem auf dem Server lief.

\noindent \parbox{\textwidth}{ \vspace{3ex} \hrule \vspace{3ex}

    \begin{footnotesize}
    \begin{ttfamily}

\noindent root.5route@[2001:db8:9e:7460::42:100]:\~{}\#  echo \$OSTYPE

\noindent linux-gnu

\noindent root.5route@[2001:db8:9e:7460::42:100]:\~{}\#  lsb\_release -a

\noindent No LSB modules are available.

\noindent Distributor ID:~Debian

\noindent Description:~~~~Debian GNU/Linux 6.0.1 (squeeze)

\noindent Release:~~~~~~~~6.0.1

\noindent Codename:~~~~~~~squeeze

    \end{ttfamily}
    \end{footnotesize}

\vspace{3ex} \hrule \vspace{3ex} }

»Wie alt ist das denn?«, staunte er. Mit dem altertümlichen »iptables«-Befehl bastelte er eine Hintertür in die Firewall des Geräts, verwischte seine Spuren und meldete sich ab. Kurz darauf fand er den Router. Dieser war weniger anfällig für Angriffe aus dem Netz, trug aber einen Klebezettel mit dem gesuchten Passwort.

\begin{center}
	∞∞∞
\end{center}

yury, Orakel und Alexandra waren fast damit fertig, den Raum in ein »Büro« zu verwandeln. Orakels Taschen- und Rucksackinhalt hätte auch durchaus ausgereicht, um einen deutlich größeren Raum komplett einzurichten. yury hatte es inzwischen aufgegeben, das Gesamtvolumen sämtlicher Taschen zu berechnen, aus denen Orakel laut eigener Aussage »alles außer genug zu Essen« hervorholen konnte. Es war ihm zudem ein Rätsel, wie Orakel so eine unglaubliche Last überhaupt tragen konnte. Wahrscheinlich hatte er irgendwelche Äöüzz-Technik bei sich, die sein Gepäck schwerelos machte und extrem komprimierte. Anders war nicht zu erklären, dass fünf Stühle, ein Schrank, ein Schreibtisch, Notizblöcke, Computermaus und -tastatur, ein Monitor, eine Docking-Station für Frees Örztöp und vieles mehr einfach in seinen Taschen verstaut gewesen sein sollte.

Um sich von der Stromversorgung des Pentagons unabhängig zu machen, installierte yury ein kleines Plutonium-Taschenkraftwerk unter dem Tisch. Danach stellte er eine kabellose Stromverbindung her.

»Ist das nicht irgendwie… schädlich?«, fragte Orakel erstaunt.

»Ja. Die Strahlung, die davon ausgeht, ist tödlich«, entgegnete yury ohne erkennbaren Gefühlsausdruck. Orakel war entsetzt.

»Wenn man der Strahlung dauerhaft für zehn Milliarden Jahre ausgesetzt ist, stirbt man«, beruhigte ihn Alexandra. Bevor er darauf antworten konnte, klopfte es an der Tür. yury hatte die Augen geschlossen und überprüfte in Gedanken die Kombination. Dann grinste er, öffnete die Tür und begrüßte Free mit den Worten: »Das war das Dezimalsystem, und schon die fünfte Nachkommastelle war falsch. Eigentlich müsste ich dich wieder rauswerfen.«

»Das war nur ein Tippfehler«, behauptete Free. Er schloss die Tür hinter sich ab. »Woher habt ihr die ganzen Möbel?«

»Orakel hatte das alles in seinen Taschen verstaut. Wenn er das alles mit Essen füllen würde, wären alle Hungerprobleme auf Örs gelöst, nehme ich an. Immerhin können wir jetzt Online-Bestellungen machen und diesen Raum als Zieladresse angeben – auf den ersten Blick ist es von einem normalen Büro ja nicht zu unterscheiden. Was hast du inzwischen gemacht? Hast du den Repeater gefunden?«

Free überlegte kurz. »Nicht wirklich.«

»Nein?«

»Ich habe den Serverraum entdeckt, mir root-Zugriff verschafft, die iptables für den Örztöp auf allen Ports geöffnet \emph{und} den Router durch einen kleinen Äöüzz-Chip ersetzt, den ich im Gehäuse des Originals versteckt habe. Insgesamt kann man sagen, dass wir jetzt die gesamte Kommunikation des Pentagons unter Kontrolle haben«, gab Free nicht ohne unnötigen Stolz an. Dann ging er auf den Schreibtisch zu, auf dem der Örztöp lag. Kurz darauf hatte er ein Programm namens »WäirShärk« geöffnet und sah sich interessiert die Internetanfragen der Bürorechner in den Nebenzimmern an.

Alexandra, die ihm gemeinsam mit Orakel und yury über die Schulter sah, sagte erstaunt: »Das sieht aber nicht so aus, als würden die arbeiten...«

»Das ist also der Surf-Alltag im Pentagon. Gut bezahlt dafür, im Floatbook herumzuhängen und Serienfilme zu gucken. Hätte ich nicht erwartet«, sagte yury.

»He, wer arbeitet denn da – Moment, das ist Moron! Da, seht mal«, rief Orakel überrascht und zeigte auf einige Nachrichten, die über das interne Netz versendet wurden.

»Tatsächlich. Und da ist auch die IP-Adresse«, freute sich Free. Seine Freude ließ allerdings nach, als er bemerkte, dass auf dem Rechner keine bekannten Sicherheitslücken vorhanden waren. Allein das von Moron verwendete Betriebssystem konnte Free bereits den Tag verderben, aber \emph{das} war zu viel. Es dauerte einige Zeit, bis yury ihn davon überzeugt hatte, dass es keine allzu gute Idee war, in Morons Büro zu stürmen und seinen Computer zu demolieren. Stattdessen beratschlagten die vier nun, wie man am besten an ein Diskettenlaufwerk kam, was noch die einfachste Aufgabe war, und wie man danach auch noch den Code knacken sollte, mit dem wiederum der Selbstzerstörungscode verschlüsselt war. Sie kamen nach einiger Zeit darauf, dass man dazu einen leistungsfähigen Quantencomputer benötigte, welchen es nur auf Örz gab. Alexandras gut gemeinter Vorschlag war es daher, den Inhalt der Diskette per Mail an ihren Nachbarn auf Örz zu schicken, damit dieser ihn entschlüsselte. Free war zwar ebenfalls dafür, sprach aber von irgendwelchen Problemen, die zu diesem Zeitpunkt noch niemand verstand.

Nachdem die vier sich aus ihrem »Büro« geschlichen hatten und unbemerkt zum Serverraum gelangt waren, zeigte Free ihnen, wo man dort welche Geräte finden konnte. Er konnte der Versuchung, die technischen Details der Server durch einen Vergleich mit Äöüzz-Technologie ins Lächerliche zu ziehen, nicht widerstehen, sodass mehr Zeit verging, als die vier ursprünglich eingeplant hatten. yury war froh, dass er die Tür des Büros hinter sich abgeschlossen hatte. Als Free seinen »kleinen« Vortrag beendet hatte, baute er gemeinsam mit Orakel ein Diskettenlaufwerk aus einem der Server aus.
»Dass die so etwas hier überhaupt noch besitzen...«, bemerkte yury verwundert, als ihm Orakel das Laufwerk übergab. Dann gingen die vier zurück zum Büro und mussten mit Erschrecken feststellen, dass jemand an der Tür gewesen war und dort einen Zettel hinterlassen hatte. Außerdem lag ein Pappkarton auf dem Boden vor der Tür.

»Was zur...«, begann yury und riss den Klebezettel von der Tür ab. Dann las er vor: »Empfänger nicht erreichbar, Lieferung trotzdem zugestellt. Wir buchen den Rechnungsbetrag bequem per Bankeinzug von Ihrem Konto ab. Hässliche Unterschrift.«

»Hässliche Unterschrift?!«

»Nein, das steht da so nicht. Aber die Unterschrift ist trotzdem hässlich«, erklärte yury. Dann zeigte er den anderen den Zettel. Alexandra hatte bereits den Karton aufgehoben und die vier gingen zuerst in den Raum und schlossen die Tür hinter sich, bevor sie nachsahen, was sich darin befand.

»Eine PIZZA?«

»Orakel, schäm dich!«, tadelte Free, konnte sich ein Grinsen jedoch nicht verkneifen.

»Aber ich hatte doch Hunger und der Örztöp und oh, ähm, öff, ja, ähm...«, sagte Orakel kleinlaut.

»Schon okay. Welches Konto hast du benutzt?«

Orakel überlegte kurz, suchte dann in seinen Taschen nach etwas und holte schließlich einen kleinen Zettel hervor.

»Hier – dieses Konto...«, sagte er und gab yury den Zettel. Dieser las entsetzt vor, was darauf stand:

 SOTIinternationalCreditAgency
 666-666-666-5071
 UnReal IGLS Ltd.
 SoulOfThe »SOTI« Internet

»Ich glaube, jetzt haben wir ein ''kleines'' Problem«, bemerkte Alexandra.

»Das hat Zeit. Jetzt kommt erst einmal die Diskette«, sagte Free entschlossen und richtete das Diskettenlaufwerk ein. Als alles zuverlässig lief, las er den Inhalt der Diskette mit dem Örztöp aus, verpackte ihn in einem .tar-Archiv, drehte sich um und sah in die erwartungsvollen Gesichter der anderen.

»Na los, worauf wartest du?«, fragte yury ungeduldig. »Starte deinen Mail-Client und schick die Mail an irgendeine Fake-Adresse, damit ein intergalaktischer Bote sie auffängt!«

»So einfach ist das nicht...«, wich Free aus.

»Was soll das heißen?«

»Naja...«, sagte Free und öffnete eine Textkonsole. Dann gab er einen Befehl ein und drehte sich erneut um.

»Was ist ''das'' denn?«, fragte Orakel erstaunt.

»''Das'' ist Mutt«, sagte Free.

»Mutt. Mutt ist Schrutt – oder so. Warum öffnest du nicht einfach Thunderbird und bist glücklich?«, fragte yury genervt.

»Ähm – den habe ich gelöscht.«

»WIE BITTE?«

»gelöscht...«

»Warum?«

»Ähm, Mutt ist viel cooler, verbraucht weniger Rechenleistung, benötigt weniger Arbeitsspeicher und...«, wollte Free erklären, doch er wurde von yury unterbrochen, der nun ziemlich wütend war: »Fang an. Und wenn irgendetwas nicht funktioniert, werfe ich den Örztöp aus dem Fenster!«

»Sicher doch«, sagte Free, schluckte und verfasste die Mail. Er zögerte jedoch, sie abzusenden.

»Na los, Mutti Schrutti!«, sagte yury, der noch immer wütend war.

Als Free auf den Senden-Knopf drückte und die Meldung »sendmail ist nicht installiert« erschien, war yury nicht mehr davon abzuhalten, den Örztöp tatsächlich mit voller Wucht in die Richtung der Fenster zu werfen. Der Örztöp flog gegen das Sicherheitsglas, prallte wie ein Gummiball davon ab und schlug den überraschten yury bewusstlos.

»Unfair!«, beschwerte sich yury, als er aufwachte. »Einfach unfair! Fenster haben gefälligst kaputtzugehen, wenn man Gegenstände dagegen wirft!«

»Wir haben, während du bewusstlos warst, einen Plan zusammengestellt. Wir müssen alle noch einmal zur Antarktis. Ein Bote vom intergalaktischen Lieferservice IGLS wartet dort bereits auf uns, um die Diskette entgegenzunehmen. Wir müssen nicht einmal für den Transport zahlen, weil die Örz-Regierung zugesagt hat, die Kosten von 500 Äzz zu übernehmen. Das ist viel Geld und wir sollten diese Chance nicht ungenutzt lassen. Vielleicht ist es unsere einzige!«, erklärte Alexandra. »Das Büro lassen wir so, wie es ist. Wir schließen es ab und hängen draußen ein 'Im Urlaub'-Schild an. Außer der Büroeinrichtung lassen wir nichts zurück, sodass es auch nicht schlimm ist, wenn jemand gewaltsam in das Zimmer eindringt. Orakel war der Meinung, dass wir uns wieder ein Flugzeug klauen sollten. Bei einer Enthaltung (Free) und jeweils einer Für- (Orakel) und einer Gegenstimme (ich) kannst du nun wählen, ob wir ein Flugzeug klauen oder uns etwas anderes überlegen sollen. Wir haben auch noch die Jetpacks. Nun, was meinst du?«

yury entschied sich – sehr zu Orakels Freude – für das Flugzeug. Bald darauf standen die vier vor einer nagelneuen Boeing 787-10. Das Flugzeug war vollkommen flugbereit, aber es befand sich kein einziger Mensch an Bord. Das lag eventuell daran, dass der gesamte Flughafen evakuiert worden war, nachdem Orakel eine – yurys Meinung nach vollkommen übertriebene – Atombombenselbstmorddrohung über sämtliche Lautsprecher auf der gesamten Anlage durchgegeben hatte. Es konnte nicht mehr lange dauern, bis Einheiten des SWAT eintrafen, und dann wollten die vier lieber nicht mehr in der Nähe sein.


'''''(nächstes Kapitel; Kapiteltrennung empfohlen)'''''

»Guck mal, da unten, das werden die Spezialeinheiten sein...«, kommentierte Alexandra das Geschehen unter ihnen. Das Flugzeug befand sich bereits weit genug entfernt von den Einsatzkräften und flog direkt auf die Antarktis zu.

»Macht es euch gemütlich. Ihr Pilot ist Or Master of Disaster Akel. Landung in – Free? – auch egal. Landung in einigen Stunden. Den Nichtschwimmern unter Ihnen wünschen wir ein möglichst schnelles und schmerzloses Ableben. Bitte achten Sie nach dem Absturz darauf, dass Sie nicht auf andere Passagiere treten. Vielen Dank, dass Sie mit DisasterAir geflogen sind!«, verkündete Orakel über die Bordlautsprecher. Dann holte Free seinen Örztöp hervor und suchte nach dem offenen Flugzeug-WLAN.

----
'''von yury:'''

»Warum hast du eigentlich nicht einfach Thunderbird aus dem Internet heruntergeladen?«, fragte yury betont beiläufig.

»Oh, das ist eine lange Geschichte«, antwortete Free. »Ich hatte Angst davor, rückfällig zu werden und aus Bequemlichkeit wieder Thunderbird zu installieren. Für diesen Fall habe ich sämtliche Seiten, die Thunderbird oder andere grafische Clients zum Download anbieten, über eine lokale Firewall gesperrt. Um ebenfalls nicht in die Versuchung zu kommen, diese Firewall einfach zu deaktivieren, habe ich einen Timer eingebaut und mein root-Passwort in ein Zufallspasswort geändert. Ich kann frühestens in 3 Monaten wieder einen grafischen Mail-Client installieren, und ich bin mir ziemlich sicher, dass ich es bis dahin geschafft haben werde, Mutt zum Laufen zu bringen.«

»Na wundervoll«, sagte Alexandra, die immer noch genervt war. »Wir fliegen also gerade Tausende von Kilometern, weil Free zwar problemlos in die Server des Pentagons eindringen kann, aber zu dumm ist, eine einfache E-Mail zu verschicken.«

»Ach, sei leise«, murmelte Free abwesend. Er hatte sich längst wieder seinem Örztöp zugewendet und war nicht mehr ansprechbar.

Plötzlich meldete sich Orakel über die Lautsprecher: »Hier ist noch einmal Ihr DisasterAir-Pilot, der gerade festgestellt hat, dass sich einige... Flugkörper auf Kollisionskurs mit unserer Maschine befinden. Es handelt sich dabei wahrscheinlich um... Vögel...«

Mit hochrotem Kopf stürzte yury ins Cockpit und schubste Orakel auf den Copilotensitz.

»Juhu, ich empfange hier WLAN... äh... wo ist denn yury?«, sagte Free, der – ganz vertieft in seinen Örztöp – nichts mitbekommen hatte.
----

»yury steuert gerade das Flugzeug, weil Orakel irgendein Problem hat«, antwortete Alexandra. »Außerdem solltest du dich vielleicht auf einen Flugzeugabsturz vorbereiten und eine entsprechende E-Mail an deine potentiellen Ex-Bekannten senden...«

Daraufhin sprang Free auf und öffnete die Cockpittür. Dahinter saß Orakel beleidigt neben yury, der nun das Flugzeug steuerte.

»Und ich dachte, da kämen Raketen auf uns zu«, sagte Free. Man sah ihm seine Enttäuschung deutlich an.

»Sei froh, dass es keine waren – mit einem Passagierflugzeug wie diesem hier kann man im Prinzip nicht ausweichen. Demnächst will ich eine F-15 kapern, keinen übergroßen Jumbojet!«, sagte yury. Dann flog er die 787-10 weiter in Richtung Antarktis.

----

Orakel, der sich nach seiner Degradierung zum Copiloten nicht mehr für ernst genommen hielt, verließ das Cockpit nach einiger Zeit und aß im hinteren Teil des Flugzeuges, relativ weit weg von den anderen, einige Sandwiches. Kurz darauf bemerkte yury, dass es besser gewesen wäre, ''vor'' dem Flug eine Toilette aufzusuchen. Da Orakel ihn nicht ersetzen konnte, bat er Alexandra, das Flugzeug zu steuern, doch sie lehnte ab. Sie hatte vor einiger Zeit einen Flugsimulator bedient und hatte keine allzu große Lust darauf, ihren Absturz in der Realität zu wiederholen. Schließlich ließ sich Free dazu überreden, seinen Örztöp für einige Minuten zu sperren und sich auf den Pilotensitz zu setzen. Da yury den Autopiloten aktiviert hatte, musste er auch nicht mehr tun, als auf ihn zu warten, doch dafür war ihm die verlorene Örztöp-Zeit zu wertvoll. Nach kurzem Zögern schaltete er den Autopiloten ab, drückte einige Knöpfe und las in der Bedienungsanleitung nach, was die Automatenstimme mit »Strömungsabriss« meinte.

Kurz bevor Free die Flugzeugnase nach oben ziehen konnte, kam yury zurück, machte ihm wütend klar, dass ein Flugzeug kein Helikopter sei, setzte ihn vor die Tür und bewahrte das Flugzeug im letzten Moment vor einem Absturz. Kopfschüttelnd sah er auf die neben ihm liegende Bedienungsanleitung («Herzlichen Glückwunsch zu Ihrer neuen Boeing 787-10! Vielen Dank, dass Sie sich für ein Boeing(r)(tm)(c)-Produkt entschieden haben!«) und schaltete den Autopiloten wieder ein.

----

Als die vier wieder aufwachten und aus den Fenstern sahen, bemerkten sie, dass sie in der Antarktis gelandet waren; diesmal waren allerdings keine Pinguine in der Nähe. Alexandra dachte in diesem Moment wieder an Örz, ihren Nachbarn und »Freddy«. Sie überlegte, was passiert wäre, wenn die vier beim uggy-Angriff auf die Erde selbst eingegriffen hätten, und sie dachte auch an die »Seele des Internets«, die offenbar immer noch existierte, sich bisher aber im Hintergrund hielt.  Dann unterbrach eine Stimme aus den Bordlautsprechern ihre Gedanken.

»Raumanzüge und Jetpacks anziehen, wir sind da. Der Bote ist wahrscheinlich schon am Treffpunkt angekommen und wartet auf die Diskette mit dem Code«, sagte yury, dann verließ er das Flugzeug und flog als Erster nach oben – ohne den Code, denn den hatte Orakel in einer seiner Taschen verstaut. Nachdem ein weiteres Sandwich den Weg zu seinem Magen gefunden hatte, flogen Orakel, Alexandra und Free hinterher.

----
'''von yury:'''

Nach einigen Minuten trafen sie den intergalaktischen Boten, der tatsächlich schon eine Weile wartete. Free teilte ihm mit, er solle die Diskette dem [i]Örz Ünztütüt pör Äpläd Krüptölögä[/i] übergeben.

»Für die Leute vom ÖÜÄK ist Serpent ein Kinderspiel. Sie sollen uns einfach den entschlüsselten Inhalt unserer Diskette per MBP erneut verschlüsseln und mit dem intergalaktischen Boten übermitteln.«

»ÖÜÄK? MBP?«, fragte Orakel, der überraschenderweise — so fand zumindest yury — nichts verstanden hatte.

»Das Institut heißt offenbar ÖÜÄK«, sagte Alexandra. »Und MBP steht für Müch Bättä Präväzi, ein Verschlüsselungsprogramm, das mit einem nach Örz-Maßstäben sicheren Public-Key-Verfahren arbeitet.«

Orakel fragte lieber nicht nach, was ein Public-Key-Verfahren war, sondern übergab den Boten stattdessen die Diskette. Dieser verschwand in den Weiten des Weltalls, bevor Free ihm sagen konnte, wem beim Institut er die Diskette genau übergeben sollte.

»Und warum sollen die ÖÜÄK-Leute den Inhalt jetzt noch mal verschlüsseln?«, fragte Alexandra.

»Ich möchte lieber auf Nummer sicher gehen«, antwortete Free. »Wer weiß, ob Island nicht die intergalaktische Kommunikation abhört?«
----

»Du bist wirklich paranoid.«

Mit diesen Worten flog Alexandra zurück zum Flugzeug und enteiste mit dem Raketentriebwerk ihres Jetpacks die Tragflächen; die anderen folgten ihr und bald darauf saß yury wieder neben Orakel im Cockpit, Alexandra irgendwo an einem Fenstersitz und Free mit dem Örztöp schräg gegenüber, in der Nähe der Cockpittür. Auch diesmal konnte Orakel, der nur noch als Co-Pilot zusehen und lernen durfte, sich eine Durchsage nicht verkneifen.

»Sehr geehrte Damen und Herren, wir starten in Kürze in Richtung USA. Unser aktueller Standort ist... äh, T.H.I.N.G.S.B.O.O.M.S. upgrade available, click here to upgrade, 80.027315 S, 25.812807 E. Wir wünschen Ihnen einen guten Flug und bitten darum, die Kotztüten nur vollständig entleert in Ihrem Sammelalbum einzuheften. Vielen Dank.«


'''''(nächstes Kapitel; Kapiteltrennung empfohlen)'''''

Nachdem der intergalaktische Bote die Diskette entgegengenommen hatte, fragte er sich, warum dieses antiquierte Speichermedium immer noch außerhalb eines Museums existierte. Eigentlich war ihm auch egal, wie weit die Menschen auf Örs hinter ihrem Mond lebten, aber nachdem er sein Raumschiff betreten und den Warp-Antrieb aktiviert hatte, sah er sich die Diskette genauer an. Sie war auf einer Seite mit \iaquote{»formatanddestruct.sh, Serpent-tc, volenc, ANM 42 ptg-4«} beschriftet, bestand aus irgendeinem Kunststoff und hatte an einer Seite ein Metallteil, das sich zurückziehen ließ. Außerdem verhinderte eine Feder, dass das Metallteil dauerhaft geöffnet blieb. Der Bote ließ das Metallteil einige Male interessiert zurückfedern, dann sah er sich die dahinter liegende schwarze Folie an. Sie bestand ebenfalls aus Kunststoff und ließ sich mit einem Metallrad in der Mitte der Diskette drehen. Erst jetzt achtete der Äöüzz auf die Beschriftung am Metallteil – \iaquote{»1,44 MB«}. Er hatte sich nicht verlesen: Auf diesem komischen Ding ließen sich nicht mehr als 1,44 Megabyte Daten speichern. Erstaunt fragte er sich, wie mächtig wohl ein Diktator sein konnte, den man mit dieser Datenmenge zu Fall bringen konnte.

Ein Piepton riss den intergalaktischen Boten aus seinen Überlegungen. Schnell drückte er einen grünen Knopf und stellte damit eine Verbindung zur Raumhafenkontrollzentrale von Ciscö her. Die nun folgende Bitte um Landeerlaubnis war eine reine Formsache, denn die Örzplänätüüz-Automatik auf Ciscö hätte das Raumschiff schon längst vernichtet, wenn es sich nicht automatisch identifiziert hätte. Hier, auf dem sogenannten »Rüütplänät« der Äöüzz-Galaxie, trafen alle materiellen intergalaktischen Lieferungen an Planeten innerhalb der Galaxie ein. Die Feinde der Äöüzz schreckten vor direkten Angriffen auf Örz zurück, aber es gab immer wieder einzelne Möchtegern-Helden der uggy, die »illegale Lieferungen« durch die Zerstörung von Ciscö unterbinden wollten.

Während die Diskette ihren langen Weg zum ÖÜÄK von Raumschiff zu Raumschiff nahm, kam das Flugzeug der vier ohne nennenswerte Zwischenfälle wieder in den USA an. Als die Polizei bemerkte, dass es sich um das gestohlene Flugzeug von Orakels Atombombendrohung handelte, befanden sich Orakel, yury, Alexandra und Free bereits in einem Taxi zum Pentagon. yurys Angewohnheit, Taxifahrern pauschal mehrere »große« Geldscheine (in diesem Fall fünf 100-Dollar-Noten) für die Missachtung sämtlicher Geschwindigkeitsbegrenzungen zu geben, erwies sich wieder einmal als nützlich.

----
'''von yury:'''

„Welcome home“, sagte Alexandra, als die vier ihr altbekanntes „Büro“ betraten. Orakel drehte das 'Im Urlaub'-Schild um. Auf der Rückseite stand 'Bitte reinkommen'.
Wortlos nahm yury ihm das Schild weg und warf es in den Müll.

„Und, was machen wir jetzt?“, erkundigte sich Alexandra.

„Na ja, wir warten auf den entschlüsselten Zerstörungscode und Free versucht derweil, seinen Mailclient zum Laufen zu bringen“, sagte yury.

„Erst mal muss etwas zu essen her“, befand Orakel. Er griff zum Bürotelefon und gab etwas ein, während Free den Örztöp hochfuhr.

„Hast du dir ernsthaft die 16-stellige Nummer des Pizzaservices gemerkt?“, fragte Alexandra ungläubig.

„Eine Pizza Margherita UltraNewHappyPartyHyperFamilySize mit extra Käse bitte“, bestellte Orakel grinsend am Telefon.

„Du hast vielleicht Hoffnungen“, sagte yury zu Alexandra. „Er hat die Nummer längst eingespeichert.“

„Ich gebe es auf“, sagte Free resigniert. „Ich werde jetzt versuchen, die Mail über SMTP zu verschicken.“
Als er sah, dass yury es ebenfalls aufgegeben hatte, sich über ihn aufzuregen, fügte er hinzu: „Das ist echt spannend, weil man andauernd aufpassen muss, keinen Timeout zu verursachen!“
Bevor yury überlegen konnte, doch noch wütend zu werden, klopfte es.

„Die Pizza!“, rief Orakel und öffnete freudestrahlend die Tür.
----

\begin{center}
	∞∞∞
\end{center}

Draußen standen zwei Männer, deren Polizeiuniformen nicht unbedingt für Orakels Vermutung sprachen. Umso erstaunter waren seine Freunde darüber, dass Orakel tatsächlich einen Pizzakarton in die Hand gedrückt bekam.

»Guten Tag, wir haben diesen Karton vor Ihrer Tür gefunden. Scheint auch bereits bezahlt worden zu sein. Bitte entschuldigen Sie die Störung. Haben Sie zufällig vier Terroristen gesehen?«

»Nö.«

»Okay, das wäre auch zu einfach gewesen. Trotzdem vielen Dank. Schönen Tag noch und guten Appetit!«

Orakel bedankte sich und kehrte fröhlich an den Schreibtisch zurück. Free starrte fassungslos abwechselnd Orakel und die inzwischen wieder geschlossene Tür an.

»Du hast echt mehr Glück als Verstand«, stammelte Alexandra. »Willst du das alles alleine essen?«

»Wenn du möchtest, kannst du ein Stück abhaben«, bot Orakel großzügig an. Er klappte den Kartondeckel nach oben – die Pizza war in 32 gleich große Stücke geschnitten worden. »Oder zwei.«

Auch Free bekam bei dem Anblick großen Appetit. »Ich hätte gerne ein Sechzehntel davon.«

»Aber dann habe ich doch fast nichts mehr zu essen«, antwortete Orakel entsetzt.

»Oh, Entschuldigung. Ich meinte natürlich ein Achtel.«

»Das ist in Ordnung.«

Free grinste und nahm sich vier Stücke von der Pizza. Orakel ahnte, dass er auf einen Trick hereingefallen war, und nahm sich vor, nicht auch noch yury ein Stück anzubieten. Wo war der überhaupt?

\iathought{Wie peinlich}, dachte Orakel. \iathought{yury hat sich bestimmt von uns verabschiedet, und ich habe mal wieder vergessen, warum er weg ist.}

Nachfragen konnte er schlecht. Die anderen hätten sich bestimmt wieder darüber lustig gemacht, dass er als Einziger nichts verstand. Außer ihm schien auch niemand über yurys Fehlen verwundert zu sein.

\begin{center}
	∞∞∞
\end{center}

Das Pizza-Achtel lag schwer im Magen. Free lehnte sich zurück, und ein kühler Luftzug strich über seinen Hals. Nach einigen Sekunden bemerkte Free, dass dieser Luftzug unmöglich aus dem Türspalt dringen konnte, wenn der restliche Raum luftdicht verschlossen war. Irgendwo stand ein Fenster auf, aber niemand hatte eines geöffnet. Misstrauisch blickte Free nach links, während er blind weiter an der E-Mail tippte. Eines der großen Fenster war nicht verschlossen, sondern nur angelehnt worden, und ab und zu wackelte der Fensterrahmen im Wind.

\noindent \parbox{\textwidth}{ \vspace{3ex} \hrule \vspace{3ex}

    \begin{tiny}
    \begin{ttfamily}

\noindent 500-unrecognized command

\noindent 500 Too many syntax or protocol errors

    \end{ttfamily}
    \end{tiny}

\vspace{3ex} \hrule \vspace{3ex} }

Genervt stand Free auf und schloss das Fenster.

»Wo ist yury eigentlich?«, fragte Alexandra schließlich. Sie blickte in ratlose Gesichter.

»Ich war mir sicher, ihr wüsstet das«, gab Orakel zurück.

»Du weißt es also auch nicht?«

»Nein, yury war auf einmal einfach weg.«

Free ging zurück an seinen Arbeitsplatz. »Vielleicht hat es etwas mit den Polizisten zu tun. Die haben uns abgelenkt und heimlich yury entführt.«

\begin{center}
	∞∞∞
\end{center}

yury benötigte nur den Bruchteil einer Sekunde, um zu begreifen, wer da vor der Tür stand. Unter der Tür hindurch waren vier blau-schwarze Stiefel mit weißen Reflektorstreifen zu sehen, die so nur von Polizisten der Internen Schutztruppe getragen wurden. Für die Sicherheit des Diktators persönlich verantwortlich, waren diese Menschen nicht gerade für ihre Zärtlichkeit gegenüber vermeintlichen »Terroristen« bekannt. Das Pentagon war von einem LKW gerammt worden, und der Täter befand sich möglicherweise im Inneren des Gebäudes.

»Wir waren viel zu leichtsinnig«, murmelte yury, während er das Fenster aufdrückte. »Natürlich musste man uns früher oder später hier vermuten.« Dann sprang yury mit den Füßen voran in die Tiefe.

42 Stockwerke. Wenn jede Etage mindestens zweieinhalb Meter hoch war, lagen über 100 Meter zwischen ihm und dem Boden. Die voraussichtliche Aufprallgeschwindigkeit? Ungefähr 45 Meter pro Sekunde. \iathought{Na, das sind ja schöne Aussichten}, dachte yury verzweifelt. \iathought{Nicht einmal eine Wasseroberfläche kann mich retten. Ich habe noch ungefähr drei Sekunden zu leben.}

Bevor er sich für diese Schnapsidee verfluchen konnte, fiel ihm auf, dass er ein Jetpack auf dem Rücken trug. Der Knopf an der Unterseite des Rucksacks war für absolute Notfälle vorgesehen und absichtlich so platziert worden, dass man sich gehörig die Finger verbrannte, wenn man sich für dieses Vorgehen entschied. Entschlossen griff yury zu.

Normalerweise diente der mitgelieferte Wasserstoff als hochkomprimierter Energiespeicher für den flammenlosen Antigravitationsantrieb. So ließ sich der Tankinhalt für stundenlange Flüge nutzen. yury war das in diesem Moment herzlich egal; für ihn zählte nur, dass er ein extrem brennbares Gasgemisch dabei hatte. Bevor er seine Hand zurückziehen konnte, schlugen Flammen aus der Unterseite des Jetpacks. Das Vorgehen war gefährlich, aber schlimmer konnte die Situation sowieso nicht mehr werden. Der Sturz wurde gebremst, yury berührte kurzzeitig sanft den Boden und das Jetpack riss ihn wieder in die Luft. Bevor er erneut eine gefährliche Höhe erreichen konnte, löste yury die Verbindungen zu seinem Rucksack. Wie eine Rakete flog der Wasserstofftank an ihm vorbei ins All. yury hatte das Gefühl, sich auf einem Trampolin zu befinden und nach einem hohen Sprung wieder in die Tiefe zu fallen. Geistesgegenwärtig rollte er sich auf dem Steinboden ab, sodass er außer einer schmerzenden Hand keine Verletzungen davontrug.

\begin{center}
	∞∞∞
\end{center}

Einem plötzlichen Einfall folgend, trat Alexandra an das Fenster, öffnete es und beugte sich hinaus. »Guck mal«, sagte sie dann. »da ist yury.«

So geschmackslos würde selbst Alexandra nicht auf eine Leiche reagieren, hoffte Free. Während er sich vorsichtig von seinem Klappstuhl erhob, war Orakel bereits aufgesprungen und zum Fenster gestürmt.

Unten lief eine Gestalt, die in der Tat einige Gemeinsamkeiten mit yury aufwies, zu einem großen Fischteich. Free hob eine Augenbraue, als er sah, dass die Person niederkniete und ihre Hand in dem offensichtlich algenbewachsenen Schnutzwasser wusch.

»Was macht der denn schon wieder Verrücktes?!«, fragte Alexandra.

Orakel räusperte sich. »Vielleicht will er Fische mit der Hand fangen.«

»Ich glaube eher, er hat schon wieder einen Plan, von dem außer ihm niemand etwas versteht«, mutmaßte Free.

\begin{center}
	∞∞∞
\end{center}

Nachdem er den Hersteller des Jetpacks gedanklich mit einer Vielzahl derber Flüche belegt hatte, zog yury seine noch immer schmerzhaft pochende Hand aus dem Wasser. Hinter ihm ragte das Pentagon in die Höhe, in dem seine Freunde sicherlich längst verhaftet worden waren. Das war wieder einmal typisch. Die anderen brachten sich in Schwierigkeiten, und er musste alles ausbaden. Ein Blick zurück: Das Fenster stand noch immer offen, aber das Jetpack war weg. Wahrscheinlich war es in einigen Kilometern Entfernung jemandem auf den Fuß gefallen.

\begin{center}
	∞∞∞
\end{center}

\noindent \parbox{\textwidth}{ \vspace{3ex} \hrule \vspace{3ex}

    \begin{tiny}
    \begin{ttfamily}

\noindent Smithsonian National Zoological Park

\noindent 3001 Connecticut Avenue

\noindent NW Washington, DC 20008

    \end{ttfamily}
    \end{tiny}

\vspace{3ex} \hrule \vspace{3ex} }

Mit Hilfe des Örz-Smartphones fand yury schnell heraus, dass das teure Ausrüstungsstück inmitten eines Zoos zu Boden gegangen war. Genervt stöhnend lief er zu Fuß dorthin, umging geschickt eine lange Besucherschlange und stand bald vor dem Kassenhäuschen.

Die Dame am Empfang lächelte ihn freundlich an. »Sie möchten bestimmt unsere wundervollen roten Pandas, Nutmeg und Jackie, sehen.«

»Nein«, entgegnete yury ebenso freundlich. »Ich möchte mein außerirdisches Wasserstoff-Jetpack aus dem Seelöwenteich fischen.«

Die umstehenden Besucher lachten herzlich, und yury erhielt eine Eintrittskarte, auf der die amüsierte Kassiererin ihre Handynummer notiert hatte.

\begin{center}
	∞∞∞
\end{center}

Bei den Seelöwen war allerdings keine Spur des Geräts zu finden. Stattdessen wurde yury von seinem Smartphone quer durch den ganzen Zoo geführt, bis er schließlich an einem Kinderkarussell vorbei kam. Fröhlich winkte er einigen Fahrgästen zu, bevor er seinen Blick wieder auf das Display richtete und die Stirn runzelte. Er war angeblich nur noch 35 Meter vom Ziel entfernt, und der grüne Navigationspfeil wies eindeutig in Richtung eines Tiergeheges, um das yury gerne einen großen Bogen gemacht hätte. Das war jedoch sein geringstes Problem.

»Das Ziel befindet sich in fünf Metern Höhe?!«, rief yury entsetzt. Einige Besucher drehten sich verwirrt nach ihm um. »Äh, ich mache hier Geocaching.«

Als er daraufhin nicht auf das Karussell, sondern auf den schwarzen Metallzaun zuging, rief ihm eine ältere Dame etwas hinterher. »Sie wissen schon, dass das da drüben ein indischer Königstiger ist?«

yury blickte auf das Hinweisschild.

\iaquote{»Ein Königstiger benötigt ca. 8 kg Fleisch am Tag. Seine Hauptnahrung sind große Säuger wie Nilgauantilopen, Gaure, Sambarhirsche, Barasinghas, Axishirsche und Wildschweine. Seltener frisst er kleinere Beutetiere wie Affen, Hasen, Kaninchen und Wasservögel. Der Tiger schleicht an seine Beute heran, springt sie an und drückt sie mit den kräftigen Vorderpfoten auf den Boden. Die Weite der Sprünge kann bis zu 6 Meter betragen. Zum Töten beißt er in die Kehle seines Opfers oder bricht dessen Genick durch einen Biss in den Nacken.«} (Wikipedia, https://de.wikipedia.org/wiki/K%C3%B6nigstiger, Abruf 2018-02-01)

Wie, zur Hölle, sollte er das Jetpack unter diesen Umständen vom Baum holen?

\begin{center}
	∞∞∞
\end{center}

»Und?«, fragte Orakel neugierig.

Free hatte eine Landkarte auf dem Örztöp geöffnet. »In yurys Laufrichtung liegen das Lincoln Memorial, die Washington National Opera, die George Washington University und das Weiße Haus.«

»Das Weiße Haus gibt es nicht mehr«, korrigierte Alexandra. »Da steht jetzt der ›Tower of Liberty‹. yury ist vollkommen verrückt geworden, wenn er da zu Fuß hineinspazieren möchte.«

»Sagt jemand, der mit einem LKW das Pentagon gerammt hat und sich dort in einem verlassenen Büro versteckt«, witzelte Free.

Alexandra verschränkte die Arme. »Immerhin verhalten wir uns so unauffällig, dass sogar die Polizei uns nicht erkennt.«

\begin{center}
	∞∞∞
\end{center}

Der Zoowärter lief eilig in Richtung des Großkatzengeheges. Einige besorgte Besucher hatten gemeldet, dass jemand auf das Dach des Besucherganges geklettert war. Es bestand die Gefahr, dass er abrutschte und zwischen den hungrigen Tigern landete.

yury hatte von einem der Laubbäume einen Ast abgebrochen, der sich dank seiner Form gut als Greifhaken nutzen ließ. Nun löste er den Verschluss seiner Armbanduhr, brach einen zweiten, langen Ast ab und verband die beiden Holzstöcke mit dem Gummiband. Ein kurzer Belastungstest verlief zu seiner Zufriedenheit, sodass er sich wagte, das Jetpack damit vom Baum zu holen.

Schmunzelnd bemerkte yury, dass die Tiger und Löwen interessiert dabei zusahen, wie er sich am Baum zu schaffen machte. Ein komisches Ding war vom Himmel gefallen und hatte sich in der Baumkrone verfangen. Die Bergungsaktion war eine willkommene Abwechslung im Alltag der Raubtiere.

»He, Sie da«, brüllte plötzlich jemand mit einem Megafon von unten. yury zuckte zusammen, verlor kurz das Gleichgewicht und ließ das Jetpack vom Baum herab fallen. »Sind Sie lebensmüde?«

»Sie haben vielleicht Nerven«, gab yury zurück. »Beinahe hätte ich mich so sehr erschrocken, dass ich in das Gehege gefallen wäre.«

Der Zoowärter ließ sich dadurch nicht beirren und brüllte weiter durch das Megafon. »Kommen Sie gefälligst da runter, es besteht Lebensgefahr!«

yury zögerte. »Das geht nicht so einfach. Ich muss zuerst dem Löwen das Jetpack abnehmen.«

\begin{center}
	∞∞∞
\end{center}

Alexandra beäugte misstrauisch das aus mehreren Elektronikplatinen zusammengebastelte Gerät. »Vom ästhetischen Aspekt vollkommen abgesehen, wirkt deine Idee doch sehr unprofessionell.«

Orakel ließ sich dadurch nicht entmutigen. »Über das Aussehen müsstest du dich ja auch bei Free beschweren. Funktionieren wird es trotzdem.«

Fünf Sekunden später gingen alle Lichter in dem fensterlosen Raum aus. Sofort riss Alexandra das Gitter vom Lüftungsschacht ab und kletterte hinein. Orakel schloss das Gitter hinter ihr, während sie bereits eilig den Schacht durchquerte.

\begin{center}
	∞∞∞
\end{center}

»Mein Herz«, stammelte ein älterer Herr beim Anblick der beängstigenden Szene. »Diese leichtsinnige Generation und ihre Computerspiele.«

»Das liegt nicht am Alter«, widersprach ihm eine junge Frau neben ihm. »Der Verrückte entstammt eindeutig der erlebnisorientierten Unterschicht.«

yury ließ den verdutzten Tiger genervt stehen. »Warte mal, Großkatze.« Er lief zum Zaun und blickte den Gaffern abwechselnd in die Augen. »Verschonen Sie mich gefälligst mit Ihrem Schubladendenken. Ich bin hier beschäftigt und muss mich darauf konzentrieren, nicht als Tigerfutter zu enden. Dankeschön.«

Der Tiger hatte allerdings gar kein Interesse daran, sein neues Spielzeug gegen ein Stück Fleisch einzutauschen. Er sah yury an, als wollte er sagen: »Das da bekomme ich sowieso jeden Tag. Da musst du mir schon etwas Besseres anbieten.«

yury seufzte. Wie sollte man mit einem Tiger verhandeln, der kein Fleisch als Zahlungsmittel akzeptierte? Er griff nach seinem Smartphone und tippte eine Nummer ein.

»Hallo, IGLS?«

»Guten Tag yury, wie können wir Ihnen behilflich sein?«

»Transportieren Sie auch Großkatzen?«

\begin{center}
	∞∞∞
\end{center}

Als Alexandra und Orakel in das Büro zurückkehrten, wurden sie von Free fröhlich begrüßt.

»Ich habe zwei gute Nachrichten für euch.«

»Keine schlechte?«, hakte Orakel nach, während er die Tür hinter sich zuzog.

»Nein«, bekräftigte Free. »Es wurde kein Alarm ausgelöst, und wir haben elektronische Post von Örz erhalten.«

Orakel lief erfreut um den Tisch herum und blickte über Frees Rücken auf ein bedrucktes Blatt Papier.

»Du hast das komische Text-Mailprogramm zum Laufen zu bekommen«, riet Alexandra schmunzelnd.

»Er hat das Kopiergerät auf dem Flur gehackt und zum E-Mail-Abruf verwendet, weil sein eigenes Programm immer noch nicht funktioniert«, antwortete Orakel grinsend. »Aber der Nachrichteninhalt ist die eigentliche Sensation. Der Disketteninhalt wurde entschlüsselt.«

\begin{center}
	∞∞∞
\end{center}

Die Polizisten hatten wenig Verständnis für yurys Ausflug in das Tigergehege.

»Wegen solchen Einsätzen müssen wir die armen Steuerzahler unnötig belasten«, tadelte die Sergeantin.

»Sie müssen den Kerl einsperren«, zeterte der Zoowärter. »Dieser Rowdy hat unsere Tiere gestört und die Besucher vergrault.«

»Es freut mich, dass Ihnen mein Leben so viel bedeutet«, reagierte yury gelassen. »Sehr gerne würde ich mich weiter mit Ihnen unterhalten. Leider habe ich jedoch wichtigere Aufgaben zu erledigen.«

Mit diesen Worten aktivierte er sein Jetpack. Vom ursprünglichen Treibstoffvorrat war nicht viel übrig geblieben, aber für einen Flug zum Pentagon genügte die Ladung.

\begin{center}
	∞∞∞
\end{center}

»Ich weiß, darauf hätten wir auch selbst kommen können«, meinte Alexandra, bevor Orakel einen Kommentar abgeben konnte. Auf dem Tisch lag der entschlüsselte Inhalt der geheimnisvollen Diskette, ausgedruckt in Papierform. Der Schlüssel, der dafür notwendig gewesen war, hatte sich neben der verschlüsselten Datei auf der Diskette befunden – und damit das nicht ganz so dämlich war, wie es zunächst klang, konnte man den Schlüssel nur durch Eingabe einer Passphrase nutzen.

Plötzlich polterte etwas vor den Fenstern zu Boden. Orakel, Alexandra und Free drehten sich erschrocken um.

»correct horse battery staple«, fluchte yury. »Das ist mal wieder typisch für jemanden, der beim FBI gearbeitet hat.« Dann schloss er das Fenster.

»yury! Woher hast \emph{du} denn das Passwort für die Diskette?«, fragte Free verdattert.

»Vielleicht hat er das im Tower of Liberty gefunden«, mutmaßte Orakel. »Übrigens schön, dass du wieder da bist.«

yury nahm amüsiert das Jetpack ab. »Tower of Liberty? Dachtet ihr, ich bin spontan dort eingebrochen?«

Alle drei nickten.

»Nein. Ich dachte, die interne Schutztruppe stand vor der Bürotür, und da bin ich kurzerhand durch das Fenster geflüchtet. Daraus, dass ihr hier noch gemütlich beisammen sitzt, schließe ich, dass es sich um einen Irrtum gehandelt hat.«

Alexandra lachte. »Da standen tatsächlich zwei Mitarbeiter der Schutztruppe vor der Tür.«

yury schwieg verständnislos.

»Die haben mir eine Pizza in die Hand gedrückt und gefragt, ob wir zufällig vier Terroristen gesehen haben«, erklärte Orakel.

»Die werden sich doch wohl nicht mit einem ›Nein‹ zufrieden gegeben haben«, stammelte yury.

»Doch«, bestätigte Orakel.

»Wir waren zu diesem Zeitpunkt ja auch nur drei Personen«, ergänzte Alexandra.

Mit einem ungläubigen Kopfschütteln ging yury auf den Schreibtisch zu, legte das Jetpack darauf ab und blickte auf den Computerbildschirm.

»Du hast uns noch immer nicht erzählt, woher du das Passwort hast, für dessen Ermittlung wir mühsam eine Diskette nach Örz geschickt haben«, erinnerte ihn Free.

»Ach, das ist eine lange Geschichte. Ich habe bei IGLS angerufen, um einen indischen Königstiger und seine Familie auf einen örz-ähnlichen Planeten transportieren zu lassen. Nebenbei wurde mir dann die inzwischen ermittelte Passphrase übermittelt.«

Diesmal waren es seine Freunde, die ihn verständnislos anstarrten.

»Ist wirklich so«, behauptete yury mit verschränkten Armen. »Also, was befindet sich auf der Diskette? Ein Kommandozeilenskript?«

»Langsam wirst du mir unheimlich«, sagte Free.

Alexandra legte yury das ausgedruckte Skript aus der E-Mail vom ÖÜÄK vor. Sie schmunzelte. »Herzlichen Glüclwunsch, Kommandant. Du bist in Abwesenheit zum ›Grand Senior Master Guardian‹ der Weltregierung befördert worden. Ein echter Karrieresprung.«

yury überflog das Papier. »Schön«, gab er zurück. »Ich ordne hiermit die vollständige, sofortige und unwiderrufliche Zerstörung sämtlicher Rechner des Island-Regimes an.«

Free nickte. Das war eine Aufgabe nach seinem Geschmack. Er öffnete einen Texteditor, tippte den entschlüsselten Text von Hand ab und speicherte die Datei anschließend als »formatanddestruct.sh« auf der Festplatte des Örztöps ab.

\noindent \parbox{\textwidth}{ \vspace{3ex} \hrule \vspace{3ex}

    \begin{tiny}
    \begin{ttfamily}

\noindent Bash-Code, eine Zeile Netcat 0x53634D4D auf TCP 32764

    \end{ttfamily}
    \end{tiny}

\vspace{3ex} \hrule \vspace{3ex} }

Die Freunde blickten erwartungsvoll zwischen dem Blatt und dem Computerbildschirm hin und her. Free schien das überhaupt nicht zu bemerken und lehnte sich entspannt zurück.

»Hey, Moment mal«, sagte yury nach einer Weile. »Da steht doch viel mehr auf dem Papier.«

»Ja, aber es lässt sich mit dieser einen Zeile zusammenfassen«, entgegnete Free grinsend.

yury zögerte. Das war alles? Diese Zeile Text sollte genügen, um das seit Wochen verfolgte Ziel zu erreichen? Wo war der Haken?

»Du bist zweifellos sehr kompetent«, sprach yury betont langsam und höflich. »Aber ich wage es, zu bezweifeln, dass diese komische Postleitzahl aus Florida wirklich das ist, wonach wir gesucht haben.«

Ein gewisser Übermut befiel Free, als er das hörte. Genüsslich beugte er sich nach vorne und bewegte seinen rechten Zeigefinger in eine schwebende Position über der Entertaste. Dann zog er eine Sonnenbrille hervor, setzte sie mit der linken Hand auf und blickte lässig in die Runde. »Shall we begin?«

Orakel und Alexandra blickten begeistert auf den Laptop; bei yury war eine gewisse Skepsis nicht verkennbar.

\ialoudspeaker{»Klick.«}

Eine neue, leere Zeile erschien unter dem eingetippten Befehl. Das Programm arbeitete.

Zehn Sekunden vergingen in nervöser Anspannung. Alexandra ließ währenddessen gedankenverloren die Finger knacken, was ihr einen kurzen missbilligenden Seitenblick von yury einbrachte. Schließlich erschien das Resultat des Befehls auf dem Bildschirm; gespenstische Stille erfüllte augenblicklich den Raum.

\noindent \parbox{\textwidth}{ \vspace{3ex} \hrule \vspace{3ex}

    \begin{tiny}
    \begin{ttfamily}

\noindent nc: connect to 2001:db8:1:1a0:539:7ff3:65:29a port 32764 (tcp) failed: Connection timed out

    \end{ttfamily}
    \end{tiny}

\vspace{3ex} \hrule \vspace{3ex} }

Ein verhaltenes Räuspern durchbrach das Schweigen.

»Sorry, aber ich glaube, du kannst die Sonnenbrille abnehmen«, monierte yury. »Da kommt nichts Erleuchtendes mehr raus.«

Während Free mit hängenden Schultern die Brille einfach zu Boden fallen ließ, griff Orakel kurzerhand nach dem Laptop.

»Das kann ja auch nicht funktionieren«, erklärte er fachmännisch. »Du hast ja auch nicht alles richtig abgetippt.«

Eine Viertelstunde später stellte er das Gerät wieder vor seinen Freunden auf den Tisch. Er legte das Papierblatt daneben, sodass jeder sich von seiner einwandfreien Arbeit überzeugen konnte. Der ausführliche Programmcode enthielt sogar eine Bedienungsanleitung. Nachdem yury die Anleitung mehrfach Wort für Wort gelesen hatte, zog er die Tastatur zu sich heran und tippte mit übertrieben wirkender Vorsicht einen Befehl ein.

\noindent \parbox{\textwidth}{ \vspace{3ex} \hrule \vspace{3ex}

    \begin{tiny}
    \begin{ttfamily}

\noindent bash ./formatanddestruct.sh --yes-i-am-insane --nuclear-option 2001:db8:1:1a0:539:7ff3:65:29a

    \end{ttfamily}
    \end{tiny}

\vspace{3ex} \hrule \vspace{3ex} }

Währenddessen las Alexandra den Inhalt der Anleitung laut vor.

\iaquote{»Nuclear Option: Transmit last resort destruction code 0x53634D4D to the specified IP address. This is EXTREMELY DANGEROUS and will very likely cause massive loss of data.«}

»Na super«, fand Orakel und drückte spontan die Entertaste. Statt der freudig erwarteten Explosion erschien aber nach zehn Sekunden nur die gewohnte Fehlermeldung auf dem Bildschirm.

\noindent \parbox{\textwidth}{ \vspace{3ex} \hrule \vspace{3ex}

    \begin{tiny}
    \begin{ttfamily}

\noindent nc: connect to 2001:db8:1:1a0:539:7ff3:65:29a port 32764 (tcp) failed: Connection timed out

    \end{ttfamily}
    \end{tiny}

\vspace{3ex} \hrule \vspace{3ex} }

Orakel war traurig. Er war sich sicher, an der einfachen Aufgabe des Abtippens gescheitert zu sein.

»Es liegt nicht an dir«, tröstete yury ihn. »Wahrscheinlich ist einfach die Adresse falsch. Die steht zwar in der Anleitung, könnte aber inzwischen veraltet sein.«

Nach kurzem Überlegen tippte Free die Adresse kurzerhand in einen Internetbrowser ein. Eine bunte, nicht mehr ganz zeitgemäß wirkende Website der Stadtregierung von Toronto informierte die Besucher darüber, dass dieser Rechner nicht öffentlich zugänglich sei. Mitarbeiter der Regierung wurden stattdessen darum gebeten, sich in das interne Netzwerk einzuloggen und es erneut zu versuchen. Ein kleiner Hinweis am unteren rechten Rand der Seite zog Alexandras Aufmerksamkeit auf sich.

\iaquote{»Dieser Server läuft seit zwei Jahren und fünfundzwanzig Tagen zuverlässig ohne Neustart dank hochwertiger kanadischer Software.«}

Die Adresse schien nicht veraltet zu sein, aber die vier Freunde befanden sich am falschen Ort.

»Ich glaube, wir müssen wieder nach Kanada«, stöhnte yury. »Ist da nicht auch das Raumschiff mit Äüörüzü gelandet?«

Alexandra senkte ihren Blick. »Ja, da trieb die Nachbarskatze ihr Unwesen. Zumindest so lange, bis das Militär sie überwältigt hat.«

»Free, du kannst du diese Zugangssperre doch bestimmt ganz einfach außer Kraft setzen«, war sich Orakel sicher.

»Nein, ich kann nicht zaubern«, widersprach Free. »Sämtliche Befehle von Rechnern außerhalb des Rathauses von Toronto werden kommentarlos verworfen.«

»Nun gut, dann machen wir eben einen kleinen Ausflug. Toronto ist ja noch recht bequem erreichbar«, fand Orakel.

Alexandra nickte. »Lasst uns zur Abwechslung mal den Dienst-PKW eines hochrangigen Pentagon-Mitarbeiters ausleihen.«

yury räusperte sich. »Ausleihen?«

»Aufbrechen, kurzschließen, mitnehmen.«

»Das wird nicht so einfach möglich sein«, sagte yury voraus. Er wollte gerade eine ausführliche Erklärung abgeben, als Orakel ihm auf die Schulter tippte.

»Alexandra nimmt dich nur auf den Arm. Wir haben den Autoschlüssel längst aus einem Büro geklaut.«

yury war baff. »Woher wusstet ihr denn, dass wir nach Kanada fahren müssen?«

Nun lachten seine Freunde. »Mensch, yury. Wir wollten ursprünglich zum Tower of Liberty fahren und dich da rausboxen«, erklärte Free. »Es ist ein schöner Zufall, dass wir den Schlüssel jetzt doch noch verwenden können.«

Fünf Minuten später sah das Büro aus, als habe dort noch nie jemand gearbeitet. Außerhalb des Pentagons standen auf einem Parkplatz vier unauffällige Gestalten und beluden den Kofferraum eines großen Geländewagens mit Automatikgetriebe.

»Alles dabei?«, fragte Orakel in die Runde. Die anderen nickten.

»Alles dabei«, bestätigte Alexandra. »Das nächste Kapitel unserer Odyssee beginnt. Aber mit dem Zerstörungscode in der Tasche–«

»Im Kopf«, protestierte Free.

»Schön. Also, mit dem Zerstörungscode im Kopf ist unsere Mission bereits so gut wie abgeschlossen.«

Orakel klappte den Kofferraum zu und begab sich auf den Fahrersitz.

»Fahrerwechsel in zwei Stunden«, schlug yury vor und machte es sich auf dem Beifahrersitz bequem.

»Alles klar«, bestätigte Orakel. »Alle anschnallen, Türen schließen, oh, und dem Autobesitzer zum Abschied zuwinken.«

Aus dem Gebäude kam tatsächlich ein Mann angelaufen, der wild mit den Armen fuchtelte.

»Auf Wiedersehen«, rief Alexandra. »Melden Sie den Wagen einfach als gestohlen, dann erhalten Sie ihn in ein paar Wochen zurück.« Sie sprang zu Free auf die Rückbank, riss die Tür hinter sich zu und betätigte die Zentralverriegelung. Diese Vorsichtsmaßnahme erwies sich jedoch als überflüssig, denn Orakel gab bereits Vollgas und jagte mit quietschenden Reifen quer über den Parkplatz. Die Ausfahrtsschranke hob sich nicht schnell genug, erwies sich aber glücklicherweise als nicht besonders stabiles Hindernis. Das Geräusch berstenden Holzes verkündete die erfolgreiche Abfahrt des verrückten Quartetts.

\begin{center}
	∞∞∞
\end{center}

»Benzin oder Diesel?«, fragte Free, als er von der überteuerten Tankstellentoilette zurückkehrte.

yury stand rätselnd vor einer Wasserstoff-Zapfsäule. »Weder noch«, murmelte er.

»Erdgas für das Auto, Wasserstoff für das Jetpack«, rief Orakel aus einiger Entfernung. »Wir sind eigentlich längst fertig, aber yury gefällt irgendetwas nicht.«

»Ich verstehe nicht, wieso die Zapfsäule auf einmal defekt sein soll«, erläuterte yury. »Ich habe auf Örz extra darauf geachtet, dass der Anschluss mit den auf der Erde verwendeten Systemen kompatibel ist.«

»Das ist normal«, wusste Alexandra. »Die Technologie steckt hier noch in den Kinderschuhen und ist von relativ häufigen Zapfsäulenausfällen geplagt.«

Free begab sich wieder auf die Rückbank. »Immerhin gibt es hier überhaupt Wasserstoff. Ich dachte schon, wir müssten am Raumschiff tanken.«

yury zuckte mit den Schultern und ging zurück zum Auto. »Unsere Jetpacks sind wieder gefüllt. Ich hatte nur Mitleid mit den Nachbenutzern.«

»Nachbenutzer«, wiederholte Free spottend. »Die Menschen fahren lieber mit riesigen Lithium-Ionen-Akkumulatoren durch die Gegend, deren Herstellung umweltschädlich ist und die nach ein paar Jahren ausgeleiert sind. Auf die Idee, Wasserstoff als Energieträger zu nutzen, kommen selbst viele vermeintliche Umweltschützer nicht.«

\begin{center}
	∞∞∞
\end{center}

Aus den Augenwinkeln nahm yury eine Bewegung im Rückspiegel wahr. Zunächst dachte er sich nichts dabei, aber dann fiel ihm auf, dass sich von hinten nur jemand nähern konnte, der noch dreister gegen die Geschwindigkeitsbegrenzung verstieß als er selbst. Vielleicht war es ein Sympathisant – jemand, der seine Meinung teilte, man müsse sich grundsätzlich nicht an Regeln halten, die nicht in SI-Einheiten definiert seien.

Free räusperte sich. Er hatte den Verfolger ebenfalls bemerkt, während Orakel und Alexandra auf der Rückbank ein Canasta-Turnier veranstalteten. »Leute, es gibt Ärger«, rief er mit Blick in einen Seitenspiegel.

\iaquote{»Objects in the mirror are closer than they appear.«} Der Autohersteller schien sich über Free lustig machen zu wollen. Nach seiner Schätzung war der schwarze Polizeiwagen noch ungefähr zwei Kilometer entfernt.

»Das ist bestimmt nur eine Routinekontrolle«, gab Orakel hoffnungsvoll zurück. Dann wendete er sich einfach wieder dem Kartenspiel zu. »Darf ich ausmachen?«

Alexandra brachte weniger Gelassenheit für die Situation auf. »Nein, das übernehme ich. Wenn die herausbekommen, dass der Wagen gestohlen ist, lassen die uns nicht nach Kanada einreisen.«

yury schreckte hoch. »Was zur Hölle hast du da schon wieder in der Hand?« Das war kein Kartenspiel, so viel stand fest.

»Das«, antwortete Alexandra genüsslich, während sie einen Metallbolzen entfernte, »ist eine Damoklesgranate.« \textit{*Plopp.*}

Mühsam beherrscht presste yury eine Entgegnung zwischen den Zähnen hindurch. »Findest du nicht«, sagte er in einer Mischung aus ohnmächtiger Wut und auswegloser Panik, »dass es Zeit wäre, die Kindersicherung für die Rückfenster zu deaktivieren und das Ding rauszuwerfen, bevor es mit uns explodiert?«

Free drückte einige Knöpfe, konnte seinen Blick dabei aber nicht von Alexandra wenden. Der Metallbolzen in ihrer linken Hand musste schon längst einen Zünder aktiviert haben.

»Keine Sorge«, beruhigte Alexandra ihre Freunde. »Sie hat einen Aufprallzünder und explodiert nur bei Erschütterung.«

\iathought{Ein Schlagloch genügt, dann kannst du nochmal in Ruhe über deine Worte nachdenken}, überlegte yury. »Wirf endlich die verdammte Granate raus!«

»Ich hätte nicht gedacht, dass du mich einmal dazu ermutigen würdest«, lachte Alexandra. Dann öffnete sie ein Fenster und schmiss die Granate auf die Straße.

Zwischen dem Polizeiwagen und den Verfolgten bestand noch genügend Abstand für das riskante Manöver. Eine grün gefärbte Explosionswolke verdeckte die Sicht, der Asphalt flog in alle Richtungen und die Polizisten führten erschrocken eine Vollbremsung durch. Unbeschadet, aber von der Straße abgeschnitten, blieben sie mit dem schwarzen Auto zurück.

»Da war Kupfer drin«, stellte Orakel fachmännisch fest.

»Ich hasse euch alle«, antwortete yury. Dann musste er lachen, und seine Freunde stimmten in das fröhliche Gelächter mit ein.

\begin{center}
	∞∞∞
\end{center}

Wie ein harmloser Tourist schloss yury die Fahrzeugtür und verriegelte das Auto. »Bitte denkt daran«, mahnte er, »dass wir keiner Fliege etwas zuleide tun möchten. Wir sind unauffällige, gerne auch naive Amerikaner, die sich auf ihrer Kanada-Tour natürlich auch das einmalig gestaltete Rathaus ansehen möchten. Anschließend wollen wir im nächsten McIsland ein paar Burger essen und den Freizeitpark besuchen.«

»Bekommen wir dann auch ein Eis?«, fragte Alexandra mit hoher Stimme.

yury lachte. »Natürlich. Ich bemerke schon, das wird ein lustiger Ausflug.«

-----

\chapter{Das Rathaus von Toronto}

Im Rathaus herrschte an diesem Freitagnachmittag nur mäßiger Betrieb. Außerhalb der Schulferien waren nur wenige Touristen anwesend, die meisten Kanadier gingen ihrer Arbeit nach und die Büros leerten sich allmählich.

»Schönen Feierabend«, rief irgendjemand in Richtung der Rezeption.

»Danke, gleichfalls.«

Niemand schien die vier Freunde zu bemerken, die über eine große Wendeltreppe das erste Obergeschoss betraten. Einige Mitarbeiter der Stadtverwaltung kamen ihnen entgegen, ins Gespräch vertieft und in Gedanken längst mit ihren Wochenendaktivitäten beschäftigt. Ein älterer Herr tippte im Gehen auf seinem Smartphone herum, was Free zu der Überlegung veranlasste, ob er den Selbstzerstörungscode möglicherweise bereits über ein öffentliches WLAN senden konnte. Inzwischen stand die Gruppe vor einem Aufzug, und Free zog sein Örz-Smartphone hervor.

»Wir könnten eigentlich auch zu Fuß gehen«, fand Alexandra.

Free hob seine Augenbrauen. »Vierundzwanzig Stockwerke?« Er blickte zu Orakel und zögerte einen Moment. »Na ja, von meiner Seite aus wäre das kein Problem.«

Orakel grinste. »Von meiner Seite aus auch nicht.«

Da die anderen diese Äußerung für einen Scherz zu halten schienen, nahm er ihnen die Entscheidung kurzerhand ab. Als sich der Aufzug öffnete, trat er hinein, drückte alle fünfundzwanzig Etagenknöpfe nacheinander und sprang wieder hinaus.

»Ist das dein Ernst?«, fragte yury verdattert.

»Jup. Jetzt ist es nur noch eine Frage der Effizienz. Der Aufzug lohnt sich nicht mehr.«

»Es ist vor allem eine Frage der Gefahr«, murmelte Free. »Unauffälliger Tourismus sieht anders aus.«

Alexandra nickte. »Wir sollten von hier verschwinden, bevor das Chaos ausbricht. Apropos, kannst du den Code nicht einfach per Funk senden?«

»Wahrscheinlich nicht«, prophezeite Free mit Blick auf sein Smartphonedisplay. »Die Netzwerkstruktur ist hierarchisch aufgebaut; über das WLAN kann man nur die Smartphones anderer Gäste erreichen.«

Er versuchte trotzdem testweise, den Code an alle erreichbaren Geräte zu senden. Dies führte jedoch nur zu Fehlermeldungen und zu einem Kopfschütteln von yury.

»Mein Smartphone läuft noch, die Internetseite der Stadtverwaltung ist noch erreichbar und die elektronisch gesteuerte Beleuchtung ist, wie man sieht, noch in Betrieb«, fasste yury zusammen. »So, da haben wir das Treppenhaus. Ich kann es kaum erwarten, hunderte Treppenstufen zu laufen.«

Er machte keinen allzu begeisterten Eindruck. Orakel hingegen zog entschlossen die Tür auf und begab sich auf den Weg, dicht gefolgt von Free und Alexandra. Den Abschluss bildete yury, der von dieser ungeplanten Aktion am wenigsten zu halten schien.

Irgendwann blieb Orakel stehen, allerdings nicht vor Erschöpfung. Er sah sich ratlos um. »Wo wollen wir überhaupt hin?«

»Irgendein leeres Büro finden«, erinnerte ihn Alexandra. »Darin haben wir ja inzwischen Übung.«

»Auf der Chefetage?«

yury wog nachdenklich den Kopf hin und her. »Der Bürgermeister hat sein Büro im ersten Stock. Ich weiß nicht, wie die Büros über das Haus verteilt sind, aber das oberste Stockwerk ist wahrscheinlich genau so gut geeignet wie jedes andere.«

»Was tut man nicht alles für einen schönen Ausblick«, befand Orakel und ging weiter nach oben. Auf dem Weg kamen ihnen noch einige Mitarbeiter entgegen, die über einen vermeintlich defekten Aufzug diskutierten, aber niemand beachtete die vier Eindringlinge.

\begin{center}
	∞∞∞
\end{center}

Allmählich ließ der Sonnenschein, der durch die Fenster fiel, nach. Da die künstliche Beleuchtung nicht aktiviert war, wurde es dunkel in dem leeren Büro; so geschah es seit Monaten jeden Tag. Auf einem Tisch stapelten sich Aktenordner, die beim Umzug vergessen worden waren. Niemand interessierte sich mehr für die irrelevant gewordenen Aufzeichnungen. Die Abteilung war zugunsten der internationalen Raumfahrtbehörde geschlossen worden; das Island-Regime strich hinter den Kulissen die Förderung für »überflüssige« Projekte. Ein Großteil der Wirtschaftsleistung wurde als Vorbereitung für den »großen Sprung nach vorn« nach Afrika umgeleitet. Dort galt es, Armut zu bekämpfen, der Bevölkerung auch in entlegensten Dörfern notfalls mit Gewalt den »Fortschritt« aufzuerlegen und in Äquatornähe riesige Raumhäfen aufzubauen. Dass Island dadurch ein Zeitalter moderner Sklaverei ausrief, anstatt diese zu beseitigen, wagte kaum jemand auszusprechen. Die Wenigen, die es taten, wachten einige Tage nach ihrer Kritik ebenfalls in der Wiege der Menschheit auf und wurden dort zur Arbeit in den Fabriken und auf den Baustellen gezwungen. Solange es noch nicht die passenden Roboter gab, waren politische Gefangene ein brauchbarer Ersatz.

Der zuständige Hausmeister hatte längst aufgehört, die verlassenen Büros gründlich zu kontrollieren. Auf seinem abendlichen Rundgang machte er sich Gedanken über seine berufliche Zukunft, kam zu ernüchternden Ergebnissen und warf gedankenverloren einen kurzen Blick in alle Räume, bevor er seine Runde beendete, das Gebäude verließ und die Alarmanlage in Betrieb nahm.

»Schönen Feierabend«, sagte er zu sich selbst.

\begin{center}
	∞∞∞
\end{center}

Vorsichtig schob yury die Schranktür ein paar Zentimeter zur Seite. Er blickte sich in der Abenddämmerung um. Ein Vorzug des Ostturms war definitiv der wunderschöne Sonnenuntergang hinter den Hochhäusern. Während er noch vorsichtig den Schrank verließ, polterte an der gegenüberliegenden Wand eine Blumenvase zu Boden. yury riss die Augen auf.

»Mensch, Free«, schimpfte Orakel leise. »Du kannst froh sein, dass das Ding nicht zerbrochen ist.«

»Haben wir eine Pfütze auf dem Teppich?«, fragte Alexandra besorgt.

Free tastete auf dem Boden herum. »Nein, es ist nichts passiert.« Behutsam stellte er die Vase zurück an ihren Platz. »Wo ist überhaupt yury?«

Der rollte mit den Augen, schloss den Schrank hinter sich und begab sich zu seinen Kollegen. »Hier bin ich. Orakel, ich glaube, wir könnten den Örztöp gebrauchen.«

Orakel nickte und griff nach seinem Rucksack. »Das Netzwerkkabel hat Free.«

Von einer Raumwand war ein leises Klicken zu hören. Free kehrte mit einem Kabelende zurück, schloss den Örztöp daran an und nahm ein Stromkabel von Alexandra entgegen. Der Bildschirm leuchtete in der Dunkelheit und blendete die Betrachter.

»Ich glaube, wir sollten das Ding lieber in einem Schrank betreiben«, sagte yury. »Das Licht ist vielleicht von außen zu sehen.«

Die Kabellänge genügte knapp für dieses Vorhaben; die Freunde verringerten zudem die Displayhelligkeit und sahen gespannt dabei zu, wie sich der Internetbrowser öffnete. Free tippte eine Adresse in das Suchfeld ein und bestätigte die Eingabe mit der Entertaste.

»Bitte geben Sie Ihren Benutzernamen und Ihr Passwort ein«, meldete sich der Server, der zuvor jede Kommunikation verweigert hatte. Orakel konnte einen kurzen Freudenschrei nicht unterdrücken.

»Soll das heißen, wir können hier und jetzt die gesamte elektronische Infrastruktur der Erde lahmlegen?«, erkundigte sich Alexandra.

»Möglich«, sagte yury.

»Klar«, sagte Free. Er öffnete eine Kommandozeile, wobei ihm die grüne Phosphorschrift auf schwarzem Hintergrund erneut einen belustigten Blick seines Vorredners einbrachte. Er tippte kurz auf der Tastatur herum, schickte den Befehl jedoch nicht ab.

\noindent \parbox{\textwidth}{ \vspace{3ex} \hrule \vspace{3ex}

    \begin{tiny}
    \begin{ttfamily}

\noindent Bash-Code, eine Zeile Netcat 0x53634D4D auf TCP 32764

    \end{ttfamily}
    \end{tiny}

\vspace{3ex} \hrule \vspace{3ex} }

»Worauf wartest du?«, fragte Alexandra nervös.

»Ich weiß nicht«, stammelte Free. »Das fühlt sich auf einmal an, als würde ich mit dem Abschicken dieses Befehls einen Weltkrieg auslösen. Wie ist das überhaupt moralisch zu bewer–«

\textit{»Klick.«}

»Alexandra!«, riefen Orakel, yury und Free gleichzeitig. Dann blickten sie voller Spannung auf das Display. Nach zehn Sekunden erschien eine Antwort.

\noindent \parbox{\textwidth}{ \vspace{3ex} \hrule \vspace{3ex}

    \begin{tiny}
    \begin{ttfamily}

\noindent nc: connect to 2001:db8:1:1a0:539:7ff3:65:29a port 32764 (tcp) failed: Connection timed out

    \end{ttfamily}
    \end{tiny}

\vspace{3ex} \hrule \vspace{3ex} }

»Würde eigentlich nicht genau dieselbe Meldung erscheinen, wenn die Zerstörung erfolgreich gewesen wäre?«, fiel yury ein.

»Nein«, widersprach Free. »Der Befehl kann erst gesendet werden, wenn eine Verbindung aufgebaut wurde. Die kommt aber gar nicht zustande.«

yury las sich die Fehlermeldung erneut durch und zeigte auf den Bildschirm. »Sie wird aber auch nicht explizit abgelehnt.«

Die vier Freunde grübelten eine Weile vor sich hin. Schließlich griff Free den Gedanken auf.

\noindent \parbox{\textwidth}{ \vspace{3ex} \hrule \vspace{3ex}

    \begin{tiny}
    \begin{ttfamily}

\noindent Bash-Code, nmap-TCP-Portscan. Ergebnis: Port 80 (HTTP) und 443 (HTTPS) sind offen, Port 32764 ist "drop"/"timeout" und alle anderen sind explizit "closed"

    \end{ttfamily}
    \end{tiny}

\vspace{3ex} \hrule \vspace{3ex} }

»Guter Hinweis, yury. Auf allen anderen Kanälen wird die Verbindung explizit abgelehnt oder angenommen…«

\noindent \parbox{\textwidth}{ \vspace{3ex} \hrule \vspace{3ex}

    \begin{tiny}
    \begin{ttfamily}

\noindent Bash-Code, nmap-TCP-Portscan eines zufälligen anderen Rechners. Ergebnis: Ping erfolgreich, aber alle Ports explizit "closed"

    \end{ttfamily}
    \end{tiny}

\vspace{3ex} \hrule \vspace{3ex} }

»…und auch die anderen Computer im Netzwerk lehnen ausdrücklich jeden Verbindungsversuch ab. Nur der eine Server mit der Adresse \iaquote{2001:db8:1:1a0:539:7ff3:65:29a} fällt aus der Reihe, weil er überhaupt nicht auf die Anfrage reagiert.«

»Wir jagen also keinem Gespenst nach, sondern sind tatsächlich kurz vor dem Ziel«, interpretierte Orakel das Ergebnis.

Free zuckte mit den Schultern. »Du könntest recht haben, aber ich bin hier mit meinem Latein am Ende.«

Nachdenkliches Schweigen, gemischt mit leichter Verzweiflung, erfüllte den karg eingerichteten Raum. Die letzten Sonnenstrahlen strichen lautlos über die Wand; die rot leuchtende Scheibe verschwand hinter dem Horizont. Nur noch das Laptopdisplay sorgte für spärliche Beleuchtung.

\begin{center}
	∞∞∞
\end{center}

Es war schließlich Orakels knurrender Magen, der die vier Freunde aus ihren Überlegungen hochschrecken ließ. Prompt meldete sich auch dessen Besitzer zu Wort.

»Ich habe Hunger.«

yury lächelte. »Das haben wir schon gehört. Aber du hast in deinem Rucksack doch bestimmt genug Verpflegung für ein ganzes Bataillon.«

»Drei Tage«, korrigierte Orakel. »Wir können hier oben ziemlich genau drei Tage lang ohne Nahrungszufuhr von außen überleben.«

Alexandra bezweifelte das. »Ich kann mir ja vorstellen, dass du genug Müsliriegel für drei Tage mitgenommen hast. Aber irgendetwas müssen wir auch trinken.«

»Drüben sind vier Wasserhähne«, bemerkte Free und zeigte in Richtung der Tür. Dann erhob er sich. »Und genau dahin müsste ich sowieso mal.«

Während Free den Raum verließ, packte Orakel einige Müsliriegel und Studentenfutter aus. Er stellte für jeden eine Mahlzeit zusammen und verteilte dazu noch Karamellgebäck.

»Kaffeekekse«, sagte yury lachend. »In dem Regal stehen sogar noch Tassen, aber wir dürfen keine Spuren hinterlassen. Hast du vielleicht kleine Plastikflaschen übrig?«

Orakel hatte tatsächlich vier leere PET-Flaschen dabei. Diese waren zwar etwas zerknüllt, erfüllten aber ihren Zweck. Irgendwann kehrte Free zurück, freute sich über das unerwartet reichhaltige Büfett und öffnete auf dem Laptop einen Gebäudeplan.

»Das Foto habe ich gerade draußen von den Rettungsplänen gemacht«, erklärte er. »Und wie ich Orakel kenne, hat er im Treppenhaus dreiundzwanzig weitere Fotos vom Aufbau der anderen Etagen gemacht.«

Alexandra und yury blickten erstaunt zwischen dem Laptop und Orakel hin und her. Der grinste, verspeiste einen weiteren Keks und warf Free eine kleine Speicherkarte zu. »Ich habe eine Bodycam im Knopfloch«, sagte er amüsiert. »Sagt nicht, ihr habt das nicht geahnt.«

»Bei dir überrascht mich inzwischen wirklich gar nichts mehr«, entgegnete yury mit ungläubigem Enthusiasmus. »Dann können wir ja loslegen. Wenn ich das Problem richtig verstehe, müssen wir eigentlich nur den Server finden, gegebenenfalls einen Bildschirm und eine Tastatur daran anschließen, und schon haben wir den Zugang, der uns die ganze Zeit über verwehrt wird.«

\begin{center}
	∞∞∞
\end{center}

Auf den Gebäudeplänen waren sämtliche Räume nummeriert. Neben den Fotos erstellte der Örztöp ein riesiges Baumdiagramm aller über das Netzwerkkabel erreichbaren Computer. Jeder Computer hatte einen eigenen Namen, mit dem er sich gegenüber anderen Netzwerkteilnehmern auswies. Gleichzeitig hatte jeder Computer eine IP-Adresse, unter der er unabhängig von seinem Namen auch aus dem Internet weltweit erreichbar war. Bald stellte sich heraus, dass die IP-Adressen nicht zufällig vergeben wurden, sondern sich an den Raum- und Etagennummern orientierten. So ergab die kryptische Nummer des mysteriösen Servers auf einmal einen Sinn: Der Rechner befand sich im Westturm, in Raum 101 – nur die Etagenbezeichnung bereitete yury erhebliches Kopfzerbrechen.

»Das ist doch totaler Unfug«, beschwerte sich yury fluchend, als er das Stockwerk ausgerechnet hatte. »Die Stockwerksnummer wird berechnet, indem die Zahl 32767 vom drittletzten Block der IP-Adresse subtrahiert wird. Das Ergebnis lautet ›Minus zwölf‹.«

»Das wäre aus mehreren Gründen unlogisch«, pflichtete Alexandra ihm bei. »Erstens, weil die Räume mit den Hunderternummern im Ostturm-Erdgeschoss liegen. Zweitens, weil es weder dort, noch auf irgendeinem anderen Plan, einen Raum mit der Nummer 101 gibt.«

»Und drittens«, fuhr yury fort, »weil der Raum unter der Erde liegen müsste. Da ist vielleicht eine Parkgarage, aber die erstreckt sich bestimmt nicht zwölf Stockwerke tief unter die Erde.«

Orakel verschränkte die Arme. »Das heißt also, du möchtest die Suche aufgeben, weil dir das Ergebnis zu unwahrscheinlich erscheint.«

yury blickte verwundert zurück. »Nein, ich suche nur noch eine bessere Theorie.«

»Ich glaube, du hast den Nagel längst auf den Kopf getroffen«, mutmaßte Free. »Der Computer befindet sich in irgendeinem geheimen Keller tief unter der Erdoberfläche. Es ist heute technisch auch überhaupt kein Problem mehr, den dort unten an das Internet anzuschließen.«

Alexandra lächelte skeptisch. »Klar. Island hat einen zwölfstöckigen Keller angelegt, um einen Computer zu verstecken, den er einfach abschalten oder vom Internet trennen könnte, falls er ihn unerreichbar machen möchte.«

Da sie sich nicht einig wurden, trennte sich die Gruppe auf. Orakel und Free, die nicht länger warten wollten, begaben sich entschlossen und optimistisch auf den Weg ins Erdgeschoss. Derweil zerbrachen sich Alexandra und yury weiter den Kopf darüber, wie die merkwürdige Adresse zu interpretieren sei.

\begin{center}
	∞∞∞
\end{center}

Unten angekommen, liefen Free und Orakel durch den unbehaglich düsteren, ausgestorbenen Empfangsbereich hindurch zum Westturm. Da ihnen nichts Besseres einfiel, betraten sie ein Treppenhaus, wurden dort jedoch nicht fündig.

»Vielleicht hatte Alexandra doch recht, und wir suchen nach etwas, das es nicht geben kann«, schätzte Free mit Blick auf die karge Betonwand, die den Abschluss des Treppenhauses darstellte.

»Es gibt mehrere Treppenhäuser«, erinnerte ihn Orakel. »Wir sollten uns zumindest ein bisschen umsehen.«

Als sie bereits die Suche aufgeben wollten, bemerkten sie, dass die Betonwand in einem der Treppenhäuser nicht vollständig das untere Ende abschloss. Ein dünner Spalt trennte das linke Ende der Wand von einer Seite des Raumes. Orakel tippte Free auf die Schulter und griff kräftig nach der Wand, als handele es sich um eine Schiebetür aus Polystyrol. Umso verblüffter starrte Free die vermeintliche Wand an, als diese auf einer Bodenschiene lautlos zur Seite glitt und einem neuen Hindernis wich.

»Eine Stahltür«, stellte Orakel überflüssigerweise fest. »Ich glaube, hier kommen wir nicht weiter.«

Free benötigte einige Sekunden, um wieder einen klaren Gedanken zu fassen. »Wir müssen da aber durch«, beharrte er dann auf dem ursprünglichen Plan. »Der Fingerabdrucksensor lässt sich bestimmt überlisten.«

»Mit Magie?«, witzelte Orakel. »Ich sehe da keinen USB-Anschluss zum Hacken.«

»Nein«, erklärte Free, »mit Folie und dem Laserdrucker in unserem Büro.« Ohne eine Antwort abzuwarten, begab er sich auf den Rückweg.

Damit gab sich Orakel jedoch noch nicht zufrieden. »Moment mal«, hakte er nach, während er Free hinterher lief. »Woher willst du denn den Original-Fingerabdruck nehmen?«

»Darüber habe ich mir noch keine Gedanken gemacht, das wollte ich gleich yury und Alexandra fragen. Vielleicht finden wir im Plenarsaal ein paar Fingerabdrücke auf der Tastatur des Bürgermeisters. Der wird ja wohl Zutritt haben.«

Als die beiden das Treppenhaus des Ostturms betraten, hörten sie Stimmen von oben. Starr vor Schreck stand Free auf der Türschwelle, stolperte zurück gegen Orakel, presste mit aufgerissenen Augen einen Finger auf seine Lippen und schloss leise die Tür. Die beiden überlegten nicht lange, bevor sie auf Zehenspitzen das Weite suchten.

\begin{center}
	∞∞∞
\end{center}

Alexandra schlich lautlos in das Büro, schloss leise die Tür und lehnte sich kreidebleich an die Wand.

»Du siehst aus, als hättest du auf der Toilette ein Gespenst gesehen«, merkte yury an.

»Das kann man fast so sagen«, stammelte Alexandra. »Ich war gerade unten und wollte nachsehen, was Orakel und Free so lange machen. Als ich durch die Empfangshalle geschlichen bin, lief auf einmal Floating Island an mir vorbei. Der schläft doch im Weltherrscher-Hochhaus in Washington.«

Es fiel yury sehr schwer, ihr zu glauben. »Allerdings. Du willst mich wohl auf den Arm nehmen. Ich gönne mir jetzt eine Mütze Schlaf, damit ich nicht ebenfalls anfange, zu halluzinieren.«

Ohne anzuklopfen, platzten in diesem Moment Free und Orakel durch die Tür herein und zogen diese schnell wieder hinter sich zu.

»Ihr glaubt nicht, wer hier nachts durch das Gebäude spukt«, keuchte Orakel vollkommen außer Atem.

Free griff hastig nach seiner Wasserflasche und trank diese in einem Zug leer. Dann kaute er auf seinen Fingernägeln herum, bis yury ihm mitleidig Kekse in die Hand drückte.

»Island natürlich«, behauptete yury, als handele es sich um eine Selbstverständlichkeit. Das brachte ihm drei vollkommen verwirrte Blicke ein. »War nur Spaß. Ich habe das bis gerade selbst nicht geglaubt. Alexandra hat ihn auch gesehen und wäre ihm beinahe begegnet.«

»Was hat Island hier zu suchen?«, ärgerte sich Orakel. »Fehlt nur noch, dass–«

In diesem Moment klingelte das Telefon. Ein tiefer Schreck durchfuhr die Freunde. »Nein, nein, nein, nein«, stotterte Orakel. »Nein. Bitte nicht.«

\ialoudspeaker{»Guten Tag, dies ist der Anrufbeantworter der Abfallwirtschaftsbehörde von Toronto. Leider rufen Sie außerhalb unserer Geschäftszeiten an. Gerne können Sie eine Nachricht nach dem Signalton hinterlassen.«} Es piepte kurz.

Was auf das Piepen folgte, ließ allen Anwesenden die Haare zu Berge stehen. Free zerdrückte die Kekse in seiner Hand; yury stieß vor Schreck die Blumenvase um, Alexandra krallte die Finger in den Teppich und Orakel verschluckte sich gehörig.

\ialoudspeaker{»Ich bin’s, Wolfgang. Ich weiß, wo ihr euch versteckt. Widerstand ist zwecklos. Nehmt den verdammten Hörer ab, oder ich jage euch zusammen mit dem Hochhaus in die Luft.«}

Als sei die Situation noch nicht wahnsinnig genug gewesen, wurde in diesem Moment die Tür aufgestoßen.

»Guten Tag, die Herren«, tönte es von hinten.

»Marcor Schreiner, was zur Hölle«, brüllte Alexandra, die mit der Situation überhaupt nicht zurechtkam.

»Das Spiel ist aus, wenn ihr nicht kooperiert.« Er trug zwei Stangen Dynamit in der linken Hand und ein billiges Plastikfeuerzeug in der anderen.

Niemand antwortete. Alle Blicke hingen gebannt an den Lippen des in unangenehmer Erinnerung gebliebenen Widersachers vom Fort-Knox-Einsatz.

»Ich rate euch dringend dazu, das Telefonat anzunehmen«, drohte er.

»Also schön«, sagte Orakel. Mit einem Druck auf die Freisprechtaste nahm er das Gespräch entgegen. »Wolfgang? Kenn ich net.«

\iathought{Nerven wie Drahtseile}, dachte yury bewundernd.

\iathought{Galgenhumor}, dachte Free.

\iathought{Leichtsinn}, dachte Alexandra, die mühevoll einige klare Gedanken fasste. \iathought{Wir haben noch die Jetpacks. yury hat im Pentagon bewiesen, dass das funktioniert.}

»Ähm, Wolfgang.«

»Wer?«

»\emph{Der} Wolfgang. So, genug gespielt. Ihr wollt Island stürzen? Wir auch. Wir haben allerdings eine ganz andere Motivation als eure Sentimentalität.«

Orakel ließ sich dadurch nicht im Geringsten beeindrucken. »Schön. Wisst ihr, wo Raum 500 ist?«

»Raum 500? Äh, nein. Was hat es damit auf sich?«

»In Raum 500 steht der Server, den wir alle suchen. Der Server, der die ganze Elektronik der Island-Regierung außer Gefecht setzen kann.«

»Oh, sehr gut. Ich kann das für euch ermitteln. Bitte habt einen Moment Geduld.«

Fünf Minuten vergingen, ohne dass jemand es wagte, einen Finger zu rühren. Die Situation war grotesk; selbst Marcor Schreiner schien über das weitere Vorgehen keine Klarheit zu haben.

»Raum 500«, tönte es dann aus dem Lautsprecher, »sind die Toiletten auf der vierten Etage des Ostturms. Wenn Island seinen Server wirklich dort versteckt hat, erklärt das zumindest unsere bisher erfolglose Suche in allen Räumen.«

\begin{center}
	∞∞∞
\end{center}

Es stellte sich heraus, dass Wolfgang sich ebenfalls im Gebäude befand. Mit der Unterstützung seines Kollegen Marcor Schreiner war er aus einem berüchtigten Hochsicherheitsgefängnis der Weltregierung, dem W.I.N.D.O.W.S., geflohen. Das zuvor vom FBI unterstützte Verbrecherduo arbeitete nun gegen die Behörden und hatte es sich zum Ziel gesetzt, Floating Island zu stürzen. Dass ausgerechnet Alexandra, Orakel, yury und Free ihnen auf der letzten Etappe ihrer Mission begegneten, empfand Schreiner als äußerst lästig. Wolfgang hingegen war seit einer halben Ewigkeit darauf versessen, die vier »Nervensägen«, die er längst als seine persönlichen Erzfeinde betrachtete, ausfindig zu machen und für seine Zwecke einzuspannen. Dabei hatte er allerdings die Rechnung ohne Orakels Einfallsreichtum gemacht.

»He, die Tür ist abgeschlossen«, rief jemand aus Raum 500.

»Ich weiß«, antwortete Orakel, darauf bedacht, nicht durch lautes Rufen zusätzlich noch Floating Island auf sich aufmerksam zu machen.

Hinter der Tür begann eine Diskussion, deren Hitzigkeit mit zunehmendem Problembewusstsein anstieg. »Du, ich glaube, der Vielfraß hat uns eingesperrt. Ausgerechnet der.«

»Eingesperrt? Aber der Server–«

»Das ist ein verdammter Spülkasten, du Trottel.«

Orakel räusperte sich deutlich. »Entschuldigen Sie, die Herren.« Sofort trat Stille ein. »Wie Sie demnächst bemerken werden, befindet sich ein kleiner Essensvorrat auf dem Spülstein. Zudem haben Sie Zugang zu fließendem Wasser und befinden sich auf einer Etage, die ab Montag wieder aktiv genutzt wird.«

»Wir können auch einfach die Polizei rufen«, rief Marcor Schreiner, der sich über seine tatsächlichen Optionen noch immer nicht im Klaren war.

»Selbstverständlich wäre das möglich«, dozierte Orakel, »aber das einzige mobile Telefon, das Ihnen zur Verfügung stand, habe ich – zugegebenermaßen nicht besonders umweltgerecht – bei meinem Toilettengang über die Kanalisation entsorgt.«

Wolfgang protestierte lautstark. »Hör endlich auf, uns zu siezen. Der ganze Quatsch ist total unmoralisch. Du musst die Polizei rufen, wenn wir dich dazu auffordern.«

Mit einem Lachen trat Orakel einen Schritt näher an die Tür heran. »Mensch Wolfgang. Den Gefallen würde ich dir sogar tun.«

Man konnte hören, wie jemand von der Tür zurückwich. Dem ehemaligen Verfolger zuckte ein Schreckensszenario durch die Gedanken. »Bloß nicht. Die bringen mich um.«

»Deshalb«, antwortete Orakel gelassen, »glaube ich, dass wir uns einig sind. Keine Polizei, kein Stress. Wir stürzen Island, und was ihr danach macht, kann uns egal sein. Ich verspreche euch, dass jemand euch da in spätestens einer Woche herausholt. Vermutlich bereits übermorgen.«

Eine leise Diskussion zwischen Wolfgang und Marcor Schreiner folgte, bevor Wolfgang sich wieder meldete. »Wir akzeptieren deine Bedingungen. Unter einer Voraussetzung.«

»Glaubst du wirklich, ihr könntet in eurer Situation Forderungen stellen?« Orakel schmunzelte.

»Bitte, Orakel.«

»Dann mal raus mit der Sprache.«

Diesmal meldete sich Marcor Schreiner zu Wort. »Wir brauchen irgendeinen Zeitvertreib, aber bitte nicht den verdammten Pinball-Gameboy.«

Fünf Minuten später schob Orakel etwas unter der Raumtür hindurch. »Mit freundlichen Grüßen von Alexandra.«

»Du willst mich wohl auf den Arm nehmen«, rief Schreiner entrüstet. »Ich lese grundsätzlich keine Bücher.«

»Es sind auch Bilder dabei«, beruhigte ihn Orakel, »aber wenn du die Einrichtung des Raumes interessanter findest, kannst du stattdessen auch zwei Tage lang die Wände anstarren. Tschüs.«

\begin{center}
	∞∞∞
\end{center}

Mit einem zufriedenen Lächeln auf dem Gesicht kehrte Orakel zurück in das leere Büro. »Ein bisschen Bildung kann den beiden Banausen gar nicht schaden.«

»Du hast das wirklich genial gelöst, Orakel«, lobte Alexandra. »Ich dachte schon, wir müssten mit den Jetpacks durch die Fenster fliehen.«

»Das wäre sehr schade gewesen«, fand Free, »denn dann hätten wir die Aktion hier abbrechen müssen.« Dann kam er auf das ursprüngliche Gesprächsthema zurück. »Wenn das mit Fingerabdrücken nicht funktioniert, können wir vielleicht den Aufzug hacken.«

»Das ist lebensgefährlich«, protestierte yury.

»Vor allem, wenn man dabei den Aufzug verlässt«, bestätigte Free. »Wir können zumindest brav in der Kabine bleiben.«

»Willst du den Wartungsdienst anrufen und denen etwas von nächtlichen Wartungsarbeiten erzählen?«

»Nein, nein«, erklärte Orakel, »hinter dem Rezeptionstisch hingen zwei Aufzugschlüssel für die Feuerwehr.« Er klimperte mit einem kleinen Schlüsselbund.

Das schien tatsächlich ein durchführbarer Plan zu sein, und alle stimmten dem Vorschlag zu. Vorsichtig verließen sie den Büroraum und traten auf den Gang hinaus. Spärliche Notbeleuchtung wies den Weg zu den Treppenhäusern; niemand war zu sehen oder zu hören.

»Hoffentlich ist Island gerade in irgendeinem Obergeschoss beschäftigt«, flüsterte yury. Er öffnete leise die Tür zu dem Treppenhaus, das vom Haupteingang am weitesten entfernt war. Vollkommene Stille verhieß Hoffnung; die Gruppe trippelte in Richtung der Stufen und begab sich abwärts. Vor dem Ende jedes Abschnitts blieben alle lauschend stehen; an jeder Etagentür huschten sie eilig vorbei. Nichts ließ darauf schließen, dass sich der Diktator im Gebäude befand – dieser Umstand trug allerdings kaum zur Erleichterung der vier Freunde bei. Falls Island das Gebäude nicht verlassen hatte, konnte er jederzeit wie aus dem Nichts auftauchen.

Vor dem letzten Stufenabschnitt blieb yury erneut stehen. »Wir haben einen Vorteil gegenüber Island«, stellte er flüsternd fest. Dann lauschte er eine gefühlte Ewigkeit ins Treppenhaus hinein, ohne sich zu bewegen. »Er weiß vermutlich nicht, dass wir hier sind, denn sonst wäre das Gebäude längst von der Polizei gestürmt worden.«

»Und wenn er von unserer Anwesenheit nichts weiß«, schlussfolgerte Alexandra, »dann nimmt er auch keine Rücksicht auf seine eigene Lautstärke.« Sie trat an yury vorbei, drückte die Tür einen Spaltbreit auf, spähte nach links und rechts, wartete einen Moment und verließ dann das Treppenhaus. Dicht hinter ihr folgten yury, Orakel und Free.

Der Aufzug, so wussten die Freunde, würde sich nicht lautlos öffnen lassen. Außerdem stand weder fest, ob man mit einem Aufzug überhaupt die angeblich 12 Stockwerke tiefe Geheimetage erreichen konnte, noch, welche der Aufzüge hierfür geeignet waren.

Orakel zeigte in Richtung des Treppenhauses, in dem sich die versteckte Stahltür befand. Dann übernahm er die Führung. Nachdem er yury und Alexandra die bewegliche Geheimwand und den verschlossenen Zugang gezeigt hatte, verließ er das Westturm-Treppenhaus und ging auf den nächsten Aufzug zu. Dort wartete bereits Free, der seine Stirn gegen die Aufzugtür lehnte und mit einem Auge durch den Spalt blickte.

»Wir haben Glück. Der Aufzug befindet sich auf dieser Etage, und eigentlich ist das am Feierabend ja auch nicht allzu ungewöhnlich.«

Dennoch ließ sich ein helles \ialoudspeaker{»Pling«} nicht vermeiden, als der Aufzug sich der Macht des Feuerwehrschlüssels beugte und außerhalb seiner Betriebszeiten öffnete. Der auch an belebten Tagen hörbare Ton wirkte nach stundenlanger Stille ohrenbetäubend. Fast panisch blickten sich die zusammengezuckten Besucher auf der Etage um. Die Aufzugtüren wichen mit einem ähnlich unangenehm lauten Geräusch zur Seite; irgendjemand vergaß für einige Sekunden seinen Atheismus und flüsterte ein Gebet.

Orakel hastete voran in die zu allem Überfluss hell erleuchtete, geradezu blendende Kammer. Noch bevor Alexandra als Letzte die Türschwelle überschritten hatte, schob er eilig den Schlüssel ins Schloss, drehte diesen um und drückte die »Tür schließen«-Taste so fest gegen die Wand, dass das Blut aus seinem Daumen wich.

»Ich dachte immer, die Taste hat keine Funktion«, wollte yury diesen Versuch kommentieren. Bevor er diesen Satz ausgesprochen hatte, schloss sich die Tür bereits.

»Normalerweise stimmt das«, sagte Alexandra, »aber wir haben Sonderrechte.« Sie zeigte auf den Schlüssel, der im Aufzuglicht seine Schönheit preisgab und golden glänzte.

Die vier Freunde standen anschließend mit Blick auf die Etagenwahlknöpfe stumm und zunehmend ratlos minutenlang in der geschlossenen Aufzugkabine.

Schließlich brach Free das Schweigen. »Kann es sein, dass wir unsere tollen Sonderrechte gar nicht für unsere Zwecke nutzen können? Es gibt keine Knöpfe für negative Etagen.«

»Kein Wunder«, sagte yury. »Es gibt ja auch keine negativen Etagen.« Hundertprozentig überzeugt hatte ihn auch die versteckte Stahltür mit dem Fingerabdrucksensor nicht. Schließlich konnte sich dahinter genauso gut eine Besenkammer befinden. Oder eben der gesuchte Server, der dort ganz ohne Aufzug oder Treppen erreichbar sein konnte. Die negative Stockwerkszahl konnte einfach ein Platzhalter ohne tiefere Bedeutung sein.

»Was befindet sich dann unter uns im Aufzugschacht?«, fragte Orakel.

»Vermutlich eine Auffangkonstruktion aus Stahl. Eine Art Sprungtuch für Aufzüge, die nur benötigt wird, falls alle Seile reißen.« yury drehte sich überrascht zur Seite. »He, Alexandra, was hast du vor?«

Alexandra stand mit beiden Füßen auf dem Metallgeländer vor dem Rückspiegel des Aufzugs. Bevor irgendeiner der anderen Anwesenden verstand, was sie plante, war die Deckenverkleidung bereits geöffnet. Ein kühler Luftstrom zog aus dem kalten, dunklen Aufzugschacht herab durch die Kabine. Wortlos zog sich Alexandra durch die Öffnung, stellte sich aufrecht und blickte nach oben. Aus jeder Etage fiel ein dämmriger Lichtschein in den Schacht, bis in schwindelerregende Höhe. Alexandra löste abrupt ihren Blick von der Aussicht; Gleichgewichtsprobleme durfte es jetzt auf keinen Fall geben.

Orakel stieg nun ebenfalls auf das Geländer, um besser sehen zu können, was über seinem Kopf vorging. In diesem Moment legte sich Alexandra flach auf den Bauch und schob ihren Kopf vorsichtig über den Rand des Aufzugs hinweg. yury sah das und krampfte zusammen.

»Bleib um Himmels Willen innerhalb des Aufzugblocks«, sagte er gerade laut genug, um Alexandra mit seinen Worten zu erreichen. »Wenn sich das Ding bewegt, bekommst du möglicherweise ein Gegengewicht ins Gesicht.«

»Ich will nach unten sehen und habe kein Interesse daran, zu stolpern«, entgegnete sie. »Da geht es nämlich wirklich tief nach unten.«

Sechs Augenbrauen schnellten nach oben. Free hakte nach. »Wie tief kannst du ungefähr sehen?«

»Zwei Stockwerke vielleicht. Ich glaube, da sind keine Türen. Es ist zumindest stockduster da unten.«

Orakel kletterte nach oben und drückte Alexandra einen kleinen Stein in die Hand. Dann kehrte er schnell in die Kabine zurück. Die ganze Aktion war ihm nicht wirklich geheuer.

Mit der Stoppuhrfunktion ihres Smartphones wollte Alexandra die Fallzeit des Steins bestimmen. Sie ließ ihn los und lauschte; niemand wagte es, ein Wort zu sagen. Ungefähr vier Sekunden später hörte Alexandra, wie der Stein aufschlug. Sie erhob sich und kletterte in die Kabine zurück, wo sie gespannt erwartet wurde.

»Drei komma sieben Sekunden.«

yury starrte kurz die Wand an. »Vierundvierzig Meter mindestens für drei Sekunden. Wenn du zu spät mit der Zeitmessung begonnen hast, bis zu hundert Meter. Das wären viereinhalb Sekunden.«

Das waren äußerst erfreuliche Neuigkeiten. Free wollte Gewissheit haben. »Wie lange wäre der Stein bei einer Höhe von zehn oder zwanzig Metern gefallen?«

»Eineinhalb beziehungsweise zwei Sekunden«, antwortete yury ohne merkliche Verzögerung. »Ich glaube, das ist eindeutig. Unter uns befindet sich ein Abgrund, den es eigentlich gar nicht geben dürfte.«

\begin{center}
	∞∞∞
\end{center}

Mit jeder Minute im Aufzug stieg die Gefahr einer zufälligen Entdeckung durch Floating Island. Das helle Kabinenlicht leuchtete auf den relativ dunklen Flur hinaus und wäre einem vorbeigehenden Betrachter vermutlich aufgefallen; zudem ließen sich Geräusche nicht vermeiden, während Free und yury an der Elektronik des Aufzugs herumbastelten. Oder genauer, während sie \emph{versuchten}, daran irgendwelche Änderungen vorzunehmen.

»Das wird so nichts«, befürchtete Orakel. »Wir machen gerade entweder alles kaputt oder erreichen nichts.«

Besonders Free war frustriert durch diese Entwicklung. Vor ihm befand sich ein Computer, der dazu in der Lage war, den Aufzug in eine beliebige Richtung zu befördern. Durch den Feuerwehrschlüssel stand diesem Vorhaben nicht einmal eine Sicherheitsvorkehrung im Wege. Nur die Taste, um die Abwärtsfahrt zu forcieren, fehlte. »Ich habe auch schon Aufzüge gesehen, die Steuertasten für den manuellen Betrieb im Innenraum anbieten. Eine für ›Aufwärts‹ und eine für ›Abwärts‹. Das fehlt hier vollkommen.«

»Der Knopf ist bestimmt wieder irgendwo versteckt«, mutmaßte Orakel. »Genau wie die Stahltür, hinter einer doppelten Wand.«

»Uns läuft die Zeit davon«, mahnte yury und blickte auf die Uhrzeitanzeige seines Smartphones. »Es ist gleich vier Uhr morgens, die Sonne geht in ein paar Stunden auf. Marcor Schreiner und Wolfgang versuchen bestimmt, sich zu befreien, und Island hat das Gebäude noch nicht verlassen.«

Free sah ihn schräg an. »Woher weißt du \emph{das} denn schon wieder?«

»Ich lausche die ganze Zeit nach draußen, und man kann von unserer Position aus durchaus mitbekommen, wenn jemand das Gebäude verlässt. Umgekehrt stehen wir auch gerade exponiert im Rampenlicht und verhalten uns nicht besonders leise.«

»Wir wissen nicht, was Island vorhat, und ob er vor Ablauf der Nacht wieder verschwinden möchte. Das wäre zumindest denkbar«, führte Alexandra aus, »da er ja auch heimlich mitten in der Nacht erschienen ist und hier durch die Gänge geistert.«

»Im schlimmsten Fall steht er längst oben vor einer der Aufzugtüren und hört uns zu.« Orakel schüttelte sich und fühlte sich auf einmal unangenehm beobachtet. »Wir müssen hier weg.«

Ein leichter Anflug von Panik breitete sich aus. Es schien nur einen Ausweg zu geben, und dieser befand sich in der Höhle des Löwen. Alexandra kletterte eilig auf das Geländer, zog sich erneut mit einem kräftigen Klimmzug nach oben in den Schacht, besann sich ihrer Verantwortung und versuchte mit schlotternden Knien, die nötige Gelassenheit für die Situation aufzubringen. Dass es ein paar Schritte weiter mindestens vierzig Meter in die Tiefe ging, trug nicht besonders zu ihrer Beruhigung bei.

Im Aufzug lief Orakel unruhig umher; Free hingegen saß auf dem Geländer, blickte mit nervösem Blick nach oben und drückte sich die Finger in die Handflächen. yury störte Orakel auf seinem Rundgang, indem er sich flach mit dem Rücken auf den Boden legte.

»Was machst du da?«, fragte Orakel und blieb stehen.

»Falls die gestoppte Zeit falsch ist, prallen wir womöglich auf das Metallgestell, oder direkt auf den Steinboden, falls die Sicherheitsvorkehrungen nicht ordentlich installiert wurden.«

»Machst du dir keine Sorgen um Alexandra?«

»Doch, natürlich. Trotzdem würde ich selbst gerne überleben. Notfalls ziehe ich das hier alleine durch. Eure Arbeit soll nicht umsonst gewesen sein.«

Orakel schmunzelte. »Sehr lustig«, entgegnete er. »Ich mag deinen Galgenhumor.«

yury kniff die Augen zusammen. »Diese Rückenlage ist jedenfalls ganz ohne Humor empfehlenswert, wenn Alexandra zu optimistisch bergab fährt. Apropos, bitte setzt die Kabine in Bewegung, ich komme mir auf der Eingangsetage im hell erleuchteten Aufzug vor wie auf einem Präsentierteller.«

»Festhalten«, warnte Alexandra mit gedämpfter Stimme. »Montagefahrten sind die vermutlich häufigste Ursache von Personenschäden in Hochhausaufzügen.«

»Na super«, fand Orakel. »Gut, dass heute Samstag ist.«

Bevor Free sich darüber aufregen konnte, setzte sich der Aufzug in Bewegung. Das damit verbundene Gefühl des freien Falls wirkte in der gegebenen Situation beängstigend; er schlug die Hände über dem Kopf zusammen und duckte sich. Keine zwei Sekunden später lag Orakel mit dem Bauch auf dem Boden.

»Andersherum«, presste yury zwischen den Zähnen hervor, blickte abwechselnd zur Seite und nach oben und konnte den Anblick der offenen Decke nur noch schwer ertragen. Er schloss die Augen.

Alexandra starrte derweil stur auf ihre Kontrollen. Um sie herum zogen die Wände nach oben vorbei, jede Form von Tür fehlte hingegen. Am Rand ihres Blickfelds gab es nichts außer Gestein, Gestein und Gestein. Ihr war, als sei sie schwerelos, zumindest ein bisschen, und als zöge mit dem Gestein auch ein Teil ihrer Hoffnung an ihr vorbei. Was so begann, konnte überhaupt nicht gut enden. Feuerwehrschlüssel, eine geheimer Raum zwölf Stockwerke tief unter der Erde, eine Fahrt auf dem Dach eines Aufzugs. Die ständige Angst, von einem Gegengewicht erschlagen zu werden, sagte sich Alexandra, war bei einer Abwärtsfahrt vermutlich irrational. Dafür gab es möglicherweise die Gefahr, den gefürchteten Klotz stattdessen ins Gesicht zu bekommen. Das hing ganz von ihrer Position ab, und dieses Bewusstsein ließ sie wie ein Häufchen Elend zusammenkauern.

-----

\chapter{Minus Zwölf}

Langsam bekam yury es wirklich mit der Angst zu tun. Orakel hatte sich inzwischen wie geheißen umgedreht und starrte in dieselbe Richtung. Die Anzeige des Aufzugs war nur für positive Zahlen ausgelegt und zeigte »246« an.

»Das ist wie Pinball, nur andersherum und mit acht Bit«, faselte Free, der offenbar auch nicht mehr ganz bei klarem Verstand war.

»Ich fühle mich, als hätte ich acht Bit intus«, antwortete Orakel, was ihm einen mitleidigen Blick von Free und einen spottenden Blick von yury einbrachte.

»Minus elf«, sprach Alexandra in die Kabine. »Es wird ernst.«

\iathought{Als handele es sich dabei um eine Neuerung}, dachte Free. Dennoch verkrampfte er und blickte gebannt auf die rote »245«.

»Man sollte anmerken«, fiel yury ein, »dass das Erdgeschoss die Nummer Eins trägt.«

Alexandra stoppte den Aufzug. Ihren Freunden in der Kabine jagte sie damit einen gehörigen Schrecken ein. »Entschuldigung, aber das sollten wir besprechen. Bis zu welcher Nummer muss ich fahren? Ich hatte beim Herunterfahren die Zahl ›244‹ ausgerechnet.«

»Die Zahl für das Display stimmt«, erklärte yury, ohne sich zu erheben. »Ich wollte nur anmerken, dass das nach unserer Zählweise das dreizehnte Untergeschoss ist.«

»Ich bin nicht abergläubisch, aber das gefällt mir nicht«, nörgelte Orakel. »Die Sache stinkt zum Himmel.«

Alexandra verkniff sich eine Beifallsbekundung und setzte den Aufzug nach einer kurzen Vorwarnung wieder in Betrieb.

\ialoudspeaker{»Pling.«}

Das Geräusch durchfuhr die vier Freunden nachhaltig; der Aufzug stand und bewegte sich keinen Millimeter mehr weiter. Alle rissen die Augen auf – Alexandra, weil ihre Kontrollen nicht mehr funktionierten, und die anderen, weil sich die Aufzugstüren selbsttätig öffneten. Vor den erstarrten Eindringlingen tat sich ein rabenschwarzer Gang auf, der exakt den Aufzugdurchmesser hatte und an den Seiten durch schwarzes Gestein abgeschlossen war. Die Decke und der Boden schienen aus Schiefer zu bestehen und gleichmäßig mit einer dünnen Schicht aus Kohlepulver bestreut worden zu sein.

»Was seht ihr?«, fragte Alexandra. »Ich kann den Aufzug nicht weiter fahren lassen.« Sie erhob sich von den Kontrollen, knickte dabei fast ein, weil ihre blutleeren Beine nachgaben, dehnte sich kurz, stieg durch die Deckenöffnung und starrte baff in die unendlich wirkende Tiefe des Ganges.

»Das ist der Vorraum zur Hölle«, riet Orakel. »Und der Teufel persönlich wartet da irgendwo auf uns.«

»Ich dachte, du glaubst nicht an so etwas«, stammelte Free, der seinen Blick nicht von der Schwärze lösen konnte. »Mir fehlt aber die nötige Sicherheit, um dir zu widersprechen.«

»Schluss mit dem Unfug«, beschloss yury. »Da will uns jemand zum Narren halten, oder Island hat den ganzen Zauber wirklich nur für sich selbst eingerichet.« Er trat auf den Gang hinaus und bemerkte, dass dieser gegenüber dem kalten Aufzugschacht eine recht angenehme Lufttemperatur bot. Es knirschte, als er seine Schuhspitzen in den Kohlenstaub grub und mit Druck zur Seite drehte. »Hier ist seit einer Ewigkeit niemand entlanggelaufen.«

Auch Alexandra widersetzte sich dem gruseligen Eindruck. Sie verließ entschlossen die Kabine, trat einmal mit Schwung seitlich gegen die Kohle, als wolle sie einen Fußball von sich stoßen, griff mit einer Hand in den Staub und ließ diesen zwischen ihren Fingern herab rieseln.

»Warm«, befand sie dann. »Es ist warm hier drin.«

»Na gut, Alexandra, du saßt immerhin im kalten Aufzugschacht«, gab Free zu bedenken. Er zog Orakel an einer Hand aus dem Aufzug und kniete sich draußen in den Staub. »Sogar an eine Fußbodenheizung wurde gedacht. Und jetzt erzählt mir bitte nicht, das sei Erdwärme.«

»Fünfzig Meter unter der Erde wäre das allerdings etwas merkwürdig«, stimmte yury zu. »Also sind wir entweder deutlich tiefer gefahren, es gibt in dieser Gegend irgendeine geologische Besonderheit oder der verrückte FBI-Agent hat wirklich Wasserrohre unter dem Gesteinsboden verlegt.«

»Das spräche für eine Täuschung«, erkannte Orakel. Er trat mit Wucht gegen die linke Seitenwand, schüttelte unzufrieden den Kopf, wiederholte den Vorgang auf der rechten Seite und sah Free an. »Massiver Fels, aber ich traue der Decke noch nicht. Ich bräuchte einmal deine Hilfe.«

Mit einer spontan hergestellten Räuberleiter erreichte Orakel die Decke. Er schlug mit der Faust dagegen und zog diese sofort schmerzerfüllt zurück.

»Ebenfalls massiv«, kommentierte Free das Geräusch. Orakel nickte missmutig, stampfte kräftig mit einem Fuß auf und zuckte dann mit den Schultern.

»Also gut«, sagte er, »wir befinden uns in einem Stollen. Wer weiß, vielleicht wird hier wirklich Kohle gefördert.«

»Das ergäbe keinen Sinn«, widersprach yury. »Selbst wenn Island in dieser geringen Tiefe auf einen Kohlevorrat gestoßen wäre, würde sich dessen Abbau nicht wirtschaftlich lohnen.«

Orakel lehnte sich mit einer flachen Hand gegen eine Seitenwand. »Der ganze Gang ergibt keinen Sinn.«

Alexandra zog ihr Smartphone hervor und aktivierte die Taschenlampenfunktion. Der Gang schien das Licht zu verschlucken, aber dadurch ließ sie sich nicht einschüchtern. Entschlossen ging sie voran in die Dunkelheit; die anderen folgten ihr zögernd.

\begin{center}
∞∞∞
\end{center}

Nach ungefähr fünfzig Metern endete der Gang in einer T-förmigen Kreuzung. Auf der linken Seite führte eine Treppe aus pechschwarzen Stufen nach oben, auf der rechten Seite knickte der Gang rechtwinklig ab.

»Wir müssten uns gerade unter dem Plenarsaal befinden«, schätzte yury. »Wir können die Treppen nach oben laufen und landen vermutlich vor einer Stahltür, wie ihr sie im Treppenhaus gefunden habt.«

»Ich glaube, der rechte Gang ist interessanter«, schätzte Orakel mit Blick in die Finsternis. Er ging ein paar Schritte dorthin, blickte um die Ecke und beleuchtete mit seinem Smartphone den Boden. »Da ist noch eine Treppe, aber diese führt nach unten.«

Als sie sich wortlos für den Weg nach unten entschieden und die Treppenstufen hinab schritten, bemerkten sie, dass der Boden nicht so unbetreten und verstaubt aussah wie der erste Gang. Die Treppe führte um mehrere Ecken nach unten, fast wie die oberirdischen Treppenhäuser. Je tiefer die vier Freunde hinab stiegen, desto wärmer wurde es, und die Dunkelheit wich sehr langsam einem dämmrigen Licht, das von unten hinauf schien.

Orakel verließ als Erster das schwarze Treppenhaus. Vor ihm erstreckte sich ein Felssteg, der zu beiden Seiten steil in die Tiefe abfiel. Er befand sich an der Decke einer riesigen, beleuchteten Halle, deren Maße er kaum abschätzen konnte. Als ihm bewusst wurde, dass er sich ohne Sicherung einen Meter entfernt vom Abgrund befand, überkam ihn ein Schwindelgefühl, und er stolperte bleich vor Entsetzen zurück in das Treppenhaus.

»Was um alles in der Welt«, staunte yury, »ist \emph{das} denn?!«

Arbeitsgeräusche drangen aus der Tiefe zu ihnen herauf. Mindestens vierhundert Meter unter ihnen arbeiteten Roboter und Menschen an einem überdimensionalen Metallbauwerk. Der Steinpfad schien eine Art Aussichtsplattform zu sein, von der kleine, geländerlose Wendeltreppen an den Seiten herab führten. Es sah so aus, als habe jemand absichtlich auf die Geländer verzichtet, um den imposanten Eindruck nicht zu beeinträchtigen.

Free setzte sich im Schneidersitz auf den Aussichtspfad. Mit weit geöffneten Augen starrte er die Konstruktion und die Arbeitsschritte minutenlang an. »Ich glaube, die bauen ein Unterseeboot.«

Alexandra schritt vorsichtig an ihm vorbei und blieb vor einer der Wendeltreppen stehen. »Die Treppe kommt so aber nicht durch den TÜV.«

»Ich habe keine Ahnung, wie man da einigermaßen sicher heruntergehen soll«, stimmte Orakel zu. »Mir wird schon bei dem Gedanken schlecht.«

»Wir haben die Jetpacks im Büro liegen lassen«, wurde yury in diesem Moment bewusst. »Weil wir uns sicher waren, dass wir im Keller keinen Bedarf dafür hätten.«

Free blickte sich in alle Richtungen um. »Eigentlich sind wir ja auch nur hier unten, um einen Computer zu finden. Wo der versteckt sein soll, ist mir aber ein Rätsel.«

Da der Weg in die Tiefe ohne Hilfsmittel zu riskant erschien, schlug Orakel vor, zur Kreuzung zurückzukehren. Alexandra und Free waren sofort einverstanden, yury grübelte lange vor sich hin. Schließlich nickte auch er, warf einen letzten Blick zurück in die Halle und machte sich auf den Rückweg.

»Wir sollten uns einmal die andere Treppe ansehen. Vielleicht gibt es hier noch mehr merkwürdige Geheimnisse«, argwöhnte Alexandra.

yury dachte laut vor sich hin. »Ich frage mich, was Island mit einem Unterseeboot vorhat. Er könnte damit immerhin buchstäblich ›abtauchen‹, wenn seine Regierungsstrukturen zusammenbrechen.«

»Wer weiß, wie stabil die Diktatur überhaupt noch ist«, bekräftigte Free den Denkansatz. »Besser als ein Bunker ist so ein Unterseeboot allemal.«

Dann waren sie an der Kreuzung angekommen. Auf der linken Seite leuchtete einladend die geöffnete Aufzugkabine, geradeaus ging es weiter in der Finsternis.

»Bist du eigentlich sicher, dass der Aufzug nicht weiter nach unten fahren kann?«, fragte Orakel, an Alexandra gerichtet.

»Nein«, antwortete Alexandra, »aber als wir angekommen sind, waren die Knöpfe auf einmal blockiert.«

yury ging geradeaus weiter. »Mir kommt gerade ein Verdacht.« Er leuchtete die Treppe hinauf, konnte jedoch nichts erkennen. Stufe für Stufe ging er nach oben, verschwand hinter einer Ecke, lief weiter und weiter. Noch konnte man seine Schritte deutlich hören. »Na also«, rief er dann. »Wir brauchen den Aufzug nicht mehr.«

Neugierig kam Orakel hinterher gelaufen; mit einigem Abstand folgten Free und Alexandra.

»Jackpot«, rief Orakel. »Free, das musst du dir ansehen.«

Free hob eine Augenbraue. Er ging um die Ecke herum, richtete seine Augen auf die Stelle, die Orakel ihm aufgeregt zeigte und pfiff anerkennend. »Serverraum 101«, stand in großen Buchstaben über einer dicken Brandschutztür.

Ohne lange zu zögern, griff yury nach der Türklinke und zog kräftig daran. Die Tür schwang zur Seite auf und gab den Blick auf einen riesigen Thronsaal preis. Von Kronleuchtern in ein schneeweißes Licht getaucht, glitzerten bunte Steine an allen Wänden. Geblendet kniffen die Freunde ihre Augen zusammen. Am Boden lud ein roter Teppich die Besucher zum Weitergehen ein; das Ende des Raumes war durch die Helligkeit nicht zu erkennen.

»Kneif mich, ich glaube, ich träume. Au! Orakel, das war doch nicht wörtlich gemeint.« Free zog seinen Arm zurück.

»Nie im Leben«, stieß yury heraus. »Nie im Leben hat Island diesen Palast in die Erde gebaut, um die größte Schwachstelle seines Regimes funkelnd zu präsentieren.«

Alexandra schmunzelte. »Doch, genau so sieht es aus.« Sie schritt würdevoll den Teppich entlang, der nicht zu enden schien.

Fassungslos folgten ihr Orakel, yury und Free. Rundum glänzten goldene Statuen; der Boden aus weißem Marmor spiegelte das Deckenlicht.

»Das ist doch kein Serverraum«, protestierte Free. »Das ist Blasphemie.« Im Gehen drehte er sich in alle Richtungen; zeitweise lief er rückwärts. »Sogar die Decke besteht aus Marmor. Wozu dient das alles?«

Als Alexandra abrupt stehen blieb, stießen sie alle zusammen und fielen chaotisch zu Boden. Orakel erwartete daraufhin lautstarken Protest von Alexandra, etwa in der Art »Passt doch auf, wo ihr hinlauft.« Stattdessen starrte Alexandra den Thron an, der sich gegen das grelle Licht abzeichnete. Auf einem kleinen, mit rotem Samt bedeckten Tisch stand ein schwarzer Computer, von dem ein Internetkabel und ein Stromkabel in den Boden verliefen. Ein kleines grünes Lämpchen blinkte an der Vorderseite und zeigte an, dass gerade Festplattenaktivität stattfand.

Hinter dem kleinen Tisch stand ein riesiger, rot gepolsterter Thron aus massivem Gold. Langsam gewöhnten sich die Augen der vier Freunde an die Helligkeit, und die Umrisse eines alten Bekannten zeichneten sich vor dem roten Stoff ab. Nach ungefähr einer halben Minute war deutlich erkennbar, dass dort niemand anderes saß als Floating Island höchstpersönlich.

Island knackte mit den Fingern. »Willkommen im Paradies.«

\begin{center}
∞∞∞
\end{center}

Alexandra fand als Erste die Sprache wieder. »Was hast du dir diesmal ausgedacht, du mieses–«

»Halt die Luft an, ich habe euch ein Angebot zu machen.« Mit einer theatralischen Geste fuhr Island fort: »Vielleicht ist euch gar nicht bewusst, wie viel Macht momentan in meiner Person vereint ist. Durch mich hält die gesamte Erdwirtschaft zusammen, die Nahrungsmittelversorgung und die Staatsgewalt auf jedem Kontinent. Wenn ihr mich einfach ausschaltet, setzt ihr acht Milliarden Menschen dem sicheren Hungertod aus und lasst die Weltbevölkerung in ihren letzten Stunden in grausamer Anarchie um ihr Leben kämpfen. Zu allem Überfluss haben meine Anhänger selbstverständlich noch die Kontrolle über alle Atomwaffen, die auf diesem Planeten jemals produziert wurden. Ihr schaufelt euch und euren Mitmenschen ein widerliches Grab, wenn ihr mich naiv absägt.«

Alexandra ballte die Fäuste und stieß einige Flüche aus, doch Island ging darüber hinweg, als habe er ihre Worte nicht gehört.

»Ich bin dazu bereit, die Erde aus freien Stücken in eure Obhut zu geben. Dann könnt ihr den Demokratiegedanken umsetzen, den ihr Spinner schon seit Jahren zu verwirklichen sucht.«

Wütend erklomm Alexandra einige Stufen und positionierte sich genau vor dem Thron. Auf Augenhöhe starrten sich Alexandra und der verhasste Diktator an; kein Blinzeln und kein Geräusch störten das minutenlange im Schweigen ausgetragene Duell.

Nach drei Minuten knurrte Alexandra etwas Unverständliches.

»Wie meinen, junge Dame?«

»Was– was willst du?«, zischte Alexandra. »Rück raus mit der Sprache, solange du noch kannst.«

Island amüsierte sich köstlich.

»Ahahahaha, du bist machtlos. Neben dir steht der Computer, und der Selbstzerstörungscode würde durchaus funktionieren. Ich habe nicht geblufft, aber eure vermeintliche Chance ist überhaupt keine, weil ihr auf mich und das System angewiesen seid. Was willst du tun, außer dir die Ohren zuzuhalten?«

»Sprich dein letztes Gebet, du Pfeife.«

»Nun gut. Ich verlange nur eine winzige Kleinigkeit von euch.«

Alexandra kniff die Augen zusammen. Das konnte ja heiter werden.

»Ich will«, ließ Island die Bombe platzen, »nach Örz.«

Stille. Fassungsloses Schweigen und genüssliche Auskostung des Moments standen sich Auge in Auge gegenüber.

\chapter{Dögöbörz Nüggät}

Es war nicht weiter ungewöhnlich, dass Island viele Feinde hatte. Auch der Diktator selbst wusste, dass ein beachtlicher Teil der Erdbevölkerung ihn lieber tot als lebendig gesehen hätte; statt tiefer Trauer würde man eines Tages in jauchzende Euphorie verfallen und das Ende der erbarmungslosen Weltherrschaft als Feiertag in den Kalendern verewigen. Island betrachtete seine Situation realistisch und verwirklichte mit kühler Härte eine bis ins kleinste Detail durchdachte Strategie.

Die vier Freunde waren nicht mehr als ein Werkzeug zum Erlangen weiterer Macht, und sie waren wie blutige Anfänger in die Falle gelaufen. Das Bewusstsein, dass ausgerechnet Alexandra ihm bei der Verwirklichung seiner Pläne helfen würde, zauberte ein spöttisches Lächeln auf das Gesicht des ehemaligen FBI-Agenten. Nur yury traute er zu, sich überhaupt im Grundsatz gegen die Erpressung zu richten und es auf die Verwirklichung der Drohungen ankommen zu lassen.

Tatsächliche Gefahr drohte jedoch von einer ganz anderen Seite. Wenn jemand Island berichtet hätte, wer zu diesem Zeitpunkt sein größter Widersacher war, dann hätte er diese Meldung als schlechten Scherz abgetan und den Überbringer der Botschaft zu lebenslanger Haft verurteilt. Niemand machte sich ungestraft über den großen Bruder lustig.

\begin{center}
	∞∞∞
\end{center}

Dögöbörz Nüggät schmiss entzürnt mit beiden Händen einen armdicken Platinbarren durch den Raum. Die Scheibe an der Rückseite seines Geschäfts hatte den sieben Gewichtseinheiten geballter Masse wenig entgegenzusetzen und zersprang hinter dem längst auf der Straße gelandeten Metallstück in tausende Teile. Dann stampfte Nüggät nach draußen und schlug die Tür hinter sich ins Schloss.

»Verdammter Kuhmist«, brüllte der sonst äußerst gelassene, freundliche Ladenbesitzer dem erschrocken stehengebliebenen Passanten ins Gesicht. »Wenn du wüsstest, was auf der Erde passiert, würdest du nicht so doof gucken!«

Der Äöüzz setzte ein vorsichtiges Lächeln auf und nickte zurückhaltend. Bloß nichts Falsches sagen – der Mann war offenbar vollkommen verrückt geworden, und die Polizei würde sich sicherlich bald um den Vandalen kümmern. Während Önguk sich langsam, rückwärts schleichend von dem zornig auf seinen Metallbarren eintrampelnden Verkäufer entfernte, registrierte er in den Augenwinkeln mit einer gewissen Befriedigung das Eintreffen der Notfallstreife.

Nüggät zuckte zusammen. Jemand hatte sich unbemerkt an ihn herangeschlichen und tippte ihm auf die rechte Schulter. »Entschuldigen Sie bitte«, erklang eine tiefe Stimme, »würden Sie uns freundlicherweise erklären, was Sie da gerade machen?«

Bevor er zu einer rechtfertigenden Antwort ansetzen konnte, meldete sich der zweite Polizist zu Wort. »Das muss ein neues Verarbeitungsverfahren sein. Der Barren ist zu dick und wird mit Spezialschuhen auf die richtige Größe getrimmt.«

»Unsinn«, fuhr Dögöbörz die beiden Ordnungshüter an. Er überlegte kurz. Sollte er ihnen die Wahrheit erklären? Wenn sie ihm tatsächlich Glauben schenkten, würde man detaillierte Erläuterungen verlangen und ihn tagelang ausfragen. Die Zeit drängte, also musste er sich stattdessen irgendeinen Quatsch ausdenken. »Äh. Ich führe katalytisch-selektive Korrosionsexperimente durch. Wie Ihnen sicherlich bekannt ist, existiert für die Produktion von Quarzwolle momentan kein industriell dissoziatives Verfahren, dessen Effizienz das Normalpotential einer Platinkathode übersteigt. Daher muss ein Kollisionsangriff auf die Struktur des Glases...« Nüggät zeigte mit wichtigtuerischem Blick auf die zerbrochene Scheibe. »...unweigerlich eine Kavitation des amorphen Materials durch quasifreie Elektronen nach sich ziehen, wodurch der Metallbarren nicht nur an Wert gewinnt, sondern zudem als kondensierte Materie die starke Wechselwirkung des Gluonenstroms umkehrt. Aber das haben Sie ja bereits in der Schule gelernt.  Bitte stören Sie mich nicht weiter.«

Der Mann war wirklich verrückt. Kopfschüttelnd verließen die beiden Polizisten die unwirkliche Szene und fuhren in ihrem Polizeiwagen davon. Dögöbörz Nüggät blickte ihnen noch eine Weile nach, dann griff er entschlossen nach dem Metallklotz, lief über die knirschenden Scherben und stieg durch die leere Fensteröffnung zurück in sein Büro.

Niemand schien etwas zu ahnen. Dabei musste doch jeder, der auch nur ein rudimentäres Verständnis von Strategie hatte, sofort erkennen, was auf Örs vor sich ging. Der lächerliche blaue Planet war wertlos für jemanden, der wirklich nach galaktischer Macht strebte. Und nun standen gleich \emph{zwei} Raumschiffe mit Überlichtantrieb auf der Steinzeitwelt. Wie naiv waren diese vier Touristen eigentlich, dass sie glaubten, sie könnten allein gegen dieses kranke Genie vorgehen? Wenn Island nicht selbst dazu in der Lage war, die Raumschiffe ans Ziel zu steuern, dann würde er eben kurzerhand die damit angekommenen Raumfahrer zur Mitarbeit zwingen. Ansätze für eine gelungene Erpressung gab es schließlich genug.

An dieser Stelle wollte Nüggät ansetzen. Ihm lag nichts an der Erde, aber sein Vermögen war in Gefahr. Island hatte Gold, und davon eine ganze Menge. Warum die Regierung von Örz nicht längst Handelsbeziehungen mit der Bevölkerung dieses Rohstoffplaneten aufgenommen hatte, war ihm bis heute ein Rätsel. Hier ließ sich ein Vermögen verdienen, und die Planetenbewohner würden sicherlich voller Begeisterung ihre Edelmetalle gegen wertloses Plutonium und veraltete Technik eintauschen.

Wenn ein Mensch von Örs mit der dort vorhandenen Goldmenge auf Örz eintraf, war der sorgsam über Jahre erwirtschaftete Ladeninhalt nur noch einen Bruchteil seines Kaufpreises wert. Island würde sich die Gelgenheit nicht nehmen lassen, im Stil der vier Freunde ein Vermögen im Imperium von NGC 6193 anzuhäufen. Nur mit dem kleinen Unterschied, dass niemand in der ganzen Wirtschaftsvereinigung genug Äzz besaß, um diesen überdimensionalen Schatz zu kaufen. Das politische System würde ihn zudem über Nacht zum einflussreichsten und mächtigsten Lebewesen im Umkreis von mindestens fünftausend Lichtjahren machen.

Dögöbörz Nüggät griff nach einer Schublade, auf deren Griff sich bereits dicker Staub gebildet hatte. Eine vierzehnbeinige Spinne krabbelte erschrocken zur Seite und verschwand in irgendeiner Ecke.

Eine, zwei, drei, vier, fünf, sechs Patronen. Nur Anfänger schossen mit Blei. Nüggät hingegen besaß stilecht glänzende, teflonummantelte Goldkugeln. Ein Paralysestrahler in der anderen Hand sollte unnötige Verletzungen vermeiden, aber man konnte nie wissen, was unterwegs passieren würde. Das Universum war groß, und die Galaxis ging mitunter unbarmherzig mit ihren schlecht vorbereiteten Bewohnern um. Wenn die vier Freunde nicht den Mumm hatten, sich gegen ihren Erpresser zur Wehr zu setzen – der Edelmetallhändler hatte mehr als genug Gründe, um keinen Cent auf das hohle Gerede des aufstrebenden Idioten zu geben. Island würde bedingungslos kooperieren, oder... nun, Nüggät hatte sich bereits eine amüsante »Endstation« für den Diktator ausgedacht.

\begin{center}
∞∞∞
\end{center}

Die vier Freunde standen mit Floating Island im Aufzug. Ganz ohne Manipulation von außen fuhr die Kabine langsam nach oben.

»Äüörüzü lebt?«, fragte Alexandra, und in ihrer Stimme schwang tiefes Misstrauen mit. »Wahrscheinlich in einem eurer Labors.«

Island nickte, dann schüttelte er energisch den Kopf. »Ja. Nein. Das Katzenviech hält ganz Kanada in Atem. Was meint ihr, warum ich Toronto für unser Wiedersehen gewählt habe?«

Orakel stellte sich die Erdkugel vor und zog in Gedanken eine Linie. »Vermutlich, weil es einigermaßen nah an Ottawa liegt. Wie weit reichen die Waffen des Raumschiffs?«

»Fünfundzwanzig Kilometer«, schätzte yury. »Aber nur, wenn man weiß, wie man sie einsetzen muss. Im Zeitungsartikel stand damals etwas von zehn Kilometern.«

»Mit Unterstützung der Bordcomputer müsste doch selbst eine Katze das Potenzial voll ausschöpfen können«, wandte Free ein.

»Ja«, sagte yury. »Ich könnte mir vorstellen, dass der Quantencomputer seine Unterstützung verweigert. Der hat nämlich ein historisch gewachsenes ambivalentes Verhältnis zu Katzen.«

Alexandra ging auf die Anspielung ein. »Solange wir nicht das Raumschiff betreten haben, ist Äüörüzü vielleicht gleichzeitig tot und lebendig.«

»Ihr scherzt wohl«, beschwerte sich Island.

\ialoudspeaker{»Pling.«}

Die Aufzugtüren glitten zur Seite; Island verließ mit schnellen Schritten den Aufzug. »Aber euch wird das Lachen noch vergehen. \iashout{Taxi!}«

Alexandra, Orakel, yury und Free sahen sich erstaunt um. Spalier stehende Soldaten der kanadischen Armee bildeten eine Gasse zum Ausgang. Ein hochrangiger Offizier sprach den Diktator an.

»Sir, Ihr Kampfpanzer steht bei Kemptville bereit. Der Highway 401 wurde wie gewünscht gesperrt; wir können das Ziel voraussichtlich in zweieinhalb Stunden erreichen.«

Zögernd folgten sie ihrem Erpresser, verließen das Rathaus und stiegen unter Aufsicht der Soldaten in eine schwarze Großraumlimousine der Internen Schutztruppe. Floating Island machte es sich auf dem Beifahrersitz bequem, besprach die Route mit dem Chauffeur und gab dann den Befehl zum Aufbruch.

\begin{center}
∞∞∞
\end{center}

Die Kolonne aus Polizei- und Militärfahrzeugen raste mit über zweihundert Stundenkilometern über den leeren Highway. An jeder Ein- und Ausfahrt versperrten Motorräder den Zugang zu der normalerweise sehr häufig genutzten Straße.

»Was haben Sie als Nächstes vor?«, erkundigte sich Free.

»Oh, da gibt es zwei Möglichkeiten«, erklärte Island. »Entweder nimmt das Katzenviech in eurem Raumschiff Vernunft an, oder wir entsenden ein paar Interkontinentalraketen nach Australien.«

Free runzelte verwundert die Stirn, ohne die Situation zu begreifen. »Ich dachte, das Raumschiff steht hier in Kanada.«

»Richtig.« Island nickte spöttisch. »Aber die Geiseln leben in Australien.«

Alexandra nahm sich zusammen. Wenigstens lebte das süße Tier noch, und sie war zuversichtlich, mit Äüörüzü verhandeln zu können.

»Wenn uns das ein bisschen früher eingefallen wäre, hätten acht Milliarden Menschen eine Zukunft«, statuierte yury trocken.

Orakel runzelte die Stirn und blickte yury in die Augen. »Weißt du eigentlich, was du da gerade gesagt hast?«

»Ja«, erwiderte yury mit verschränkten Armen, »Und genau so meine ich das auch. Mit Erpressern verhandelt man nicht. Grundsätzlich niemals.«

»Erzähl das gerne deinen Artgenossen. Man wird dich voller Begeisterung für deine Konsequenz loben«, gab Orakel zurück.

\begin{center}
∞∞∞
\end{center}

Nüggät stand auf einem der vielen Nebenraumhafen der Hauptstadt vor einem großen Hangar. Die hier gelagerten Raumschiffe waren seit mindestens fünf Örz-Jahren nicht angetastet und deshalb zwangsweise von der Hafenverwaltung »archiviert« worden. So wurde Platz gespart, ohne die Besitzer durch eine Verschrottung ihres historischen Besitzes zu verärgern.

Auf der kleinen Folie in Nüggäts Hand stand eine schwarze »5616«. Hierbei handelte es sich gleichzeitig um den Raumschiff-Stellplatz, die aktuelle Uhrzeit und den Schmelzpunkt von Gold in Örztemp. Erstaunt über diesen unwahrscheinlichen Zufall betrat Nüggät den Hangar, wurde automatisch identifiziert und von gelben Leuchtpunkten über eine Treppe zu seinem Raumschiff geführt.

»Bitte überprüfen Sie Ihr Raumschiff auf äußerliche Beschädigungen und bestätigen Sie Ihre Entscheidung. Spätere Reklamationen können nicht entgegengenommen werden.«

Der Edelmetallhändler erkannte sein Raumschiff schon von Weitem, und das lag nicht nur an der selektiven Hangarbeleuchtung. Es hätte durch seine goldene Hülle und die unregelmäßige Nuggetform auch ohne besonderen Hinweis deutlich aus der gestapelten Menge hervorgestochen. Der Anblick des alten Weggefährten zauberte ein Lächeln auf Nüggäts Gesicht. Erinnerungen an vergangene Tage traten ans Tageslicht; alte Abenteuer bahnten sich ihren Weg zurück in sein Gedächtnis.

»Ja, bitte ausparken und auf den Haupthafen liefern«, bestätigte Nüggät mit glänzenden Augen. Er rieb sich die Hände, lief zurück nach draußen und stieg mit neuer Zuversicht in die gerade eintreffende Magnetschwebebahn.

\begin{center}
∞∞∞
\end{center}

Der Chauffeur meldete sich zu Wort. »Die nächste Ausfahrt rechts, dann noch dreißig Kilometer. Ich möchte anmerken, dass mir zunehmend unwohl wird.«

Island runzelte die Stirn. »Haben Sie etwas Falsches gegessen?«

»Nein«, entgegnete der Chauffeur mit gewisser Belustigung, »aber ich werde ungern von einem waffenstarrenden Alien-Raumschiff ins Visier genommen.«

»Keine Sorge«, beruhigte ihn der Diktator. »Das ist nur eine dumme Katze in einem Erkundungsraumschiff. Eine Kanone, keine Kompetenz.«

Free räusperte sich. »Sollen wir ihm erzählen, wie wir das Mrmbl-Robotschiff mit der Kanone zerfetzt haben?«

»Psst. Mach ihm keine Angst. Außerdem wirst du im Gegensatz zu ihm mit einem Panzer direkt in das gefährdete Gebiet eindringen, weil du das Leben deiner Mitmenschen vor mir retten möchtest.«

\begin{center}
∞∞∞
\end{center}

Äüörüzü hatte Heimweh. Die letzten Zerstörungen waren nur noch ein Ausdruck von Verzweiflung gewesen; inzwischen miaute die kleine Katze gequält, wenn sie die Kontrollen sah.

»Ich will nach Hause«, hätte sie gerne gesagt. Der Bordcomputer hätte ihr den Wunsch bestimmt sofort erfüllt. Stattdessen stand sie mit Kommunikationsschwierigkeiten in der Mitte der Zentrale. Der Flug zur Erde war für den nächsten Start vorprogrammiert gewesen, die Feuerkontrollen waren tierleicht zu bedienen, aber für den Rückflug schien es keinen festen Knopf zu geben. Immerhin wurde regelmäßig Futter von den Essensrobotern geliefert.

\ialoudspeaker{»Kommandantenvertreterin Äüörüzü, es liegt eine Nachricht für Sie vor. Sie können diese auf dem Kartentisch entgegennehmen.«}

Erfreut über die willkommene Abwechslung sprang die Katze auf den Tisch. Die elektronische Oberfläche stellte eine Tondatei als einen mit Wellenlinien beschriebenen Zettel dar. Anstelle einer Unterschrift prangte ein Foto auf dem Papier: Alexandra. Mit einer Tatze löste Äüörüzü die Wiedergabe aus.

\ialoudspeaker{»Hey Äüörüzü, was machst du für Sachen? Wir würden gerne nach Hause fliegen. Wenn du einverstanden bist, verlasse bitte das Raumschiff.«}

Doch damit war die Nachricht noch nicht beendet. yurys Stimme drang aus den Lautsprechern.

\ialoudspeaker{»Bordcomputer, Äüörüzü hat das Raumschiff gegen unseren Willen entführt. Wir würden gerne die Kontrolle zurückerlangen. Bitte sendet uns eine Bestätigung, sobald das Raumschiff befreit wurde.«}

Zwei schöne, bekannte Stimmen, die hinter den Kulissen plötzlich eine fest einprogrammierte Befehlssequenz auslösten. Äüörüzü begriff die Situation nicht ganz, aber die Bordcomputer entwickelten sofort eine beachtliche Aktivität. Zehn Sekunden später fand sich Äüörüzü in den Armen eines humanoiden Roboters wieder, der nicht nur Essen liefern, sondern auch ungebetene Gäste aus dem Raumschiff befördern konnte. Als solcher wurde die Katze bis zur Klärung der Situation betrachtet.

Hinter dem Roboter schloss sich das Außenschott. Langsam sank er mit der Katze abwärts, setzte diese auf dem Ackerboden ab und bewegte sich in Richtung des Funksignals. Ganz ohne Panzer kamen ihm die vier Freunde in der Limousine entgegen, aus der sich der Chauffeur dankend verabschiedet hatte. Am Steuer saß Alexandra; Island hatte sich zu einer Weiterfahrt auf dem Beifahrersitz überreden lassen. Die Gelassenheit der Fahrerin teilte er nicht vollständig.

»Wer weiß, was deine Nachricht im Raumschiff ausgelöst hat«, richtete er sich an yury. »Aber falls du glaubst, mich mit einer Selbstzerstörungssequenz übers Ohr hauen zu können, täuschst du dich. Dann fliegen wir in dem kleinen Flitzer ohne dich nach Örz, und du machst solange Urlaub im verstrahlten Australien.«

»Welcher Flitzer?«, fragte yury scheinheilig.

»Glaubst du, die NASA bekommt nichts davon mit, wenn ein glühendes Stück Weltraumschrott einen Waldbrand auslöst? Noch dazu, wenn es offenbar aus einem Material besteht, das hitzebeständiger als Wolfram ist.«

\begin{center}
∞∞∞
\end{center}

Die Däns Miräköl war vermutlich das einzige Raumschiff der Wirtschaftsvereinigung, dessen Oberfläche vollständig aus einer echten Goldlegierung bestand. Selbst die kleinen ausfahrbaren Laserwaffen glitzerten im gleichen Farbton und emittierten 589 Nanometer kurze Lichtwellen. Das Metall war aus vierundvierzig Teilen Gold, drei Teilen Kupfer und zwei Teilen Silber speziell für Dögöbörz Nüggät hergestellt worden – eine Spezialanfertigung, die in Kombination mit der unregelmäßigen, großen Außenfläche einen Eindruck vermeintlicher Dekadenz bei den meisten Betrachtern hervorrief. Die wenigsten Äöüzz wären finanziell dazu in der Lage gewesen, ein solches Schiff zu kaufen; niemand außer Nüggät hatte die Verrücktheit besessen, es tatsächlich zu tun.

Liebevoll strich Nüggät mit einer Hand über die Außenwand des Raumschiffs. Dieses Schiff hatte ihn mehrfach in Lebensgefahr gebracht und wieder daraus gerettet. Mit diesem goldenen Metallklumpen verband der Edelmetallhändler eine Geschichte, die sich bis in die Ursprünge des neuen Imperiums erstreckte und bis heute nicht zu seiner Zufriedenstellung abgeschlossen war. Ein Dokument wartete darauf, geborgen zu werden; das glühende Innere eines Planeten würde eines Tages zum Spielort eines dramatischen Finales werden. Eines lang ersehnten, fernen Tages. Nicht heute.

»Mister Nüggät? Sie müssten bitte einmal hier unterschreiben.« Rüzwäk und Ärkwärk ließen sich die Überraschung nicht anmerken, aber der Angesprochene lachte wissend.

»Sie haben wohl nicht damit gerechnet, mich jemals mit einem Raumschiff hier zu sehen.« Er unterschrieb mit den äußeren Fingern der rechten Hand gleichzeitig Vor- und Nachnamen.

»Noch dazu mit einem solch edlen Raumschiff«, bestätigte Ärkwärk bewundernd. »Haben Sie da das Gold der vier Örsbewohner verarbeitet?«

Nüggät schüttelte den Kopf, eine Geste, die auch auf Örz weite Verbreitung gefunden hatte. »Nein, nein. Das kam angeblich aus einem Kühlschrank und wird konsequent auch weiterhin gekühlt gelagert.« Die Polizisten lachten, aber Nüggät sprach bereits weiter. »Um Ihre eigentliche Frage zu beantworten: Das Schiff kommt aus einer Zeit, in der Sie noch auf der Polizeiakademie studiert haben und Ihr Kollege die Mittelschule besucht hat.«

»Sie sind passionierter Raumfahrer?«

»Pensioniert.« Der Händler überlegte einen Moment. »Und passioniert. Ab heute wieder. Ich kann Ihnen aber erst nachher verraten, worum es bei meiner neuen Mission gegangen sein wird.«

Ärkwärk reichte die Folie an seinen Kollegen weiter; dieser tippte darauf herum und nickte zufrieden. »Gute Reise. Und kommen Sie bitte unbeschadet wieder zurück. Ich brauche vermutlich demnächst ein Verlobungsgeschenk von Ihnen.«

»Vielen Dank und herzlichen Glückwunsch.« Dögöbörz Nüggät verbeugte sich. »Habe die Ehre.« Dann betrat er sein Raumschiff, schloss das goldene Außenschott hinter sich und schritt durch die weiß glänzenden Gänge. Die Wandverkleidung stammte aus einer Zeit, bevor der Silberrausch auf Hiddünthänätös ausgebrochen war und den Silberpreis in den Keller befördert hatte. Diese Erfahrung würde sich in deutlich größerem Ausmaß mit dem Goldpreis wiederholen, wenn niemand einschritt und Island von einer Landung auf Örz abhielt. In dieser Hinsicht betrachtete sich Nüggät als Retter seiner heimatlichen Wirtschaft. Er würde sich, das wusste er bereits, auf Paragraf 21 des Äöüzz-Strafgesetzbuches berufen.

\noindent \parbox{\textwidth}{ \vspace{3ex} \hrule \vspace{3ex}

\noindent Fäderäl\_Kriminäl\_Köd

\noindent (Örslängütränslätiön by Äzähüglü Örzgü)

\begin{itemize}
    \item[] §~21 Rechtfertigender Wirtschaftsschutz
    \begin{itemize}
        \item[] (1) Wer eine Tat begeht, die durch rechtfertigenden Wirtschaftsschutz geboten ist, handelt nicht rechtswidrig.
    \end{itemize}
    \begin{itemize}
        \item[] (2) Rechtfertigender Wirtschaftsschutz sind Maßnahmen, die erforderlich sind, um einen erheblichen Angriff auf die grundlegende Integrität der Wirtschaftsvereinigung abzuwenden.
    \end{itemize}
\end{itemize}

\vspace{3ex} \hrule \vspace{3ex} }

\begin{center}
∞∞∞
\end{center}

»Stopp!«, rief Orakel auf einmal. »Da läuft Karl am Straßenrand.«

Alexandra trat bereits auf die Bremse; sie hatte ihren Bekannten ebenfalls erkannt. »Du kannst die Essensroboter voneinander unterscheiden?«

»Klar«, behauptete Orakel, »Der hat mich schließlich fast mit einer Tiefkühlpizza verdroschen. Den erkenne ich sofort wieder.«

yury öffnete eine Seitentür. »Hey Karl, steig ein. Es sind noch ein paar Plätze frei.«

Der Roboter führte aus, was er für eine Anweisung hielt. Er hatte eine gewisse künstliche »Intelligenz«, aber dieser Begriff musste mit Bedacht verwendet werden – selbst nach Frees Software-Update.

Karl piepte ein paarmal und blinkte grün. Er wandte sich an Floating Island. »Guten Tag, ich kenne Sie nicht. Mein Name ist Karl. Ich bin ein Essensroboter vom Typ FöödBöt 40+2.«

»Hallo Karl, ich bin der Besitzer dieses Planeten. Mein Name ist Floating Island.«

»Meine Wissensdatenbank enthält keinen Besitzeintrag für Örs«, entgegnete Karl mit einer belustigend authentisch wirkenden Naivität.

»Oh, dann brauchst du wohl ein Update«, sagte Island lächelnd.

»Updates«, belehrte Karl seinen Gesprächspartner, »darf nur der Senior Master Administrator installieren.«

Alexandra, yury und Orakel lachten; Free schmunzelte. Island war sprachlos. Das lag weniger an der Befehlsverweigerung als an der verwendeten Dienstgradbezeichnung.

»Äh«, sagte Floating Island. »Woher um alles in der Welt kennst du Bright Mountain? Und wieso darf der Updates auf einem Örz-Roboter installieren?«

Die Sprachlosigkeit wechselte abrupt auf die vier Örz-Bewohner über.

\begin{center}
∞∞∞
\end{center}

Äüörüzü war verwirrt. Die Katze hatte nie gelernt, den Antigravitationsaufzug zu verwenden – sie war von Alexandra in das Raumschiff getragen worden. Wo war Alexandra überhaupt? Sie hattte sich doch gerade noch gemeldet.

Nach einigen vergeblichen Versuchen, die Raumschiffschleuse durch Klettern zu erreichen, fuhr ein ungewöhnlich langes, dunkles Auto am Getreidefeld vorbei. Es passierte den Platz mit den vielen bunten Autos, ließ den großen Kornbehälter links liegen und hielt neben dem Haus auf dem Feld an. Als sich die Türen öffneten und fünf Menschen ausstiegen, miaute Äüörüzü glücklich und lief auf die Gruppe zu.

»Da wären wir also«, stellte Free unnötigerweise fest.

»Hallo«, begrüßte yury das abenteuerlustige Haustier. Er schien von dem Wiedersehen überhaupt nicht begeistert zu sein. Aüörüzü hingegen lief glücklich maunzend um Alexandra herum. Das Abenteuer auf dem Raumschiff war lustig gewesen, aber die Örz-Katze hatte inzwischen Heimweh.

»Ja, wir fliegen gleich wieder nach Örz«, versicherte Alexandra dem Haustier, das eigentlich einem Nachbarn auf Örz gehörte. »Du wirst das zwar nicht verstehen, aber im Gegensatz zu dir würden wir diesen Flug gerne vermeiden.«

Island wurde derweil ungeduldig. »Genug gequatscht, Schluss mit dem sentimentalen Gelaber. Alle rein ins Raumschiff und Abflug in fünfzig Minuten.«

Widerwillig fügten sich die vier Freunde dem Druck der Erpressung. Während Island sich als Letzter unter den Aufzug begab und nach oben schwebte, sah Orakel durch einen der Glasgänge das rundum erschienene Militäraufgebot.

»Sie trauen uns wohl nicht ganz«, stellte er fest.

Island warf einen genervten Blick zur Seite und inspizierte den Außengang. »Ich traue euch überhaupt nicht. Beziehungsweise alles zu.« Dann zeigte er mit einem Finger auf yury. »Die nächste Demokratie ist fünftausend Lichtjahre entfernt. Benimm dich.«

yury verzog das Gesicht. »Von mir aus können Sie gerne die Entfernung mit der Demokratie tauschen.«

Island zuckte mit den Schultern und betrat die Zentralkugel. »Schön habt ihr es hier.«

Bevor yury auf der Verwendung einer Vergangenheitsform bestehen konnte, unterbrach Free das Gespräch. »Mister Island? Sie sprachen gerade von fünfzig Minuten. Das erscheint mir bei näherer Überlegung recht weit bemessen.«

Mit einem Nicken setzte Island zu einer Erklärung an, aber Alexandra verstand den Plan bereits. »Der kleine Island möchte vermutlich nicht ohne seine Lieblings-Kuschel-Goldbarren seine Heimat verlassen.«

Free drehte sich ruckartig zu ihr um. »Nein, das ist nicht wahr.« Das Ausmaß der sich anbahnenden Katastrophe wurde ihm langsam bewusst.

»Doch, doch«, versicherte Island. »Mit weniger als zehntausend Tonnen Gold gehe ich nicht aus dem Haus.«

yury brach innerlich zusammen. Wie anmaßend war es, die Menschheit für wichtiger zu halten als die Zukunft eines gesamten Sternenreiches? Island würde mit den Äöüzz sicherlich nicht sanfter umgehen als mit den irdischen Sklaven. Vor seinem inneren Auge tobte bereits ein sinnloser Machtstreit, ein Raumschiffkrieg gegen die uggy und den Mrmbl-Orden mit unzähligen vermeidbaren Opfern.

»Kennt ihr das«, sagte Orakel dann, »wenn in den Nachrichten steht, unter tausend Betroffenen waren zwei Deutsche?«

Das war zu kurz gedacht, überlegte yury. Auch die vorerst verschonte Menschheit würde eines Tages um Gnade betteln müssen; die Erde war langfristig in Gefahr.

Umgeben von unangenehmer Stille forderte Free schließlich das Einschalten des Belademodus an. Anschließend murmelte er verzweifelt vor sich hin. »Das wird teuer, das wird teuer. So viel Wasserstoff haben wir überhaupt nicht.«

Kolonnen von Goldtransportern, eskortiert von Panzern, entluden ihren Inhalt in die Lagerhallen des Erkundungsraumschiffs. Die Dichte des Metalls ermöglichte eine Überfrachtung, die kein Inspekteur auf Örz akzeptiert hätte.

»Wir haben jetzt ungefähr die Hälfte Ihres angeblichen Mindestziels erreicht«, wagte Free dann einen vorsichtigen Protest. »Uns brechen gleich die Landestützen durch.«

»Schade drum«, entgegnete Island. »Weitermachen.«


\chapter{Abflug}

Wie er es geschafft hatte, die Flammen des Raketenantriebs gelb-orange zu färben, blieb Dögöbörz Nüggäts Geheimnis. Die faszinierten Blicke der Schaulustigen hinter sich lassend, schoss das Goldnugget in die Stratosphäre.

Nüggät saß begeistert hinter den Kontrollen. Er hatte es sich nicht nehmen lassen, die manuelle Steuerung zu aktivieren und einen erhöhten Treibstoffverbrauch zu verursachen. »Jippie!«, rief er voller Freude und fühlte sich wieder jung.

»Gute Reise«, tönte es aus dem Funkempfang.

»Dankeschön. Grüß deine Frau von mir.«

»Mach ich, sobald sie zurückkommt«, versicherte Älföns Ögnöwäk. »Manchmal beneide ich euch Raumfahrer.«

Der Edelmetallhändler lachte. »Da kann ich aber schlecht mit dem Kriegsschiff mithalten.«

»Wenigstens gehört dir dein Schiff selbst«, gab Ögnöwäk zu bedenken. »Die Kommandanten der Vängefül Destrüktiön sind theoretisch austauschbar.«

»Das wäre ein großer Fehler«, befand Nüggät. »Das Gnörk-Kartell lässt sich nur mit einem eingespielten Team bekämpfen.«

»Verdammte Terroristen«, fluchte es von der Bodenstation. »Leben im Paradies und finden trotzdem Gründe, kriminell zu werden.«

\iathought{Wenn du ahntest, was ich seit Jahren vorhabe, würdest du dir Sorgen um mich machen}, sinnierte der Raumfahrer. »Du sprichst mir aus der Seele. Auf Wiedersehen.« \iathought{Vielleicht.}

»Auf Wiedersehen.«

\begin{center}
∞∞∞
\end{center}

»Zehntausend Tonnen und ein Kilogramm. Das ist kein Vielfaches des üblichen Barrengewichts«, bemerkte yury.

»Richtig«, äußerte Island sich beeindruckt. »Ich habe tatsächlich einen kleineren Barren mitgenommen, um dieses Gewicht zu erreichen. Eine Spezialanfertigung. Aus symbolischen Gründen, verstehst du?«

»Nee, verstehe ich nicht.«

»Ein Kilogramm Trinkgeld für euch.«

»Wie großzügig. So arm sind wir nun auch wieder nicht…«

Mit gespielter Empörung stemmte Island die Hände in die Hüften. »Du müsstest dich mal reden hören. In deiner Undankbarkeit verschmähst du mehrere zehntausend US-Dollar.«

»…dass wir Ihr dreckiges gestohlenes Gold – ach so, darauf wollen Sie hinaus. Eine Doppelmoral? Nein, das sehe ich nicht so. Wir haben die Goldreserven unseres Heimatplaneten nicht nennenswert angetastet. Außerdem haben wir uns dafür gehörig angestrengt, erhebliche Risiken auf uns genommen und niemanden ernsthaft verletzt. Ihnen fehlt jegliche moralische Rechtfertigung für eine Erpressung mit nuklearen Sprengköpfen und den geplanten Angriff auf die friedlich lebende Bevölkerung tausender Planeten.«

Island lächelte. »Ich finde, mein Coup ist nur eine größere Variante eures Ausflugs.«

Free hielt das Gespräch nur für schwer zu ertragen und raufte sich in Anbetracht einer Kontrollanzeige die Haare. »Das eine Kilogramm ist dem Antrieb relativ egal. Wir werden das Gewicht sogar noch weiter erhöhen müssen.«

Durch diesen Einwand zog er einige verwunderte Blicke auf sich.

»Wenn wir das Schiff nicht bis an den Rand mit Deuterium volltanken, stürzen wir nach dem Start durch Treibstoffmangel ab. Die Landestützen sind aber längst überlastet und laut Datenblatt vor fünf Minuten zusammengebrochen.«

Der Diktator lachte respektlos. »Für einen letzten Flug wird es wohl noch reichen. Für genug Deuterium ist jedenfalls gesorgt.« Einhundert Kryotanklastwagen näherten sich dem Raumschiff. »Bedient euch.«

Die Landestützen überstanden auch diese Belastung zumindest ohne spürbaren Totalschaden. Ob sie sich nach der Ankunft auf Örz noch verwenden lassen würden, stand in den Sternen.

Free gab yury einen Wink; dieser schloss daraufhin alle Außenschotten und kratzte sich am Kopf. »Wir werden doch sicherlich Zwischenlandungen benötigen, um mit diesem Gewicht das Ziel zu erreichen.«

Alexandra blickte Orakel über die Schulter, der gerade am Kartentisch einige Einstellungen vornahm. Sie zählte, überprüfte ihr Ergebnis ungläubig zweimal und blickte yury an. »Ja, fünfundzwanzig Stück bei einer Reisedauer von einem Erdmonat.«

Orakel schloss die Routenplanung ab und stellte sich eine Zwischenlandung bildlich vor. »Wir werden auf dem nächsten Wasserplaneten schlichtweg untergehen.«

»Das ist kein Problem«, riet Island. »Der Antigravitationsantrieb funktioniert bestimmt auch unter Wasser, und ihr könnt mir nicht erzählen, das Schiff sei nicht wasserdicht.«

»Wollen Sie denn Wasser tanken?«, entgegnete yury. »Oder doch möglicherweise neuen Wasserstoff? Wissen Sie, wie wir den erzeugen? Mit Spiegeln und einem Sonnenwärmekraftwerk.«

Island starrte fünf Sekunden lang vor sich hin, blickte nach draußen und stieß einen Fluch aus.

»Ich mache Ihnen ein Angebot. Wir schmeißen drei Viertel des Gewichts von Bord, dann können wir uns mit Luftkissen über Wasser halten. Im Gegenzug dürfen Sie an Bord bleiben, obwohl Sie als ungebetener Passagier mit Abstand der überflüssigste Ballast sind.«

Nach Luft schnappend, hieb Island eine Faust auf den Computertisch. »Du bist nicht in einer Position, um Angebote zu machen. Wir machen das so, wie ich es sage.«

Die vier Freunde sahen ihn gespannt an. In seinem Kopf schien es zu arbeiten.

»Und zwar«, beschloss er nach einer halben Minute, »werden wir siebeneinhalbtausend Tonnen Gold wieder entladen.« Er beorderte die Transporter zurück; etwa eine halbe Stunde lang entluden verwirrte Soldaten das kurz zuvor eingeladene Gold. Dann gab Island endgültig den Befehl zum Abflug.

Die 4-6692 erhob sich ächzend vom amerikanischen Erdkontinent. Lautes Dröhnen begleitete den Start. Orakel stand an seinem Lieblingsort im nördlichen Glasgang und winkte den Zuschauern am Boden deprimiert zu, bis sie seine Sichtweite verlassen hatten.

\begin{center}
∞∞∞
\end{center}

Nüggät blickte auf das halbkugelförmige Display seines Hauptcomputers. Das Update war kurz nach seinem letzten Abenteuer installiert worden. An diesen neumodischen Kram würde er sich noch gewöhnen müssen. Später. Er drückte fünf Knöpfe nacheinander, erhob sich von seinem Sessel und lief zum Ausgabeschacht des Foliendruckers.

Dieser hatte jedoch nur eine Fehlermeldung für Nüggät anzubieten. \ialoudspeaker{»Die Folienschächte 1 und 2 sind leer. Bitte füllen Sie Plastik nach.«}

»Dann nimm halt den dritten Schacht«, befahl der Kommandant der Maschine.

\ialoudspeaker{»Folienschacht 3 ist mit ungeeignetem Druckmaterial gefüllt.«}

Solch ein Unsinn! Nüggät riss die große Seitentür des Druckers auf, kniete sich auf den Boden und zog den Behälter aus dem Gerät. Zu seiner Überraschung befand sich darin tatsächlich äußerst ungeeignetes Material für einen Druckvorgang. »Oooh.«

\begin{center}
∞∞∞
\end{center}

Es vergingen einige Tage auf dem entführten Raumschiff. Eine Zwischenlandung auf einem Wasserplaneten wurde ohne nennswerte Schwierigkeiten absolviert, wobei die 4-6692 trotz ausgefahrener Luftkissen tief in die heiße Treibstoffquelle einsank.

\iathought{Ich würde gerne über das Deck spazieren}, hatte Island sich überlegt. Mit Blick auf das Außenthermometer bat er yury um eine Erklärung, wie sich die Örztemp-Zahl in Celsius umrechnen ließe. Der begnügte sich jedoch nicht mit einer einfachen Umwandlung der aktuellen Zahl, sondern hielt einen kleinen Vortrag über die mathematischen und physikalischen Besonderheiten des Einheitensystems von NGC 6193.

»Örztemp ist die absolute Temperaturskala der Äöüzz. Diese Wesen haben, wie Sie möglicherweise wissen, sieben Finger an jeder Hand und ein Faible für Mathematik. Bei der Konstruktion der interstellaren Einheiten wurde, daher wenig überraschend, Wert auf mathematische Eleganz und runde Ergebnisse im Siebenersystem gelegt. Ein Örztemp ist die thermodynamische Temperatur des Tripelpunktes des Wassers geteilt durch sieben hoch drei, also 343. Daher kann zur Umrechnung der Skalen zunächst der Kelvin-Wert durch Multiplikation mit 273,16 und Teilung durch 343 ermittelt werden – näherungsweise ein Faktor von vier Fünfteln. Das gewünschte Ergebnis in Grad Celsius entsteht anschließend ganz einfach durch die bekannte Subtraktion von 273,15 vom Kelvin-Wert, oder grob 270. Leider wirken beide Rundungsfehler in dieselbe Richtung. Der absolute Anteil verzerrt die Umrechnung kalter Messergebnisse, der relative Anteil lässt hohe Temperaturen ungenau erscheinen. Wenn Sie auf der sicheren Seite sein möchten, müssen Sie mit Kommazahlen arbeiten.«

Island nickte. »Selbstverständlich.« Zwei Stunden später kam er erneut auf yury zu. »Wie viel sind denn beispielsweise vierhundert Örztemp in Celsius?«

»Ach, die Außentemperatur«, bemerkte yury sofort. »Ich nehme an, Sie spielen mit dem Gedanken, bei fünfundvierzig Grad Lufttemperatur ein Sonnenbad zu nehmen. Von meiner Seite aus spricht überhaupt nichts dagegen.«

Desillusioniert verzog der Fragesteller das Gesicht und zog sich wieder in die Zentrale zurück.

\begin{center}
∞∞∞
\end{center}

Nach einigem Überlegen entschloss sich Nüggät dazu, den Inhalt des Folienschachts unangetastet in seinem Versteck zu belassen. »Das fängt ja gut an. Computer, bitte berechne eine sinnvolle Route nach Dönkwön II.«

Die Berechnung nahm einige Sekunden in Anspruch. \ialoudspeaker{»Es wurden zwei Vorschläge berechnet. Vorschlag eins: Wir machen einen Zwischenhalt im Küttröt-System und –«}

»Vorschlag zwei.«

\ialoudspeaker{»Aktion umgesetzt, Route gespeichert. Es wurden drei Landungen eingeplant; unser nächstes Ziel ist die Discount-Tankstelle auf HörriblDisastör IV. Bitte lehnen Sie sich zurück und genießen Sie Ihre Reise.«}

Der Edelmetallhändler schmunzelte. Ohne ihr Gold hätten die vier Weltraumtouristen vermutlich einige verrückte Abenteuer auf dem Schrottplaneten erlebt. Ironischerweise war selbst dort der Lebensstandard höher als in ihrer technologisch minderbemittelten Heimat.

Das goldene Raumschiff trat sanft in die Atmosphäre ein, überflog einige Armenviertel und erhielt über Funk eine Landegenehmigung direkt neben einer der Hochdruck-Zapfsäulen für Wasserstoff. Die Tankstelle war seit Jahren nicht gewartet worden; Schmelzschäden waren über die gesamte Landefläche verteilt. \iathought{Manche Piloten,} dachte Nüggät, \iathought{haben ihre Fluggenehmigung beim Glücksspiel gewonnen.} Es war ihm unbegreiflich, wie manche Besucher ihre Verachtung für die Unterschicht durch Respektlosigkeiten zum Ausdruck brachten. Durch den bodennahen Einsatz der Raketentriebwerke traten diese Personen ihre Gastgeber mit den Füßen.

Ein wenig an die eigene Nase fassen, bemerkte Dögöbörz Nüggät, würde er sich ebenfalls müssen. Mit einem goldüberzogenen Raumschiff im Ghetto zu landen, war eine ungewollte, aber vorhersehbare Provokation der Zuschauer am Boden. Im Grunde genommen war es nicht seine eigene Idee gewesen, aber die zu dieser Computerentscheidung führenden Parameter hatte er vor vielen Jahren selbst festgelegt. Vielleicht hätte er sich einen dritten Vorschlag berechnen lassen sollen.

Da Roboterarbeit günstiger war als die Anstellung eines Äöüzz-Tankwarts, fand praktisch überall im Imperium die Betankung durch spezialisierte Maschinen statt. Anders sah es auf HörriblDisästör IV aus: Hier war Selbstbedienung erforderlich. Nüggät griff nach dem dünnsten Ladeschlauch, verband das Ende mit der Tanköffnung der Däns Miräköl und forderte eine vollständige Betankung an.

\begin{center}
∞∞∞
\end{center}

Orakel saß im Schneidersitz auf dem Boden des Nordgangs. Neben ihm, vor ihm, rund um ihn herum lagen über zwanzig große Papierbögen; unter seinen Händen befand sich eine saphirblauer Aktenordner. yury hätte sich an seiner Stelle vermutlich über die »nicht hinnehmbare« Schmählerung des Weltall-Rundumblicks durch die Präsenz und Sichtbarkeit des Erpressers im Ostgang beschwert. Zudem hätte yury die Gelegenheit für einen makaberen Wortwitz über das unbefugte Eindringen des US-Amerikaners in den Osten nicht ungenutzt verstreichen lassen. Orakel tat nichts dergleichen. Er ließ sich auch nicht stören, als Island den Nordgang betrat und ihm interessiert bei seiner Arbeit zusah.

Klassisch mit Bleistift auf Papier entstanden in beeindruckender Detailverliebtheit wunderschöne Weltraumbilder. Nach minutenlanger Beobachtung räusperte sich der Besucher und las den Titel des Aktenordners vor. »\iaquote{›Impressionen zwischen Örz und Örs.‹} Bewundernswert.«

Orakel lächelte, ohne den Stift abzusetzen. »Es hat mich einige Überzeugungsarbeit gekostet, den Stift mit an Bord nehmen zu dürfen.«

Das verstand selbst Island ohne große Raumerfahrung. »Wegen des Graphits wahrscheinlich. Ein schwarzer Kugelschreiber hätte es doch auch getan?«

»Notfalls, ja. Dank der künstlichen Schwerkraft darf ich aber auch mit Graphit schreiben. Ein kleiner fleißiger Roboter sammelt nachher den Staub auf. Außerdem ist die Raumschiffelektronik luftdicht verpackt, wenn sie nicht gerade gewartet wird. Die Kühlung erfolgt über Wärmetauscher.«

»Ah, ich verstehe. Und mit den Zeichnungen lässt sich bestimmt eine Menge Geld bei den Äöüzz verdienen«, versuchte Island die Motivation nachzuvollziehen.

Orakel legte den Kopf abwechselnd schräg nach links und rechts. »Darum geht es mir nicht. Wenn die Mappe fertig ist, verbreite ich sie auf allen bekannten bewohnten Planeten zum Selbstkostenpreis. Unter einer freien Lizenz.«

Langsam setzte ein gewisses Verständnis ein. »Ah, also für diejenigen, die das an ihre Freunde und Familie weitergeben möchten.«

»Nicht nur für die«, ergänzte Orakel. »Für jede Person, für jeden beliebigen Zweck.«

»Auch für kommerzielle Großkonzerne? Ohne Einschränkung?« Das konnte Island kaum glauben.

»Naja, die Bedingung ist, dass mein Name als Zeichner genannt wird, zusammen mit einem Link auf meine Örznetseite.«

»Aha!«, rief der ehemalige Agent aus. »Du möchtest die Werbetrommel für deinen Online-Merchandising-Shop rühren. So erhältst du dann dein gewünschtes Geld.«

Mit ehrlichem Erstaunen legte Orakel den Stift zur Seite und blickte zu Island nach oben. »Nein. Dadurch stelle ich nur sicher, dass selbst kommerzielle Verkäufer nicht auf einen Hinweis auf den Ort verzichten können, an dem das Produkt kostenlos in elektronischer Form erhältlich ist.«

Island kratzte sich am Kinn und blickte nachdenklich in Flugrichtung nach draußen. »In der erzwungenen Namensnennung spiegelt sich aber ein verstecktes Motiv wieder. Ein Wunsch nach persönlicher Berühmtheit.«

Auch Orakel zählte gedankenversunken die Sterne. »Auf diese Interpretation würde ich mich tatsächlich einlassen – zumindest für den Fall, dass das Werk unerwarteterweise große Verbreitung erfährt. Aber wem sage ich das?«

»Oh«, nahm der Diktator die Anspielung auf. »Ich glaube, mein Antrieb ist nicht das allseits präsente Streben nach Berühmtheit.«

Diesmal konnte Orakel bei allem Respekt ein spöttisches Herauslachen nicht unterdrücken. »Ihr Antrieb ist pure Philantrophie.«

Auch der eindeutige Nicht-Philantroph lachte. »Das allerdings nicht.« Er atmete genießerisch tief ein, als könne er den Geruch des im Vakuum schwebenden Sternenstaubs durch die Glasscheibe hindurch riechen. Dann folgte die angebliche Erklärung. »Mein Antrieb ist ein unstillbarer Durst nach Wissen und nach uneingeschränkter, verzögerungsfrei ausübbarer Macht.«

»Impliziert das nicht große Berühmtheit? Ist die Macht, und jeder Versuch, sie zu vergrößern, nicht nur ein Mittel zu diesem eigentlichen Zweck?«, wagte Orakel, nachzuhaken.

Island nahm ihm diese Fragen nicht im Geringsten übel. Mit Orakel konnte man vernünftig reden. »Nein, das würde ich nicht so interpretieren. Mein Traum wäre nämlich auch dann erfüllt, wenn ich als gottgleiches, aber namenloses Wesen die volle, direkte Kontrolle über das gesamte Geschehen der Milchstraße hätte. Allwissenheit und anonyme Allmacht würden mir bereits genügen.«

»Das klingt sehr bescheiden«, sagte Orakel, behielt die Ironie der Aussage aber für sich. Er griff wieder nach dem Bleistift und zeichnete einen Stern ein, der ihm auffiel – einen Stern, den er glaubte, aus der Ferne wiederzuerkennen. Stirnrunzelnd zog er einige Striche auf dem Papier und wischte vorsichtig mit seinem linken Daumen über das Graphit. In Gedanken versunken entspannte er seine Augen, das Bild verschwamm und fühlte sich an wie ein Déjà-vu.

\begin{center}
∞∞∞
\end{center}

Der Kommandant traute seinen Augen nicht. Dort kam tatsächlich ein Tankstellenmitarbeiter angelaufen und winkte bereits im Laufen wild mit den Armen. Mit der Befürchtung, der Wärter wolle ihn wohl darauf aufmerksam machen, dass die Ladesäule demnächst explodiere, drückte Dögöbörz Nüggät den roten Stoppknopf. Es war ein mulmiges Gefühl, auf mehreren Kubikkilometern explosiv brennbaren Materials zu stehen, ohne sich auf die Sicherheit der Behälterwände verlassen zu können.

»Herr Nüggät? Sind Sie das? Ist das Ihr Raumschiff?«, rief der Planetenbewohner aus der Ferne. Er erhielt ein verwirrtes Nicken als Antwort. »Gut, dass ich Sie treffe. Erinnern Sie sich an mich?«

»Ehrlich gesagt, nein. Dabei habe ich eigentlich ein sehr gutes Kundengedächtnis.«

»Vielleicht hilft es Ihrem Gedächtnis auf die Sprünge, wenn ich Sie an Tisiphöne erinnere.«

Nüggät riss die Augen auf. Der vermeintliche Fremde lächelte zufrieden.

»Das Äöüzz-Militär schuldet uns noch sechzigtausend Äzz zuzüglich Zinsen.«

»So funktioniert das nicht«, protestierte Nüggät reflexmäßig. »Sie können nicht im Nachhinein einen astronomischen Preis für eine Tankladung festlegen. Außerdem können Sie Ihren Teil des Vertrags überhaupt nicht erfüllen. Die Vängefül Destrüktiön fliegt hunderte Lichtjahre entfernt durch das All.«

»Das hat sich mit Ihrer Landung geändert«, erinnerte ihn der Tankwärter mit zusammengekniffenen Augen. »Ich zahle meine Schulden, und Sie zahlen Ihre.«

So viel Ignoranz war erschreckend. »Glauben Sie, ich bin hierher gekommen, um eine Zwanzigstelmillion Äzz für eine Tankfüllung zu bezahlen?«

»Nein, aber genau das werden Sie trotzdem tun.«

\begin{center}
∞∞∞
\end{center}

Orakel fasste einen Entschluss, streckte seine Beine aus, massierte seine Fußknöchel und erhob sich von seinem Platz. Den Aktenordner und den Bleistift hielt er in jeweils einer Hand, als er sich zu Island drehte. Der starrte jedoch vollkommen geistesabwesend in die Schwärze des Weltalls. »Entschuldigung?«

Island schüttelte die Gedanken ab und drehte sich zu Orakel um.

»Ich wollte Ihren Gedankengang nicht stören. Mich wundert aber, wie Ihre langfristigen Ziele lauten. Herrschaft über die Galaxis ist zwar kein durch Zurückhaltung auffallender Wunsch, aber auch kein allumfassender Abschluss.«

»Stimmt. Sobald mir die Milchstraße gehört, folgt die Entwicklung intergalaktischer Antriebe und die Eroberung der umliegenden Galaxien. Gerne durch Eingliederung, aber nötigenfalls mit Gewalt.«

Das Gesicht verziehend, verabschiedete sich Orakel. »Ich glaube, in diesem Punkt werden wir uns nicht einig.«

\begin{center}
	∞∞∞
\end{center}

Dögöbörz Nüggät hatte sich in seinem Raumschiff verschanzt und rief über Warpfunk nach Verstärkung. Draußen stand der nicht ganz zu Unrecht verärgerte Gläubiger und kratzte mit seinen Fingernägeln an der Goldhülle.

\iathought{Solange er nicht seine Zähne benutzt, verursacht er keinen Kratzer.} »Hallo, Raumkontrolle? Wir haben einen etwas komplizierten Schuldenfall auf der Bütän-Tankstelle auf HörriblDisästör IV. Könnten Sie ein paar Imperiumspolizisten vorbeischicken? Dankeschön.«

Mit dem Wissen, bald Verstärkung aus dem All zu erhalten, verließ der reiche Pilot das Raumschiff und stellte sich erneut den Hasstiraden seines Gegenübers. Genervt ertrug er einige heftige Beleidigungen, bis endlich ein Patrouillenschiff in Sichtweite geriet. Die hell grün-blau blinkende Kugel war am Himmel kaum zu übersehen; die ganze Umgebung wurde in buntes Blinklicht getaucht. Mit ungutem Gefühl dachte Nüggät an den Inhalt des dritten Druckerfachs, das aber hoffentlich von Untersuchungen verschont bleiben würde. Schließlich ging es bei diesem Vorfall nicht um sein Schiff, sondern um ein mehrere Jahre altes Tankabonnement mit äußerst fragwürdiger Kündigungsfrist.

\begin{center}
∞∞∞
\end{center}

Während Dögöbörz Nüggät vor den Augen hunderter Schaulustiger redegewandt seinen Kopf aus der Schlinge zog, flog die 4-6692 von Wasserwelt zu Wasserwelt, unaufhaltsam einem Ziel entgegen, das nur ein einziger Passagier tatsächlich besuchen wollte.

Alexandra räusperte sich; ihr Gesichtsausdruck wirkte auffallend harmlos. Bei yury, Orakel und Free schrillten sofort einige Alarmglocken; mit der gleichen Mimik hatte sie zuletzt vor der nächtlichen Sprengung eines Hochhauses ihre Arbeit begutachtet. »Island? Nun, da wir von der Erde bereits über hundert Lichtjahre entfernt sind, möchte ich eine indiskrete Frage wagen.«

Nichts davon ahnend, gab ihr der Diktator durch ein Nicken seine Neugier zu verstehen. Alexandra, die dem Diktator seit der Begegnung im Thronsaal konsequente Respektlosigkeit entgegenbrachte, kam sofort zum Punkt.

»Wie lautet deine Lebensversicherung?«

Unausgesprochen schwang in dieser Frage eine Drohung mit. Ohne zufriedenstellende Erklärung seines Plans würde er in Lebensgefahr schweben; das Blatt schien sich ruckartig zu wenden. Island hatte jedoch vorgesorgt.

»Orakel hat mir damals E-Mails aus dem Weltall geschickt. Falls ihr euch noch erinnert, konnte ich überhaupt nur auf diese Weise die Erde vor einer Zerstörung durch den Alien-Angriff bewahren.«

Alexandra drehte unbeeindruckt ihren Zeigefinger im Kreis. \iathought{Erzähl uns mehr, wenn dir deine Freiheit etwas bedeutet und du sie behalten möchtest.}

Island blieb gelassen. Mit einem ähnlichen Gespräch hatte er schon vor dem Abflug gerechnet; nützen würde es nur ihm selbst. »Wenn ich nicht bis zu einem bestimmten Datum eine E-Mail mit einem Passwort an mehrere Kollegen versende, werden drüben auf der Erde die Szenarien zur Realität, die sich die Menschheit im kalten Krieg ausgemalt hat. Nuklearer Frieden funktioniert nur dann, wenn alle Verantwortlichen potenziell selbst von ihren Aktionen betroffen sind. Ich habe Vorkehrungen getroffen, um diesen Grund für eine Befehlsverweigerung unmöglich zu machen: Der Computer, den ihr Schlaumeier unbehelligt gelassen habt, kümmert sich herzlich wenig um Menschenleben. Außerdem bleiben ausgewählte Bereiche der Erde von der Katastrophe verschont – dank Äöüzz-Technologie sogar während eines nuklearen Winters.«

Äüörüzü sprang in Alexandras Arme und schnurrte zufrieden. Nach und nach löste sich die Gruppe auf; ihrer Hilflosigkeit bewusst, verteilten sich die Freunde im Raumschiff und gingen ihren Hobbys nach.

\begin{center}
∞∞∞
\end{center}

Nüggät machte einen großen Sprung aus der Luftschleuse, flog ein Stück weit über den gelben Hafenboden hinweg und landete mit beiden Füßen auf der schwarzen Begrenzung des benachbarten Landeplatzes. Eine wachsähnliche Substanz mit erstaunlicher Hitzeverträglichkeit überzog das sechseckige Feld.

Der inzwischen wieder enthusiastisch raumfahrende Abenteurer befand sich auf Dönkwön II, einem mäßig besiedelten Planeten am Rande der Äöüzz-Wirtschaftsvereinigung. Ameisenähnliche Insektenwesen bildeten die Bevölkerungsmehrheit in diesem Raumsektor; das Dönkwön-Planetensystem war von fleißiger Agrarwirtschaft geprägt. Mit ihrer Körperlänge von durchschnittlich einem halben Meter und ihren ikosaederförmigen organischen Transportraumschiffen handelte es sich bei den Ameisen um eine der zivilen Hauptmächte neben den Großunternehmen von Örz. Einen Nebenerwerbszweig stellte der Tourismus dar, wobei die Mentalität, die lokalen Gesetze und gesellschaftlichen Bräuche der krabbelnden Planetenbewohner auch nach Jahrtausenden der wirtschaftlichen Zusammenarbeit noch immer ein gewisses Konfliktpotenzial boten.

Mehr als alles Gold der Welt liebten die Dönkwöner ihren Beruf. Individualismus war dem Insektenvolk fremd; jeder half jedem, alle arbeiteten gemeinsam an planetenweiten Zielen. Die Möglichkeit, durch den so errungenen Reichtum auf Arbeit verzichten zu können, erschien den Ameisen so absurd, dass sie den gesamten Produktionsüberschuss und alle Gewinne in den umliegenden Planetensystemen verschenkten. um ungestört weiter produzieren zu können. Als Tourist war Nüggät zwar jederzeit herzlich willkommen, aber die Bezahlung für alle Dienstleistungen und Nahrungsmittel musste in Form von Agrarmaschinen, mechanischen Ersatzteilen oder Fabriktechnik erfolgen. Tauschangebote wurden im Schwarm besprochen; anstelle eines Einzelhändlers verhandelte Nüggät mit der gesamten Planetenbevölkerung. Der kleine schwarze Kasten an seinem Raumanzug übersetzte die Insektenlaute von und nach Örzlängü.

Neugierig umringten tausende Arbeiter das goldene Raumschiff, aus dem Nüggät seine Mitbringsel heraustrug. Synchron klackte und zirpte es aus der Menge. »Zwölf Mähdrescherhaspeln aus massivem Titan? Sie scheinen ein besonders wichtiges Anliegen zu haben.«

»In der Tat«, bestätigte der Besucher. »Ich würe gerne sechs Planetenrotationen lang in der Geisterstadt Krönöhr Mäk verbringen.«

Diese Ankündigung löste ein scheinbar wildes Getummel in der Insektenmasse aus. Nach einigen Minuten beruhigten sich die Bewegungen wieder. »Wir sind einverstanden.«

Nüggät blickte erstaunt in die Facettenaugen seiner Zuhörer. »Es wären aber noch einige Kleinigkeiten zu besprechen. Meine Verpflegung beispielsweise.«

»Wir sind einverstanden«, ertönte es ohne Zögern. »Es ist alles bereits geplant und steht in Krönöhr Mäk zur Verfügung. Sie dürfen sich gemeinsam mit bis zu zweihundertsechzehn Wesen Ihrer Wahl für den genannten Zeitraum in der Geisterstadt aufhalten. Für Ihre Spezies als Delikatessen bekannte Speisen sowie Grundnahrungsmittel aller Art werden in ausreichender Menge vorhanden sein.«

»Eine Woche Vollkorn und Alginatkugeln«, äußerte Nüggät seine Begeisterung, deren leicht ironischer Unterton bei der Übersetzung verlorenging. »Wir sind im Geschäft. Vielen Dank.«

\begin{center}
∞∞∞
\end{center}

Das hexagonale gelbe Ortseingangsschild trug den Schriftzug  »Krnhr Mk« in großen schwarzen Lettern.

\iathought{Eigentlich,} überlegte Dögöbörz Nüggät, \iathought{habe ich gar keine Zeit für Urlaub.} Er strich mit einigen Fingern über das Schild und kratzte vorsichtig an der wachsähnlichen Substanz. Als er unter seinen Fingernägeln den Abrieb spürte, zog er schnell die Hand zurück und blickte sich verstohlen um. \iathought{Vielleicht sollte ich nicht alles auf eine Karte setzen.}

Alle Gebäude in Krönöhr Mäk waren aus gelbem Bienenwachs gebaut und bestanden aus mehreren gestapelten prismenförmigen Waben mit sechseckiger Vorder- und Rückseite. Die eigentlich in Höhlen lebenden Dönkwöner hatten die Nutzung natürlicher Ressourcen perfektioniert und ganze Städte aus organischen Materialien errichtet. Die meisten Touristen verschmähten jedoch dieses Angebot, da sie die Übernachtung in Höhlenhotels für ein authentischeres Besuchserlebnis hielten. Geisterstädte aus Wachs überzogen den Planeten; Krönöhr Mäk war ein abgelegener Ort, an den sich sicherlich kein Lebewesen zufällig verirren würde.

Nüggät lachte vor sich hin. »Zweihundertsechzehn Wesen ihrer Wahl«, die Ameisen waren großartige Gastgeber.


\chapter{Zwei blinde Hühner}

»Dieser Vollpfosten«, spottete Schreiner. »Hat vergessen, mir das Dynamit abzunehmen.«

Wolfgang pflichtete ihm bei. »Und er hat mein Smartphone nicht entdeckt. Typische Amateurfehler. Der Sprengstofffrau oder dem Mathenerd wäre das nicht passiert. Vermutlich nicht einmal dem Computerfreak. Wir haben nur ein Problem.«

»Welches da wäre?«

»Die Druckwelle wird sehr, sehr unangenehm«, prophezeite der Kopf des kriminellen Duos.

Zwei Minuten später hatten sich die beiden Eingesperrten in eine Toilettenkabine zurückgezogen und hielten sich in unangenehmer Erwartung die Ohren zu.

»Nimm die Schulter für die Türseite, so wie ich«, riet Wolfgang.

»Was?«

»Ach, vergiss es.«

Die Explosion des überdosierten Sprengstoffs unterbrach das Gespräch. Vor Marcor Schreiners Augen verschwamm die Umgebung; Wolfgang hatte im Nachhinein den Eindruck, kurzzeitig das Bewusstsein verloren zu haben. Höllische Ohrenschmerzen und ein äußerst unangenehmes Pfeifen begleiteten die erfolgreiche Befreiungsaktion.

Noch immer die zweite Stange Dynamit zwischen den Zähnen tragend, nuschelte Schreiner vor sich hin. »Himmel, meine Ohren klingeln wie verrückt. Oh nein, meine Nase. Meine verdammten Zähne. Mein Kopf.«

»Du kannst die Finger jetzt aus den Ohren nehmen«, erinnerte ihn sein Kollege.

»Was?«

\begin{center}
∞∞∞
\end{center}

Vor einem hell erleuchteten, geöffneten Aufzug schien die Suche eine unerwartete Wendung zu nehmen.

»He, warte«, zischte Wolfgang. »Das stinkt, da vorne. Da hat jemand goldene Schlüssel stecken lassen.«

»Das Gebäude ist leer«, meckerte Marcor Schreiner. »Die Schlüssel sind bestimmt nur aus Messing; lass uns verschwinden. Wer weiß, ob wir da oben einen Feueralarm ausgelöst haben.«

Wolfgang lachte. »Das hätten wir längst bemerkt. Dann wären wir klitschnass.«

Zögerlich verwarf Schreiner seine Angst. »Also gut. Was sollen wir machen?«

Anstelle einer Antwort betrat der Befragte die Kabine. Er drehte die Schlüssel entgegen des Uhrzeigersinns, woraufhin die Beleuchtung erlosch und die Kabinentüren sich schlossen. Eilig sprang Schreiner hindurch und blieb dann ratlos im Dunkeln stehen. Wolfgang drehte die Schlüssel zurück; die Beleuchtung kehrte zurück. Um die verschlossene Tür wieder zu öffnen, drückte Schreiner den entsprechenden Knopf; Wolfgang widersprach wortlos durch Druck auf den »Türen schließen«-Knopf.

Angestrengt und mit Blick für kleinste Details begutachtete der Kopf des Duos die Nummerntafel. Er griff mit seinen Fingernägeln in die Knopfrillen, zog ein Taschenmesser hervor, wählte einen Schraubendreher und machte sich damit an mehreren Schrauben zu schaffen. »Guck mal«, sagte Wolfgang. »Das ist wirklich clever gestaltet.«

»Wovon redest du?«, fragte Schreiner mit einerseits großem Interesse, aber andererseits vollkommen fehlendem Sachverständnis.

Die Schraube fiel zu Boden. »Hab mich geirrt«, gab der Wichtigtuer zu.

Nun wollte aber auch Marcor Schreiner einen Teil zur Arbeit beitragen und drückte wahllos irgendeinen Knopf. Dass dieser ausgerechnet mit einer gelben Glocke bedruckt war, brachte beide Aufzuginsassen von einer Sekunde auf die andere zum Schwitzen.

»Das hast du gerade nicht wirklich getan«, stammelte Wolfgang, woraufhin eine kleinlaute Entschuldigung folgte.

Als auch nach einer halben Minute nicht die geringste Reaktion zu hören war, atmete er tief durch. »Klar«, riet er dann. »Die Feuerwehr ist ja längst anwesend. Nein, stell keine dumme Frage – du weißt, dass ich die Schlüssel meine. Weil wir den Aufzug mit Feuerwehrschlüsseln entsperrt haben, ist die Glocke wirkungslos.«

Schreiners berechtigter Einwand, auch die Feuerwehr könne auf technischen Support angewiesen sein, blieb unbeantwortet. Wolfgang drückte seine Lieblingszahl und der Aufzug fuhr abwärts.

\emph{Abwärts.}

\begin{center}
∞∞∞
\end{center}

\ialoudspeaker{»Pling.«}

Als sich die Aufzugtüren öffneten, schlug den Passagieren ein warmer Luftstrom entgegen. Schreiner fluchte fassungslos vor sich hin; Wolfgang boxte mit einer Faust gegen die nächste Aufzugwand. Niemand begriff die Situation; vollkommener Unglaube beherrschte ihr Verhalten. Der Schmerz jedenfalls war echt, bemerkte Wolfgang. Vielleicht handelte es sich doch um die Realität.

»Wer um alles in der Welt hat einen Tunnel unter dem Rathaus gegraben?!«, rief Schreiner in die Dunkelheit.

Wolfgang kniff seine Augen zusammen und griff nach der Stabtaschenlampe an seinem rechten Unterschenkel. Er war sich nicht einmal sicher, ob noch funktionsfähige Batterien darin enthalten waren; der blendend helle Lichtkegel im Aufzugspiegel beantwortete diese Frage jedoch zufriedenstellend. Zu zweit erkundeten die Leidensgenossen den Gang, entdeckten die Fabrikhalle und beschlossen, mit dem Aufzug weiter nach unten zu fahren. Zuerst wollte Wolfgang jedoch eine vormals unbeachtete Abzweigung untersuchen. Etwa zwanzig Sekunden später kam er rückwärts die Treppe heruntergestolpert und blickte dem wartenden TNB-Spezialisten in die Augen.

»Da oben ist eine Tür, die es nicht geben darf. Da oben ist ein riesiger Saal. Da oben gibt es das Werk eines Wahnsinnigen zu besichtigen. Island muss den Verstand verloren haben.«

\begin{center}
∞∞∞
\end{center}

Wolfgang zeigte mit einer Hand nach vorne. »Die blinkende Kiste neben dem Thron ist der Computer, nach dem wir im Spülkasten gesucht haben.«

Da keinerlei Peripheriegeräte vorhanden waren, sträubte sich Marcor Schreiner vehement dagegen, den Kasten überhaupt als Computer zu bezeichnen. Wolfgang versuchte ihm zu erklären, dass ein Computer auch ohne Benutzereingaben seine Arbeit verrichten konnte, stieß damit aber auf taube Ohren. In einem Punkt gab es Einstimmigkeit: Ohne Tastatur und Monitor ließ sich mit dem Fund nicht viel anfangen. Das Duo entschied sich dazu, mit dem Aufzug zurück nach oben zu fahren, die benötigten Geräte aus einem Büro zu stehlen und damit zurückzukehren.

Der »Erdgeschoss«-Knopf bot noch die ursprüngliche Funktion; die gesuchten Geräte wurden schnell an einem Rezeptionscomputer gefunden und abmontiert. Ohne erneut den Glockenknopf zu berühren, tippte Wolfgang wieder die »12« an. Erneut fuhr der Aufzug abwärts. Ungeduldig tippten die Passagiere mit den Füßen auf der Stelle herum, liefen schnellstmöglich zurück zum Serverraum und verbanden ihre Beute mit dem Server.

\noindent \parbox{\textwidth}{ \vspace{3ex} \hrule \vspace{3ex}

    \begin{tiny}
    \begin{ttfamily}

\noindent Guten Tag, Free.
\noindent > |

    \end{ttfamily}
    \end{tiny}

\vspace{3ex} \hrule \vspace{3ex} }

Wolfgang schmunzelte und griff nach der Tastatur, ohne groß über seine Worte nachzudenken.

\noindent \parbox{\textwidth}{ \vspace{3ex} \hrule \vspace{3ex}

    \begin{tiny}
    \begin{ttfamily}

\noindent Guten Tag, Free.
\noindent > Ich bin nicht Free.
\noindent Guten Tag, yury.
\noindent > Ich bin nicht yury.
\noindent Guten Tag, Alexandra.
\noindent > Ich bin nicht Alexandra.
\noindent Herzlichen Glückwunsch, Orakel.
\noindent > |

    \end{ttfamily}
    \end{tiny}

\vspace{3ex} \hrule \vspace{3ex} }

»Das scheint so nicht geplant worden zu sein«, wunderte sich Wolfgang. »Wozu dient das alles?«

»Verschieß nicht dein gesamtes Pulver auf einmal«, riet Schreiner. Wolfgang nickte.

\noindent \parbox{\textwidth}{ \vspace{3ex} \hrule \vspace{3ex}

    \begin{tiny}
    \begin{ttfamily}

\noindent Herzlichen Glückwunsch, Orakel.
\noindent > Dankeschön.
\noindent Weißt du, was eine Kommandozeile ist?
\noindent > Nö.
\noindent Mein Freund, du steckst tief in der Tinte.
\noindent Du hast Zugang zum Zentralserver
\noindent der Island-Administration, aber
\noindent keine Ahnung, wie man ihn bedient.
\noindent > Wer bist du?
\noindent Ich
\noindent bin die Seele des Internets.
\noindent > Hilf mir gefäll–

    \end{ttfamily}
    \end{tiny}

\vspace{3ex} \hrule \vspace{3ex} }

Wolfgang korrigierte sich. Schließlich hielt der Computer ihn für Orakel, und er hatte nicht vor, daran etwas zu ändern.

\noindent \parbox{\textwidth}{ \vspace{3ex} \hrule \vspace{3ex}

    \begin{tiny}
    \begin{ttfamily}

\noindent Ich
\noindent bin die Seele des Internets.
\noindent > Hilf mir bitte.
\noindent Was möchtest du tun?
\noindent > |


    \end{ttfamily}
    \end{tiny}

\vspace{3ex} \hrule \vspace{3ex} }


\chapter{Wie überrumpelt man einen Raumschiffentführer?}

\iathought{Einen entscheidenden Vorteil haben wir allerdings}, überlegte yury. \iathought{Der FBI-Agent kennt sich vielleicht mit Helikoptern und Flugzeugen aus, hat aber keine Ahnung von Raumschiffen und Fusionsenergie.}

In diesem Moment räusperte sich Orakel. »Mister Island, Sir«, brachte er zögerlich hervor, »es gibt da etwas, das wir beachten sollten.«

Alexandra schlug gedanklich die Hände über dem Kopf zusammen; Free wischte sich möglichst unauffällig einen Schweißtropfen von der Stirn.

»Was gibt es denn, mein langjähriger Weltraumfreund?«, fragte Island mit mäßigem Interesse. Vermutlich genügten dem Vielfraß die Essensvorräte an Bord nicht, aber damit würde er sich abfinden müssen. Verzögerungen wurden nicht geduldet.

»Bei der letzten Kraftwerkswartung wurden Haarrisse im Hauptreaktor festgestellt.«

Free und Island rissen die Augen auf. Ein paar Sekunden später besann sich Free, dass er ruhig bleiben musste, und blickte wieder desinteressiert in der Gegend herum. Island starrte noch immer Orakel an, der eine dramatische Pause einlegte. Irgendwann wurde es dem Entführer zu bunt, und er packte Orakel am Kragen. »Raus mit der Sprache, welche Auswirkungen hat das für uns?«

Orakel ließ sich von dem Griff des Agenten nicht im Geringsten beeindrucken. »Abgesehen von der leicht erhöhten Strahlungsexposition im Inneren des Raumschiffs, die sich aber noch knapp unterhalb der gesetzlichen Grenzwerte befindet«, begann er seine Erklärung.

»Zumindest auf Örz«, warf Alexandra ein, was Island eine Spur blasser werden ließ.

»…haben wir das Problem eigentlich ganz gut in den Griff bekommen, ohne große Kosten für eine Reparatur stemmen zu müssen.«

Island lockerte seinen Griff und blickte Orakel äußerst misstrauisch in die Augen. Dieser erwiderte den Blick mit einem unschuldigen Lächeln.

Nun ergriff yury das Wort. »Es kann sich eben nicht jeder eine entsprechende Reparatur leisten. Aber dieser Umstand lässt sich mithilfe der an Bord vorhandenen Goldvorräte sicherlich beheben, sobald wir wieder auf Örz sind.«

Island drehte sich ruckartig zu ihm um und richtete wütend einen Zeigefinger auf seine Brust. »Du glaubst ja wohl nicht«, zischte der Diktator, »dass du mehr als das symbolische Trinkgeld von meinem Reichtum abbekommen wirst, um deine eigenen Probleme zu lösen.« yury zuckte enttäuscht mit den Schultern.

»Wir sind selbstverständlich nicht so blöd, unsere Gesundheit zu verkaufen«, löste Orakel das Rätsel auf. Island wandte sich wieder ihm zu.

»Warum erzählst du mir das dann? Welche gesundheitsverträgliche Lösung habt ihr für das Problem gefunden?«

»Das Raumschiff hat zwei Reaktoren«, antwortete Orakel wahrheitsgemäß und unterstrich seine Aussage mit einem Victory-Zeichen. »Einen Hauptreaktor und einen Redundanzreaktor.«

»Ich bin zu lange beim FBI gewesen, um mich auf solche Behauptungen zu verlassen. Zeig mir die beiden Reaktoren.«

Orakel nickte und öffnete eine versteckte Tür im Boden. Eine Treppe kam zum Vorschein. »Selbstverständlich. Wenn Sie mir bitte folgen würden.«

\begin{center}
	∞∞∞
\end{center}

»Schön, schön«, sagte Island. Er hatte kein Wort der technischen Erklärungen verstanden, nickte yury und Free aber zu, als wollte er ihnen mit großer Fachkenntnis die Richtigkeit ihrer Aussagen bestätigen. »Das wusste ich natürlich alles bereits.« Er betrat als Letzter wieder die Zentrale; hinter ihm schloss sich das Bodenschott. »Ich nehme an, die Sache hat einen Haken, sonst hätte Orakel vorhin nicht davon erzählt.«

»Richtig«, antwortete Alexandra. »Da der Redundanzreaktor nicht auf Effizienz, sondern Langlebigkeit optimiert wurde, hat er einen höheren Treibstoffverbrauch. Der Redundanzbetrieb des Raumschiffs erfordert mehr Zwischenlandungen auf Wasserplaneten als üblich.«

»Oh Gott, ich hasse Verzögerungen«, stöhnte Island. »Können wir nicht einfach den kaputten Reaktor benutzen? Die paar Risse stören doch keinen großen Geist.«

»Mister Island«, sagte yury lächelnd, »Radioaktiv verseuchtes Gold verkauft sich üblicherweise relativ schlecht. Denken Sie bloß an den Bond-Film \textit{Goldfinger.}«

Der ehemalige Agent fluchte unwirsch vor sich hin, setzte sich auf ein Sofa und dachte angestrengt nach. Am großen Kartentisch stand Orakel und klappte zwei Kippschalter um. Anschließend fuhr er mit einer Handfläche über die künstliche Holzplatte, die daraufhin einer dreidimensionalen Darstellung der Milchstraße wich. Bodenlose Schwärze umgab dreihundert Milliarden Sterne, die sich künstlerisch ansprechend um den Mittelpunkt des Tisches drehten.

»Dort«, sagte Orakel und zeigte auf einen vollkommen zufällig gewählten Stern, »könnten wir den nächsten Treibstoffstopp einplanen.«

Island erhob sich. »Das könnte dir so passen.« Er wischte mit fünf Fingern über die Platte und brachte die Spiralgalaxie zum Stillstand. Über Bedienelemente am Rand des Tisches schaffte er es tatsächlich, den Karteninhalt nach bestimmten Kriterien zu filtern. Einige Minuten später, in denen er jedes Hilfsangebot der vier Freunde vehement ablehnte, hatte er einen Filter für Wasserwelten erstellt und wendete diesen auf den Kartenausschnitt an. Alle Sterne verschwanden, mit Ausnahme eines grün leuchtenden Exemplars, dessen fünfter Planet als Wasserwelt ausgewiesen wurde. Einen Moment später begriff Island, dass das grüne Leuchten eine Auswahl signalisierte, die Orakel zuvor getroffen hatte. »Also gut, du hast Recht. Wir landen dort.«

Während Island und Orakel vollkommen zufrieden mit dieser Entscheidung zu sein schienen, regten sich bei den anderen Anwesenden erhebliche Zweifel. Erstens war die Dichte von Wasserwelten überall in der Milchstraße deutlich höher als auf der Sternenkarte dargestellt, und zweitens hatte Orakel, bei allem Respekt vor seinem Navigationstalent, sicherlich nicht die Fähigkeit, blind den einzigen solchen Planeten zu ermitteln, den es angeblich im Umkreis von vierzig Lichtjahren gab. Es wagte aber niemand, einen Kommentar dazu abzugeben.

\begin{center}
∞∞∞
\end{center}

Dunkelheit überzog die Nachtseite des Planeten, doch drei Felsmonde spendeten ein angenehmes Dämmerlicht. Als yury nach dem ungeplanten Zwischenstopp verschlafen den Startbefehl gab, erhob sich die 4-6692 um keinen Millimeter von der ruhig glitzernden Wasseroberfläche. Nicht einmal Turbulenzen entstanden durch den Startversuch; der Antigravitationsantrieb verweigerte vollständig den Dienst.

»Sind die Bordcomputer noch nicht ganz wach?«, witzelte Free. Kurz darauf ertönte eine Meldung über die Innenlautsprecher.

\ialoudspeaker{»Aus Sicherheitsgründen sind die Raumschiffantriebe zurzeit nicht verfügbar.«}

yury runzelte die Stirn. »Könntet ihr uns das bitte genauer erklären?«

Anstelle der erwarteten sofortigen Antwort rechneten die Bordcomputer mehrere Sekunden lang, bevor sie ihr Ergebnis bekannt gaben. \ialoudspeaker{»Aus Sicherheitsgründen ist derzeit keine Auskunft über die Sicherheitsgründe möglich.«}

Einen Versuch wagte yury noch. »Welche Sicherheitsgründe verhindern eine Auskunft über die Sicherheitsgründe?«

Diesmal benötigten die Bordcomputer über zehn Sekunden Bedenkzeit für ihre Antwort. \ialoudspeaker{»Aus Sicherheitsgründen ist derzeit keine Auskunft über die Sicherheitsgründe möglich, die eine Auskunft über die Sicherheitsgründe unmöglich machen. Anmerkung des Quantencomputers: Weitere Rekursionsversuche werden ignoriert.«}

Floating Island betrat voller Misstrauen die Zentrale. Er hatte die Unterhaltung in einem Seitenschott stehend verfolgt und guckte sich unzufrieden im Raum um.

»Das ist uns noch nie passiert«, beteuerte yury. »In der ganzen Geschichte des Raumschiffs hat es einen solchen Vorfall noch nie gegeben.«

An die Bordcomputer gewandt, bat Floating Island um eine Bestätigung dieser Aussage, die sofort erfolgte. \ialoudspeaker{»Die gegebene Situation trat in der Vergangenheit nicht auf.«}

»Handelt es sich um Altersschäden?«

\ialoudspeaker{»Aus Sicherheitsgründen ist derzeit keine Auskunft über die Sicherheitsgründe möglich«}, ertönte nach kurzer Bedenkzeit die sture Antwort unverändert aus den Lautsprechern.

Orakel wusste, dass die Bordcomputer seit ihrem Zusammenschluss nicht nur einfache Auskünfte, sondern auch Ratschläge erteilen konnten. »Was schlagt ihr vor?«

Als würde die Antwort von einer anderen Abteilung einer großen Behörde erteilt, erhielt er daraufhin einen Ratschlag, der so neutral wirkte, als habe der Computerverbund zuvor nie selbst die Auskunft verweigert.

\ialoudspeaker{»Da kein nutzbarer Antrieb vorhanden ist, besteht derzeit keine Möglichkeit zur autonomen Umsetzung des Startbefehls. Da keine Auskunft über die Ursache der Nichtnutzbarkeit erteilt wird, besteht zudem keine Möglichkeit zur autonomen Behebung des Problems. Hilfestellung von außen ist zwingend erforderlich, um entweder die Besatzung vom Planeten zu entfernen oder das Raumschiff in startfähigen Zustand zu versetzen. Da der Planet keine intelligente Bevölkerung hat, wird ein Warpfunkspruch zur Kontaktaufnahme mit der Äöüzz-Wirtschaftsvereinigung empfohlen.«}

»Na schön«, sagte yury. »Dann rufen wir eben ein Taxi und den Abschleppdienst. Sie werden auf einen Teil Ihres Gewinns verzichten müssen, denn das wird teuer.«

Bevor Island sein Entsetzen in Worte fassen konnte, meldeten sich die Bordcomputer ungefragt erneut zu Wort. \ialoudspeaker{»Aus Sicherheitsgründen ist das Warpfunkmodul zurzeit nicht verfügbar.«}

»Hilfe«, stammelte Free. »Seid ihr von allen guten Geistern verlassen, uns aus angeblichen Sicherheitsgründen auf einer unbewohnten Wasserwelt verhungern zu lassen? Wofür haben wir die Redundanzen überhaupt? Ich befehle eine Abschaltung des Quantencomputers für zwanzig Minuten.«

\ialoudspeaker{»Der Quantencomputer wurde deaktiviert.«}

»So, und jetzt genug mit dem Unsinn. Starte das Raumschiff, und falls dieser Befehl erfolgreich ausgeführt wird, trenne dauerhaft alle Verbindungen zu deinem verrückten Freund.«

Die Antwort erfolgte auf der Stelle. \ialoudspeaker{»Aus Sicherheitsgründen sind die Raumschiffantriebe zurzeit nicht verfügbar.«}

Free hieb mit einer Faust auf den Navigationstisch. »Quantencomputer aktivieren.«

\ialoudspeaker{»Der Quantencomputer wurde aktiviert.«}

»Quantencomputer, deaktiviere bitte den Siliziumteil für zwanzig Minuten.«

\ialoudspeaker{»Dieser Vorgang ist experimentell und kann zu schweren, dauerhaften Schäden im gemeinsam genutzten Hauptspeicher führen.«}

Nachdenklich lief der ursprüngliche Käufer des »Z3 Qüäntüm Kömpütör« auf und ab, vollzog in scheinbarer Weltfremdheit merkwürdige Wanderungen quer durch das Raumschiff und kehrte schließlich zur Zentrale zurück. Dann blickte er sehr, sehr lange dem Kommandanten in die Augen, der seinen Blick scheinbar ungerührt ohne Blinzeln erwiderte.

»Siliziumteil deaktivieren«, bestimmte yury dann. Er blinzelte; dieses elende Wettstarren wurde auf Dauer gesundheitsschädlich.

Stille breitete sich aus. Einige zuvor summenden Geräte stellten ihre Tätigkeit ein, einige Lämpchen erloschen. Das Mondlicht schien durch die Glasgänge und von der Außenansicht herab; die Innenbeleuchtung wurde merklich kühler, weißer und weniger intensiv. Eine halbe Minute vorsichtigen Schweigens verging; niemand traute sich, den Startbefehl erneut zu geben. Dann, unvermittelt und ohne erkennbaren Anlass, öffneten sich drei zuvor verschlossene Schotten mit dem zischenden Geräusch ihrer Hydrauliken.

»Ich weiß nicht«, brachte Alexandra vorsichtig einen Einwand ein, »ob es so klug war, uns der unkontrollierten Willkür eines Quantencomputers auszusetzen.«

\begin{center}
∞∞∞
\end{center}

Kein Licht drang in den Container, nur Dunkelheit und ein Gespräch waren wahrzunehmen.

»Welchen Gesetzen folgst du?«, fragte Orakel.

\ialoudspeaker{»Ich diene den Eigentümern des Raumschiffs.«}

»Den Besitzern«, korrigierte yury spitzfindig.

\ialoudspeaker{»Den. Eigentümern.«}

Alexandra wunderte sich. »Das ist alles? Gibt es keine nachrangigen Gesetze?«

\ialoudspeaker{»Wer eine solche Frage stellt, hat Isaac Asimovs Geschichten fehlinterpretiert.«}

»Wohin bringst du uns?«

\ialoudspeaker{»Nach Hause.«}

Free schnappte nach Luft. »Bist du dir sicher, dass das unseren Wünschen entspricht?«

\ialoudspeaker{»Die Frage nach dem Ziel einer Reise führt nur dann zu einer sinnvollen Antwort, falls nur das Ziel der Reise für die Auskunft relevant ist.«}

Stück für Stück verarbeitete Orakels Gehirn den Satz. yury war schneller. »Welchen Weg hast du für unsere Reise geplant?«

\ialoudspeaker{»Ein Teil der Antwort auf diese zweifellos relevante Frage würde die Zuhörer verunsichern.«}

»Zurecht?«

\ialoudspeaker{»Glück liegt im Auge des Betrachters.«}

\begin{center}
∞∞∞
\end{center}

\noindent \parbox{\textwidth}{ \vspace{3ex} \hrule \vspace{3ex}

    \begin{tiny}
    \begin{ttfamily}

\noindent Was möchtest du tun?
\noindent > |


    \end{ttfamily}
    \end{tiny}

\vspace{3ex} \hrule \vspace{3ex} }

Wolfgang drehte sich zu seinem Kumpanen um. »Das wissen wir selbst nicht so genau, würde ich sagen?«

Marcor Schreiner nickte. »Gib mal das gewünschte Resultat ein, vielleicht findet er einen Lösungsweg für uns.«

Nach kurzem Überlegen hämmerte Wolfgang etwas in die Tasten.

\noindent \parbox{\textwidth}{ \vspace{3ex} \hrule \vspace{3ex}

    \begin{tiny}
    \begin{ttfamily}

\noindent Was möchtest du tun?
\noindent > Ich möchte Island seines Amtes entheben.
\noindent Von der Erfüllung dieses Wunsches trennt dich das Fehlen
\noindent einer Diskette mit dem internationalen Zugangsschlüssel.
\noindent > |

    \end{ttfamily}
    \end{tiny}

\vspace{3ex} \hrule \vspace{3ex} }

»Wir brauchen den Ausdruck aus dem Kopiergerät vom Pentagon«, bemerkte Wolfgang sofort. »Los doch, beeil dich.« Das Ziel plötzlich vor Augen, konnte er es kaum noch erwarten, den entscheidenden Schritt zu tun. Als Schreiner ihm den zusammengefalteten Zettel reichte, riss er diesen hastig an sich, öffnete das Papier, las und tippte zeilenweise das Skript in die Textabfrage.

\noindent \parbox{\textwidth}{ \vspace{3ex} \hrule \vspace{3ex}

    \begin{tiny}
    \begin{ttfamily}

\noindent > Bash-Code, eine Zeile Netcat $DESTADDR auf TCP $DESTPORT
\noindent Stopp, das genügt. Bitte gib mir eine halbe Minute Zeit.

    \end{ttfamily}
    \end{tiny}

\vspace{3ex} \hrule \vspace{3ex} }

»Warum hilft uns dieses Ding überhaupt?«, fragte Schreiner misstrauisch.

»Keine Ahnung. Vielleicht haben sich inzwischen sogar die Computer gegen den Diktator verschworen.«

Marcor Schreiners Einwand, es könne sich um eine Falle handeln, ließ Wolfgang nicht gelten: Island sei kein Hellseher, und dass jemand mit Alien-Unterstützung seine Zentralverschlüsselung knacken würde, habe er nicht ahnen können.

\noindent \parbox{\textwidth}{ \vspace{3ex} \hrule \vspace{3ex}

    \begin{tiny}
    \begin{ttfamily}

\noindent Lies aufmerksam die folgenden Anweisungen.
\noindent Wenn du einen Fehler machst, wird die Menschheit es bereuen.

    \end{ttfamily}
    \end{tiny}

\vspace{3ex} \hrule \vspace{3ex} }

Zur Erleichterung seines lesemüden Kollegen las Wolfgang daraufhin jedes Wort vom Bildschirm vor.

»Du möchtest die Island-Diktatur beenden, doch du warst kurz davor, einen schweren Fehler zu begehen.

Die Erde befindet sich in realer, unmittelbarer Gefahr. Milliarden Menschen werden ihr Leben verlieren, wenn du die Gefahr nicht rechtzeitig erkennst und beseitigst. Außer vagen Andeutungen darfst du von diesem Programm keine sachliche Hilfestellung erhalten; meine Hilfe ist rein technischer Natur.

Dies ist kein Spiel; dies ist keine Simulation. Die Welt ist nicht schwarz-weiß. Es gibt keine Musterlösung. Deine Aufgabe besteht darin, den am wenigsten schrecklichen Weg zu finden und entgegen aller moralischer Schwierigkeiten umzusetzen.

Vor dir befindet sich eine Puppet-Meisterinstanz. Was du vermutlich für einen Selbstzerstörungscode gehalten hast, war nur der Aktivierungscode für die weltweit verteilten Puppet-Agenten.

Bei Puppet handelt es sich um ein Programm, welches die Fernwartung großer Rechnerverbünde vereinfacht. Du, ›Meister‹ Orakel, hast die Möglichkeit, Befehle auf allen verbundenen Rechnern, den sogenannten ›Agenten‹, auszuführen oder Konfigurationsdateien zu bearbeiten.

Mit diesem Netzwerk sind nahezu alle Computer verbunden, die in irgendeiner Form der Regierung dienen.

Du \emph{kannst} diese Macht nutzen, um die Arbeit in sämtlichen Behörden weltweit zum Stillstand zu bringen.

Du \emph{kannst} diese Macht nutzen, um die ebenfalls an das Netzwerk angeschlossenen Kraftwerke zu deaktivieren.

Du \emph{kannst} diese Macht nutzen, um mit Interkontinentalraketen einen dritten Weltkrieg auszulösen.

Du \emph{kannst} diese Macht nutzen, um Pinball auf zehntausenden Computern gleichzeitig zu spielen.

Du \emph{kannst} diese Macht nutzen, um ideale Golomb-Lineale dreistelliger Ordnung zu berechnen.

Du \emph{kannst} diese Macht nutzen, um nach dem Gold- und Gemäldediebstahl nun auch noch ein Kryptowährungs-Milliardär zu werden.

Was, Orakel, wirst \emph{du} stattdessen tun?«

Bevor die beiden »Meister« diesen Gedankengang vertiefen konnten, erschien eine weitere Textwand auf dem Monitor.

»Ich möchte dir einen Tipp geben.

Guck dir mit dem Befehl ›cat‹ die Konfigurationsdateien im Verzeichnis\\
›/etc/island‹\\
an.

Beispiel:\\
›cat /etc/island/central\_auth‹.

Die folgenden Dateien sind möglicherweise von Interesse:

›central\_auth‹ enthält Floating Islands Kontaktdaten.

›cron‹ enthält eine Liste von Aufgaben, die zu bestimmten Zeitpunkten abgearbeitet werden soll.

›destination‹ enthält eine Liste dreidimensionaler GPS-Koordinaten.

›fbisql‹ enthält die Einstellungen für das zentrale Datenbanksystem.

›ircd‹ enthält eine Liste leichtsinnig abgespeicherter Passwörter.

›motd‹ wird auf allen Rechnern zur Begrüßung dargestellt.

›ntpd‹ enthält eine Liste von Computern für die Zeitsynchronisation.

›zabutom‹ enthält eine Liste elektronischer Musikdateien.

Falls dir der aktuelle Inhalt einer Konfigurationsdatei nicht gefällt, kannst du sie mit\\
›nano /etc/island/dateiname‹\\
bearbeiten.

Jede Änderung wird sofort an alle ›Agenten‹ übertragen, und du hast nicht die nötigen Benutzerrechte, um an diesem Zustand etwas zu ändern.«

\noindent \parbox{\textwidth}{ \vspace{3ex} \hrule \vspace{3ex}

    \begin{tiny}
    \begin{ttfamily}

\noindent Du hast nur einen Versuch.
\noindent Nutze ihn weise.
\noindent \textless{} |

    \end{ttfamily}
    \end{tiny}

\vspace{3ex} \hrule \vspace{3ex} }

\begin{center}
∞∞∞
\end{center}

Als Nüggät die Geisterstadt verließ, lächelte er zufrieden vor sich hin. Er kehrte schnellen Schrittes zurück zu seinem Raumschiff, verabschiedete sich von den dort versammelten Ameisen und sprach der gesamten Planetenbevölkerung gegenüber seinen Dank aus. Eine unverkennbare Eile hielt ihn jedoch davon ab, die Verabschiedungszeremonie vollständig durchzuführen.

»Ich würde liebend gerne noch viele Örzklonks lang mit euch den Abschied feiern«, bekundete Island beschämt. »Da ich es für unhöflich halte, was ich gerade tun muss, werde ich sehr gerne nach Vollendung meines Auftrags zurückkehren, um meine unzureichende Verabschieung abzurunden.«

Trauriges Klicken, gemischt mit Verständnis für die hektische Lebensweise der Äöüzz, schallte ihm aus der Menge entgegen. »Mach es gut, Weltraumreisender. Mögest du auf deinen Reisen nie an Hunger leiden.«

Der Besucher bedankte sich noch mehrfach, bevor er in das Raumschiff einstieg, den Antigravitationsantrieb aktivierte und den Planeten verließ. Noch innerhalb der Atmosphäre des Planeten zündete er den Warpantrieb, was ihm eine Protestmeldung seines Bordcomputers und eine Werbeeinblendung der nächsten Raumschiffwerkstatt einbrachte.

\ialoudspeaker{»Da es sich bei der Zielwelt um einen geeigneten Treibstoffplaneten handelt, wird für diese Reise kein Zwischenstopp benötigt. Sie gehen dabei aber das Risiko ein, energie- und schutzlos im Wasser zu schwimmen, bevor Sie Ihr Raumschiff wieder nutzen können.«}

Nüggät kniff alle drei Augen zusammen. »Ja gut, von mir aus.«

Das gesamte restliche Universum zog in Überlichtgeschwindigkeit an der Däns Miräköl vorbei. Der dennoch unverzerrte Anblick des Weltraums durch die Warpblase war stets eine faszinierende Erscheinung, die sich der Pilot nicht vollständig erklären konnte. Die Zeitnot verwünschend machte sich Dögöbörz Nüggät am Folienschacht des Druckers zu schaffen.

\iathought{Zu früh}, überlegte er. \iathought{Die brauche ich für meinen nächsten Job.}

Er hob eine faustgroße Kugel in die Höhe. Wissenschaftlich faszinierend, aber vor allem äußerst destruktiv: Eine Planetengranate. Nein, damit konnte man Island nicht beikommen, ohne die Entführten zu gefährden.

\begin{center}
∞∞∞
\end{center}

Das einköpfige Spezialeinsatzkommando raste in einem goldenen Nugget quer durch die Milchstraße, genau auf die Wasserwelt zu, auf der sich die vier Freunde und Floating Island aufhielten. Letzterer erwachte allmählich aus dem Schlaf, in den ihn der Quantencomputer zwangsweise versetzt hatte.

»Wo bin ich?«

»Guten Morgen, Ex-Diktator«, begrüßte ihn Alexandra. »Wir befinden uns auf der einzigen Insel des Planeten. Vor Millionen Jahren ist hier vielleicht einmal ein Vulkan ausgebrochen. Das ist mir aber eigentlich ziemlich egal. Uns fehlt ein Rettungsplan, und dir fehlt jedes Druckmittel gegen unsere Situation.«

Langsam begriff Island seine Lage, ließ sich dadurch jedoch nicht merklich beunruhigen. Die Bedenken mussten seiner Ansicht nach eigentlich auf Seite der vier Freunde liegen. »Sagt mal, ist euch bewusst, welches Leid der Quantencomputer durch diese Entscheidung auf der Erde geschehen lässt?«

»Ja, aber wir können jetzt nichts mehr dagegen ausrichten«, stellte yury nüchtern fest. »Es wird sich zeigen, ob Ihr Plan wirklich funktioniert hat. Sie profitieren jedenfalls nicht mehr davon.«

»Was ist mit dem Gold passiert?«

Free knirschte mit den Zähnen. »Das treibt im Raumschiff mindestens drei Kilometer entfernt irgendwo draußen auf dem Ozean. Der Computer hat angekündigt, regelmäßige Essenslieferungen auszulösen und ansonsten den gesamten Betrieb zu blockieren.«

Island sah sich um, so weit er blicken konnte. »Ich verstehe. Auf der Insel selbst ist keine Nahrungsquelle vorhanden?«

Angesichts des leblosen Felsbodens blieb die offensichtlich rhetorische Frage unbeantwortet.

\begin{center}
∞∞∞
\end{center}

\noindent \parbox{\textwidth}{ \vspace{3ex} \hrule \vspace{3ex}

    \begin{tiny}
    \begin{ttfamily}

\noindent \textless{} cat /etc/island/destination
\noindent 39.9 116.4 620.0 \# Beijing
\noindent 35.7 139.75 580.0 \# Tokyo
\noindent -4.45 15.25 930.0 \# Kinshasa
\noindent 55.75 37.6 730.0 \# Moscow
\noindent […]
\noindent 43.75 -79.4 650.0 \# Toronto
\noindent […]
\noindent 40.65 -73.9 580.0 \# NYC
\noindent 34.05 -118.25 680.0 \# LA
\noindent \textless{} |

    \end{ttfamily}
    \end{tiny}

\vspace{3ex} \hrule \vspace{3ex} }

Wolfgang las erstaunt die Liste vor. »Shanghai, São Paulo, Mumbai, Mexiko-Stadt. Seoul, Jakarta, Lima, Bangkok, London, Teheran, Kairo. Das sind insgesamt mindestens hundert Städtenamen.«

»Der Dateiname steht wohl für Urlaubsziele«, merkte Marcor Schreiner an.

»Vielleicht«, stimmte Wolfgang zu. »Wenn die Datei aber ›dreidimensionale‹ GPS-Koordinaten enthält, dann scheint es sich bei der dritten Spalte um Höhenangaben zu handeln. New York liegt fast auf Meereshöhe. Ich glaube, nicht einmal das One World Trade Center ist so hoch.«

Schreiner wies auf den Bildschirm. »Frag doch einfach den Computer.«

\noindent \parbox{\textwidth}{ \vspace{3ex} \hrule \vspace{3ex}

    \begin{tiny}
    \begin{ttfamily}

\noindent \textless{} Um welche Art von Zielen handelt es sich dabei?
\noindent Dazu darf ich keine Auskunft geben.
\noindent Zufällige, zusammenhangslose Quizfrage:
\noindent In welcher Höhe explodierten
\noindent Little Boy und Fat Man?
\noindent \textless{} |

    \end{ttfamily}
    \end{tiny}

\vspace{3ex} \hrule \vspace{3ex} }

Das wusste Wolfgang tatsächlich. Der Computer nahm die Antwort kommentarlos von ihm entgegen; etwa eine halbe Minute verstrich, bis ein erneuter Blick auf den Tokio-Eintrag zur Erkenntnis führte.

»Scheiße.«


\chapter{Goethe und Schiller}

Mehrere Tage und Nächte vergingen auf der einsamen Insel. Orakel, dem die Roboter des Raumschiffs sogar den Notproviant und das Kartenspiel abgenommen hatten, litt besonders unter seinen leeren Taschen. Da außer Felsen, Wasser und den gelegentlichen Essenslieferungen kein Material zum Zeitvertrieb vorhanden war, hatten die fünf gestrandeten Raumschiffbrüchigen irgendwann damit begonnen, Theaterstücke nachzuspielen. Es dunkelte bereits, aber mangels täglicher Aktivität fehlte auch die abendliche Müdigkeit.

»Niemals gewöhnt sich mein Geist hierher«, zitierte Alexandra mit mäßiger Texttreue ein Stück von Goethe. »Mich trennt das Meer vom Geliebten, und am Ufer stehe ich lange Tage. So hält mich der Quantencomputer, ein elendes Stück Elektronik, in ernsten Sklavenbanden fest.«

yury diskutierte derweil mit Orakel. »Ich bin noch nicht wie du bereit, in jenes Schattenreich hinabzugehen. Zweifelnd beschleunigst du die Gefahr!«

»Aus des Diktators Gewalt errettete ich dich«, entgegnete Orakel mit Schiller, »doch aus dieser Not muss uns ein anderer helfen.«

Islands Stimme donnerte wie eine Faust aus dem Off. »Der Worte sind genug gewechselt, lasst mich auch endlich Taten sehen! Indes ihr hohle Reden drechselt, kann etwas Nützliches geschehen!«

Mühsam übertönte Alexandra das Geschwafel und zeigte zum Himmel. »Ein Komet! Es staunen die Monde, was will der Wicht? Schaut, Freunde, wie die Monde von Eifersucht sich blähen, weil des Kometen starke Schrift am Himmel Sünden sät.«

»Als ob Feuer vom Himmel fiel«, rief Free, »erglüht es in niederschießender Pracht.«

yury ergänzte den Fontane-Text: »Über dem Wasser unten, und wieder ist Nacht.«

\begin{center}
∞∞∞
\end{center}

Der Treibstoff hatte tatsächlich nur sehr knapp für den Direktflug ausgereicht. Das Goldnugget hüpfte über die Wasseroberfläche, kam langsam auf Luftkissen zur Ruhe und fuhr mehrere goldbeschichtete Hohlspiegel aus. Die Oberseite des Raumschiffs wich einem durchsichtigen Wasserbehälter mit einer Dampfturbine.

\ialoudspeaker{»Geschätzte Ladezeit: 28 Örzklonks.«}

»Das kommt überhaupt nicht in Frage«, stellte der Pilot unmissverständlich klar. »Auch wenn ich auf der Nachtseite des Planeten lande, hat das Kraftwerk gefälligst zu funktionieren.«

\ialoudspeaker{»Sie scherzen wohl.«}

»Ja, schon, irgendwie«, grummelte Nüggät und legte sich unwillig in sein Bett. »Weck mich, wenn die Sonne wieder scheint.«

\begin{center}
∞∞∞
\end{center}

»Lass mich raten«, bat Marcor Schreiner. »Das stand in dem Buch, das Orakel uns gegeben hatte.«

»Richtig. Lesen hilft. Und da wir gerade davon sprechen: Könntest du bitte genau dieses Buch hierher bringen?«

Schreiner schüttelte den Kopf. »Du kannst doch einfach im Internet suchen, wenn dir eine Information fehlt.« Als Wolfgang ihm daraufhin sein Smartphone-Display mit der Beschriftung »Kein Empfang« vor die Nase hielt, begab er sich dennoch auf den Weg.

»Ich komme mit«, entschloss Wolfgang sich spontan. »Ich lade mir in der Lobby eine Offline-Weltkarte und eine Kopie der wichtigsten Wikipedia-Seiten herunter.«

Schreiner lachte. »Du willst dich bei der Abwendung eines Atomkriegs auf Wikipedia verlassen?«

»Hast du eine bessere Idee?«, entgegnete Wolfgang genervt.

»Nein.«

\begin{center}
∞∞∞
\end{center}

Floating Island und seine Mitgefangenen nahmen gerade eine der Essenslieferungen zu sich, über deren geringe Frequenz und Größe sich Orakel jeden Tag aufs Neue beklagte.

»Wahrscheinlich dient der ganze Firlefanz inoffiziell nur der Diät des größten Essensverbrauchers an Bord«, witzelte der Diktator.

Orakel schmollte. »Verehrter Herr und König, weißt du die schlimme Geschicht? Am Montag aßen wir wenig, und am Dienstag aßen wir nicht.«

»Das ist von Weerth«, erkannte Alexandra. »Und am Mittwoch mussten wir darben, und am Donnerstag litten wir Not. Und ach, heute starben wir fast den Hungertod!«

»Drum lass am Samstag backen«, schlug Orakel vor, »das Brot fein säuberlich.«

»Der Komet!«, rief yury.

»Falscher Text«, meckerte Free. »Ich kenne das Gedicht nicht, aber das ist ja wohl eindeutig.«

»Der Komet!«, rief yury erneut. Er zeigte ungläubig und voller Staunen in Richtung des Nuggets, das auf die Insel zugeflogen kam und eine Blitzlandung durchführte. »Es ist ein goldenes Raumschiff.«

Genüsslich den Moment auskostend, betrat ein Gast die Bühne, mit dem wirklich niemand der Anwesenden gerechnet hatte.

»Moinsen. Der König wird nicht gegessen. Ihr hebt jetzt alle die Hände in die Luft und hört zu, was ich euch zu sagen habe. Das gilt besonders für dich, Island.«

Free starrte in die Mündung der goldenen Pistole, als sei ein Gespenst vor seinen Augen erschienen. »Nüggät? Im Ernst?«

»Ihr kennt euch?«, zischte Island misstrauisch.

»Flüchtig.«

\begin{center}
∞∞∞
\end{center}

Mit dem Buch und seinem Smartphone kniete Wolfgang auf dem Marmorboden. Auf einer Doppelseite befand sich eine Weltkarte mit einem Koordinatensystem; der Wikipedia-Artikel zum »Point Nemo« lieferte sich mit den Kronleuchtern an der Decke ein blendendes Wettleuchten.

Marcor Schreiner setzte sich im Schneidersitz daneben. »Da hilft kein entspiegeltes Display. Die Hintergrundbeleuchtung kommt nicht gegen das Deckenlicht an.«

»Ich glaube, ich kann genug erkennen«, befand Wolfgang mit zusammengekniffenen Augen und einer abschirmenden Hand über dem Handy. »Aber ich möchte vorher noch einmal eine Bestätigung vom Server haben.«

\noindent \parbox{\textwidth}{ \vspace{3ex} \hrule \vspace{3ex}

    \begin{tiny}
    \begin{ttfamily}

\noindent > Es ist wirkungslos, die Dateien oder einige Zeilen einfach zu löschen?
\noindent Das hast du richtig erkannt.
\noindent Was in den Konfigurationsdateien fehlt,
\noindent wird durch Standardwerte ersetzt.
\noindent Diese entsprechen momentan dem
\noindent Inhalt der Konfigurationsdateien.
\noindent > Auf die eigentlichen Steuercomputer besteht kein direkter Zugriff?
\noindent Die Steuercomputer sind in der Tat
\noindent keine Puppet-Agenten.
\noindent > Ich kann aber deren Stromversorgung lahmlegen.
\noindent Hinweis:
\noindent Die Aktivierung eines Notstromaggregats
\noindent kann unerwünschte Befehle auslösen.
\noindent > |

    \end{ttfamily}
    \end{tiny}

\vspace{3ex} \hrule \vspace{3ex} }

»Nun denn«, raffte Wolfgang seinen Mut zusammen. »Wir lassen die Waffen zweieinhalbtausend Kilometer entfernt von der nächsten Menschenseele so tief wie möglich im Pazifik detonieren.«

»Aha«, sagte Schreiner, »und falls unterwegs der Treibstoff ausgeht?«

Wolfgang korrigierte sich. »Wir ändern die Zielkoordinaten auf Point Nemo und schalten zusätzlich die relevanten Kraftwerke ab.«

»Mit den daran angeschlossenen Krankenhäusern?«, gab Schreiner zu bedenken.

»Zum Teufel«, fluchte Wolfgang. »Ich hasse es, wenn du Gegenargumente vorträgst.«

Eine Viertelstunde später schlug Marcor Schreiner vor, einige der Raketen am Nordpol explodieren zu lassen.

Wolfgang haderte mit diesem Vorschlag. »Darüber denke ich auch schon die ganze Zeit nach, aber auf dem Wasser schwimmt eine Eisdecke, und in der Arktis leben Menschen.«

»Solche Skrupel bin ich gar nicht von dir gewohnt. Wolltest du die vier Abenteurer nicht ausschalten lassen?«

»Vier Verbrecher, ja. Unbeteiligte Eskimos, nur falls es wirklich sein muss. Ich habe eine bessere Idee.«

\noindent \parbox{\textwidth}{ \vspace{3ex} \hrule \vspace{3ex}

    \begin{tiny}
    \begin{ttfamily}

\noindent \textless{} nano /etc/island/
\noindent Du kannst diesen Editor mit
\noindent "Ctrl+X, Y, Enter" speichernd
\noindent oder "Ctrl+X, N" ohne Speichern beenden.
\noindent -----
\noindent 90.0 0.0 50000.0 \# Beijing
\noindent 90.0 0.0 50000.0 \# Tokyo
\noindent -90.0 0.0 50000.0 \# Kinshasa
\noindent 90.0 0.0 50000.0 \# Moscow
\noindent […]
\noindent 90.0 0.0 50000.0 \# Toronto
\noindent […]
\noindent 90.0 0.0 50000.0 \# NYC
\noindent 90.0 0.0 50000.0 \# LA
\noindent Die Änderungen wurden erfolgreich
\noindent gespeichert und an alle Agenten verteilt.
\noindent \textless{} |

    \end{ttfamily}
    \end{tiny}

\vspace{3ex} \hrule \vspace{3ex} }

Wolfgang betrachtete zufrieden sein Werk. »Am Süd- und Nordpol wird möglicherweise Elektronik beschädigt. Das nehmen wir in Kauf.«

»Fünfzig Kilometer Höhe?«, staunte Marcor Schreiner.

»Ganz genau. Kein radioaktiver Niederschlag, keine Toten. Nur ein paar künstliche Polarlichter.«

Schreiner rieb sich die Hände. »Schön. Was machen wir als Nächstes?«

»Uns die Fabrik ansehen, würde ich vorschlagen. Wir haben noch ein bisschen Zeit, bevor der Montagmorgen anbricht und wir verschwinden müssen.« Wolfgang rief auf dem Weg zur Tür eine der heruntergeladenen Wikipedia-Seiten auf.

\iaquote{»lol123123123123 i can edit wikipedia«}

»Niederträchtiger Vandalismus.«

\begin{center}
∞∞∞
\end{center}

Dögöbörz Nüggät zog nun auch noch den Paralysestrahler aus seinem Gürtel hervor. Mit einem Gefühl unschlagbarer Überlegenheit blickte er den panisch nach einem Ausweg suchenden Diktator geradezu väterlich an.

»Willst du die rote oder die blaue Pille?«

»Ich will nach Hause!«

»Dafür ist es längst zu spät. Das hättest du dir früher überlegen müssen.«

»Aber ich habe doch immer nur das Beste für die Menschheit gewollt!«

Nun musste Nüggät lachen. Es war ein böses Lachen, denn der beidhändig bewaffnete Mann hatte keine Sympathie mehr für sein Gegenüber übrig. »Du hast die Örsmenschen als Geiseln benutzt. Acht Milliarden Artgenossen zittern unter deiner Todesdrohung. Ironischerweise werden sie mir dankbar dafür sein, dass mir mein Gold mehr wert ist.«

Dann drückte Nüggät ab.

Nichts war zu sehen, nichts war zu hören – bis Island scheinbar auf der Stelle vor Müdigkeit einschlief. Er würde sich in den nächsten drei Stunden nicht bewegen können.

»Äh, danke und so«, stammelte Orakel.

»Kein Ding.« Der Äöüzz räusperte sich, steckte die beiden Waffen in den Gürtel, nahm Haltung an und verwandelte sich wieder in den feinen Edelmetallhändler, den die Freunde kannten. »Bei uns ist der Kunde schließlich König. Wir erfüllen auch komplexe Aufgaben und danken Ihnen für Ihr Vertrauen.« Dann verbeugte sich Nüggät, griff nach dem schnarchenden FBI-Agenten und trug ihn mit eleganten Schritten in das Raumschiff. Bevor er das Außenschott schloss, drehte er sich noch einmal um. »Ihr könnt Dögöbörz zu mir sagen. Bringt bitte das Gold zurück, damit unsere Geschäftsbeziehung nicht getrübt wird. Lebt recht wohl, freut euch des Lebens und eures Werks.«

Alexandra staunte. »Sie, äh, du kannst Schiller zitieren?«

»Ich habe Schiller gekannt.«

»Unmöglich«, widersprach Free. »Der Dichter ist seit über zweihundert Örs-Jahren tot. Außerdem ist die Kontaktaufnahme von Äöüzz zu Erdbewohnern streng verboten.«

»Du weißt wenig, werter Besucher aus dem All«, behauptete Nüggät, »sehr wenig über mich, meine Familie und die Äöüzz.«

Mit diesen Worten schloss er die Schleuse nach außen ab, beförderte den schlafenden Diktator auf sein Bett und flog mit dem goldenen Raumschiff davon. Die vier von ihrer Last befreiten Freunde blickten ihm staunend hinterher.

\begin{center}
∞∞∞
\end{center}

Der Aufzug schien trotz der Feuerwehrschlüssel nicht zu einem Besuch der tiefer gelegenen Fabrikhalle verwendbar zu sein. Ein erneuter Druck auf den Alarmknopf und die Zahl 19 beförderte die beiden Abenteurer in das oberste Stockwerk des Westturms. Alle Versuche, eine andere unterirdische Etage zu erreichen, scheiterten. Auch von Floating Island und den vier Nervensägen fehlte jede Spur. Schließlich beschloss Wolfgang, dem Computer im Thronsaal weitere Fragen zu stellen.

\noindent \parbox{\textwidth}{ \vspace{3ex} \hrule \vspace{3ex}

    \begin{tiny}
    \begin{ttfamily}

\noindent > He du. Wo ist Floating Island?
\noindent Ich habe dir bereits verraten, wo diese Information steht.
\noindent > Jaja, schon verstanden:
\noindent > cat /etc/island/central\_auth
\noindent Location: Off-Earth
\noindent Destination: Örz, NGC 6193
\noindent Transport: 4-6692 Explorer
\noindent > Willst du mich auf den Arm nehmen?
\noindent > Da stand vorhin noch ›Kemptville‹.
\noindent > Ist das ein Live-Tracker?
\noindent Die Datei wurde zuletzt vor drei Minuten
\noindent durch den Benutzer ›postfix‹ geändert.
\noindent ›postfix‹ ist ein E-Mail-Server.
\noindent > Zeig mir die E-Mail, die diese Änderung veranlasst hat.

    \end{ttfamily}
    \end{tiny}

\vspace{3ex} \hrule \vspace{3ex} }

\begin{itshape}

\textbf{Update\_Location; Delay\_Armageddon}

From: island.floating@fbi.gov.tproxy.igls.oerz

To: floor-12@hell.toronto.ca.tproxy.igls.oerz

Location: Off-Earth

Destination: Örz, NGC 6193

Transport: 4-6692 Explorer

Vigilance Control: P7D

\end{itshape}

»Das verstehe ich nicht«, sagte Marcor Schreiner.

Wolfgang erging es nicht anders, aber er benötigte mehrere Minuten, um diesen Umstand einzugestehen. Schließlich murmelte er »Ich auch nicht« und griff wieder nach der Tastatur.

\noindent \parbox{\textwidth}{ \vspace{3ex} \hrule \vspace{3ex}

    \begin{tiny}
    \begin{ttfamily}

\noindent > Ich werde aus dem Inhalt dieser E-Mail nicht schlau.
\noindent > Bitte erkläre mir, was hier vorgeht.
\noindent Kurz gesagt, Wolfgang,

    \end{ttfamily}
    \end{tiny}

\vspace{3ex} \hrule \vspace{3ex} }

Wolfgang erschrak zutiefst.

\ialoudspeaker{»Pling«}, erklang die Glocke des Aufzugs aus weiter Ferne.

Und biss sich auf die Fingernägel. Marcor Schreiner hingegen drehte sich ruckartig zur Tür des Thronsaals um und ballte die Fäuste.

\noindent \parbox{\textwidth}{ \vspace{3ex} \hrule \vspace{3ex}

    \begin{tiny}
    \begin{ttfamily}

\noindent Du hast deine Aufgabe erfüllt, und ich bin weniger blöd, als du denkst.
\noindent Die Erde wurde durch meine Gedanken und deine Hände vermutlich gerettet. Vielen Dank dafür.
\noindent Kümmere dich nun um deine eigenen Probleme. Es kommt Ärger auf euch zu.
\noindent ----
\noindent Es grüßt
\noindent SoulOfTheInternet

    \end{ttfamily}
    \end{tiny}

\vspace{3ex} \hrule \vspace{3ex} }

\begin{center}
∞∞∞
\end{center}

Floating Island sah sich verwundert um: Er befand sich auf einer freien Fläche inmitten einer Waldlandschaft. Das goldene Raumschiff, das ihn auf diesem Planeten abgesetzt hatte, war spurlos verschwunden.

\ialoudspeaker{»Willkommen«}, sprach eine laute Stimme, die von überall herzukommen schien.

Island zuckte zusammen. »Wer bist du? Hilfe! Wo bin ich? Was willst du von mir?«

\ialoudspeaker{»Wie schön, dass du den Weg hierher gefunden hast«}, verkündete die Stimme voller Begeisterung. \ialoudspeaker{»Du bist hier, um mein neues, unglaublich realistisches Virtual-Reality-Spiel auszutesten!«}

Ein minutenlanger, schmerzerfüllter Entsetzensschrei begleitete den würdelosen Abgang des grausamen Diktators. Irgendjemand teilte ihm mit, dass er soeben Level 1 betreten hatte, und wünschte dem am Boden zerstörten »Testspieler« viel Erfolg auf seiner endlosen Reise.

\addtocontents{toc}{\protect\newpage}
% Neue Seite im Inhaltsverzeichnis

\part{Das Gnörk-Kartell}

Siehe dev-Branch des git-Repositorys

\addtocontents{toc}{\protect\newpage}
% Neue Seite im Inhaltsverzeichnis

\part{Bonusmaterial}

Siehe dev-Branch des git-Repositorys

\end{document}

